The main objective of this work was to propose a viable and secure
username/password based user identification scheme in structured P2P
networks using secure routing and building of trust between nodes.
With that in mind, we presented a quasi-identification service built on top of a DHT. It
replicates the user information needed to maintain the identification system
over multiple peers and retains identical results from a qualified majority to
certify that user public information corresponds to the correct registered
user. Also, being implemented over a DHT makes the system portable for the most
commonly used P2P networks.\\
As in other P2P systems, a 100\% secure implementation cannot be achieved
(section~\ref{sec:pseudo-secure}). Still, as seen in the
protocols evaluation (section~\ref{sec:evaluation}), an acceptable probability of error on the order
of $10^{-14}$ can be achieved with a leafset size of $L = 32$. This means that
for each $100,000$ billion transactions being made, only one would fail. Also, the system
maintains a scalable cost when the size of nodes $N$ in the DHT increases.
With this we have proven that our proposed solution is a viable solution to the user
identification problem.
This was possible thanks to the use of a reputation system and trusted
nodes management, which mitigated the effectiveness of malicious node attacks in
the network. A reputation system can also efficiently mitigate the effect of
collusion attacks.\\
To secure the system from Sybil attacks, we assume that the generation of the nodeIDs can
be verified by any node in the network, for instance the bootstrapping node or
the node with the closest nodeID. That way, a node can easily verify the correctness of the nodeID
of each node that wants to join the network, and if they do not match, the join procedure can be prevented.
We also use computational challenges to mitigate the problem of a
node registering multiple users with the system. We also mention the use of
additional challenges during user registration, like
CAPTCHA~\cite{von2003captcha}, to further increase resilience against Sybil
attacks.\\
%One weakness of the system security is the passwords itself. 
While our system uses a username-password scheme for the user identification,
we know that passwords presents serious security problems as the average user tends
to use simple keywords and/or the same password in more than one service.
A definitive solution to this problem has not
been found yet. While a strict password creation policy can mitigate this
problem, at the same time, it can also increase cases of forgotten passwords.
Our proposed system does not implement password recovery mechanisms to reduce the
risks to the security of the system~\ref{sec:risks_password_recovery}.


%- el sistema no cuenta con formas de recuperar la contraseña en caso de pérdida u olvido.
%- 


%We are currently working on developing an identification service that
%allows password recovery operations while not compromising user identity
%through the integration of scalable trust mechanisms.

%\textbf{Acknowledgements.} The work presented was partially funded by a CONICYT scholarship.

\section{Future work}


User identification systems in P2P networks are a very interesting problem.
%To obtain similar functionalities to the well-know identification systems used
%over the internet, more work will be needed. 
While the proposed system is resilient to many of the most common attacks
and issues related to badly chosen user passwords, the danger to the system can
be further reduced by enforcing a strict password creation policy, though there are still problems related to the use of
passwords in identification systems. 
Also, more research will be needed on new functionalities and addressing the
problem of forgotten passwords and offline guessing attacks.
We have provided an initial discussion of the security properties of our
protocols here, and future work should include a thorough security analysis of
the proposed solutions.

Another issue that was not addressed is the bootstrapping of the P2P system.
When the P2P system starts building up, there are no real trusted
nodes and the services rely on just the few nodes that comprise the
system. During that time there is a higher probability of failure of the protocols, and the system is
more vulnerable to byzantine node attacks. Further research is still needed on
that matter.

%La identificación de usarios es un tema muy interesante cuando se aplica a
%redes P2P. Son pocos los trabajos que toman este tema para desarrollar
%soluciones que se asemejen a los servicios más conocidos actualmente en la
%internet. Después de realizar este trabajo, destacamos las siguienes lineas
%investigativas para el futuro desarrollo de nuestro trabajo:

%\item Proponer nuevas formas de comprobar la identidad de los usuarios dentro
%del sistema de identificación. Si bien el uso de una contraseña es
%ampliamente aceptado para muchas aplicaciones, podrían verse metodos
%alternativos que mejoren su seguridad.
%

%
%\item El manejo de la confianza en sistemas distribuidos todavía puede seguir
%desarrollandose. Lo
%


