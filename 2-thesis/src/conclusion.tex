The main objective of this work was the implementation of a secure
username/password based user identification scheme in structured P2P
networks using secure routing and building of trust between nodes.
With that in mind, we presents a quasi-identification service on top of a DHT. It
replicates the user information needed to maintain the identification system
over multiple peers and retains identical results from a qualified majority to
certify that a user public information corresponds to the correct registered user.
As other P2P
systems, a 100\% secure implementation cannot be achieved
(section~\ref{sec:pseudo-secure}). Still, as seen in the
protocols evaluation (section~\ref{sec:evaluation}), an acceptable probability of error of the order
of $10^{-14}$ can be achieved with a leafset size of $L = 32$. Also the system
maintains a cost scalable when the size of nodes $N$ of the DHT increases.
This was possible thanks to the use of a reputation system and trusted
nodes management, which mitigated the effectivity of malicious nodes attacks in
the network.
An over DHT implementation allows let the system be easily adaptable  to the
most commonly used P2P networks.\\


While the proposed system is resilient to many of the most common attacks,
and issues related to the bad choice of user passwords can be mitigated enforcing a
strict password creation policy, there are still problems related to the use of
password in identification systems.
We are currently working in developing an identification service that
allows password recovery operations while not compromising the user identity
through the integration of scalable trust mechanisms.

%\textbf{Acknowledgements.} The work presented was partially funded by a CONICYT
%scholarship.
