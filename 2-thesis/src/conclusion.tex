The main objective of this work was to propose a viable and secure
username/password based user identification scheme in structured P2P
networks using secure routing and building of trust between nodes.
With that in mind, we presented a quasi-identification service built on top of a DHT. It
replicates the user information needed to maintain the identification system
over multiple peers and retains identical results from a qualified majority to
certify that user public information corresponds to the correct registered user.
As in other P2P
systems, a 100\% secure implementation cannot be achieved
(section~\ref{sec:pseudo-secure}). Still, as seen in the
protocols evaluation (section~\ref{sec:evaluation}), an acceptable probability of error on the order
of $10^{-14}$ can be achieved with a leafset size of $L = 32$. Also, the system
maintains a scalable cost when the size of nodes $N$ in the DHT increases.
This was possible thanks to the use of a reputation system and trusted
nodes management, which mitigated the effectiveness of malicious node attacks in
the network.
An over DHT implementation makes the system portable for the most commonly used P2P networks.\\


While the proposed system is resilient to many of the most common attacks
and issues related to badly chosen user passwords, the danger to the system can
be further reduced by enforcing a strict password creation policy, though there are still problems related to the use of
passwords in identification systems.
%We are currently working on developing an identification service that
%allows password recovery operations while not compromising user identity
%through the integration of scalable trust mechanisms.

%\textbf{Acknowledgements.} The work presented was partially funded by a CONICYT scholarship.
