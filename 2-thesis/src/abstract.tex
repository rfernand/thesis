
% TODO: THIS IS FROM THE MEMORY, NEED CHANGE

P2P services are robust, scalable and self-organized by nature, but have a
completely different architecture with new problems and unique requirements.
There are proposals to implement different user identification systems, but
this systems would not have the same features
 that users are accustomed to having today in commonly
viewed services, i.e, user identification based in username and password. While an
early approach exists to solve the problem, it does not contemplate the existence
of malicious nodes, as it lacks the mechanism to secure the identification
process.

 We thoroughly investigate the requirements and features of a secure
identification scheme, along with the challenges facing a P2P implementation.

 %The main features discussed are (a)
%the users sign in, (b) users posts, (c) users contacts list, (d) groups and organizations
%among users, (e) privacy settings and (f) integration with other services and
%applications.


%With more than 1,5 billion active users, social networks are one of the most
%popular Internet services.
%The  problem is that social networks services of today are not scalable, 
%having very high maintenance costs.
%On the other hand, P2P services are robust, scalable and self-organized by nature, but have a
%complete different architecture with new problems and unique requirements. 
% Therefore, this document focus on finding the existing problems in the
%implementation of a P2P social networking service.
% We thoroughly investigate the requirements and features of social networks, along with the
%challenges facing a P2P implementation. The main features discussed are (a)
%the users sign in, (b) users posts, (c) users contacts list, (d) groups and organizations
%among users, (e) privacy settings and (f) integration with other services and
%applications.
%After the analyzis of these features with their current solutions,
%we discuss the problems that hinder the satisfaction of the system requirements, ending with the identification of today's main challenges.
%Among the problems identified are (1) deficiencies in the
%implementation of complex search, which do not allow real-time
%feedback and need to improve the quality and accuracy of the results obtained,
%(2) lack of proposals for enhanced security system for these systems,
%especially related to selfish users control and encryption systems,
%and (3) vulnerabilities in the entry mechanisms of the system and (4) management of
%selfish users.\\


%\textbf{Keywords:}
{\bf Keywords:} P2P, user identification, distributed systems.
