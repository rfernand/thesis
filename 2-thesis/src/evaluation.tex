In this section we present a probability assessment of
the algorithms used in our quasi identification system, and show that it is highly
improbable to fail.

In this section, $p$ represents the probability that a single node is
malicious, and $N$ the total number of nodes in the DHT. We conduct every
assessment with two different values for $p$, being $p = 0.3$ and $p = 0.05$.
The former corresponds to the limit above which our relaxed version of the
byzantine agreement collapses. The latter corresponds to the value most
commonly found in the literature about malicious attacks in P2P
systems~\cite{p2p_certification}.
%, as well as a set of simulations and performance
%results.

%%%Theoretical evaluation

\section{Securing the nodes of a Leafset}
\label{sec:eval_leafset}
  
  \subsection{\textit{Probability of failure}}
  
  \begin{table}
    \centering
    \footnotesize
    \begin{tabular}{|c|c|c|}
      \cline{2-3}
      \multicolumn{1}{c|}{}&  \multicolumn{2}{c|}{\textbf{Probability to fail}} \\ \cline{2-3}
      \hline
      \textbf{Size of Trusted Set (L)} & \textbf{p = 0,3} & \textbf{p = 0,05} \\
      \hline \hline
      8 &  $0.0081$              & $6.25 \times 10^{-6}$  \\
      \hline
      16 & $6.56 \times 10^{-5}$ & $ 3.9 \times 10^{-11}$ \\
      \hline
      32 & $4.3 \times 10^{-9}$  & $ 1.52 \times 10^{-21} $  \\
      \hline
    \end{tabular}
    \caption{Probability of failure when securing a leafset}
    \label{tab:p_leafset}
  \end{table}
  
  Table~\eqref{tab:p_leafset} shows the probability of failure when securing the
  nodes of a leafset. The algorithm fails to get the leafset of a given node
  $K_{root}$ if $\frac{L}{2}$ consecutive nodes are malicious in a leafset. This
  probability is given by
  
  \begin{equation} \label{eq:p_leafset}
    P= p^{\frac{L}{2}}
  \end{equation}
  
  \subsection{\textit{Message complexity}}

      %As seen in~\cite{p2p_certification}
      According to the equation~\eqref{eq:p_leafset}, the probability of having
more than 4 consecutive malicious nodes when using the Diversity Trusted
Routing is around $6.25 \times 10^{-6}$. We can thus consider it highly
improbable that more than 4 routing attempts will be necessary  to secure a
leafset of a node near $K_{root}$. An upper bound of the cost to get the
leafset of $K_{root}$ is given by
      
      \begin{align} \label{eq:p_leafset}
        n &= \underbrace{O(log_{2b}(N)) + 4 \times O(log_{2b}(N))}_\text{First
and Diversity Routing} + \underbrace{Q+4}_\text{Direct IP} \\
        n &= 5 \times O(log_{2b}(N)) +  Q+4 
      \end{align}
      
      where $Q$ is the maximum number of tries to get a leafset from a node
that belongs to the leafset of $K_{root}$. The probability of not being able to
retrieve a leafset from a node that belongs to the leafset of $K_{root}$ after
$Q = 16$ tries for a leafset size $ L = 16$ is $1.52 \times 10^{-21} $. This is
highly improbable, so we can consider that $L/2 = 8$ is a reasonable upper
bound for $Q$. 
      In the best case, the cost is reduced to $n = O(log(N))$ when the root
node $K_{root}$ is honest. Thus, the message complexity of the chosen leafset
securing algorithm~\cite{p2p_certification} easily scales when the size of the
DHT increases.



%\section{Normal service request}
\label{sec:eval_service_request}
  \begin{table}
    \centering
    \footnotesize
    \begin{tabular}{|c|c|c|}
      \cline{2-3}
      \multicolumn{1}{c|}{}&  \multicolumn{2}{c|}{\textbf{Probability to fail}} \\ \cline{2-3}
      \hline
      \textbf{Size of Trusted Set (L)} & \textbf{p = 0.3} & \textbf{p = 0.05} \\
      \hline \hline
      8 &  $0.188$ & $6.64 \times 10^{-5}$ \\
      \hline
      16 & $0.079$ & $6.57 \times 10^{-8}$  \\
      \hline
      32 & $0.016$ & $8.24 \times 19^{-14}$  \\
      \hline
    \end{tabular}
    \caption{Probability of failure when doing a normal service request}
    \label{tab:p_service_request}
  \end{table}
  
  \begin{enumerate}
    \item{\textit{Probability of failure:}}
    The user registration between a node $A$ and $S$ can fail (a) if $S$does
not respond to $A$ during the registration request or during the registration
progress, or (b) if $S$ does not send the ACKs to $A$ to terminate the user
registration. In both cases, this corresponds to the probability that more than
$L/2$ nodes of $S$ are malicious. The probability of encountering exactly $k$
malicious nodes among $L +1$ is given by the binomial distribution

    \begin{equation}
      P_{K malicious} = \begin{pmatrix} L+1 \\ k\end{pmatrix} p^k (1-p)^{L+1-k}
    \end{equation}

    where $p$ is the probability that a single node is malicious. Hence the
probability of facing at most k malicious nodes is 

    \begin{equation}
      P_{\leq k} = \sum_{i=1}^{k} \begin{pmatrix} L+1 \\ k\end{pmatrix} p^i (1-p)^{L+1-i}
    \end{equation}

    Therefore, the probability of facing more than $k$ malicious nodes among
$n$ is given by

    \begin{equation}
      P_{\ge k} = P_{\leq n} - P_{\leq k}
    \end{equation}

    and the probability that $S$ will not respond is

    \begin{align}
      P_{\ge L/2} &= P_{\leq L+1} - P_{\leq L/2} \\
      &= \sum_{i=1}^{L+1} \begin{pmatrix} L+1 \\ i\end{pmatrix} p^i (1-p)^{L+1-i}
      - \sum_{i=1}^{L/2} \begin{pmatrix} L+1 \\ i\end{pmatrix} p^i (1-p)^{L+1-i}
    \end{align}

    %% TODO: figure with the likeliness of dealing with an increains gnumber k of malicious nodes among a leafset comprising 32 nodes.

    Then the probability that $S$ will not respond to the user registration
request or to not send the final ACKs is

    \begin{align}
      P_{AS} &= (1- P_{\ge L/2}) P_{\ge L/2} +  P_{\ge L/2} \\
      P_{AS} &= 2P_{\ge L/2}) + P^2_{\ge L/2}
    \end{align}


    Table~\eqref{tab:p_account_registration} shows the probability of failure
between $A$ and $S$, for a leafset size of $L = \{8,16,32\}$.

    
    \item{\textit{Message complexity:}}
    First, $A$ must get the leafset of $S$. The associated cost is $n = 5
\times O(log_{2b}(N)) + Q + 4$ (as seen in section~\ref{sec:eval_leafset}).
    The number $n$ of message inherent to the transaction itself is given by

    \begin{align}
      n &= \underbrace{2(L+1)}_\text{Init} \underbrace{r(L+1)}_\text{Data} \underbrace{L+1}_\text{ACKs}\\
      n &= (r+3)(L+1)
    \end{align}
    where $r$ corresponds to the number of data messages sent by $A$ to $S$,
and fully depends on the transaction. The total cost is then

    $$
      n_{total} = 5 \times O(log_{2b}(N)) + Q + 4 + (r+3)(L+1)
    $$    
    The total cost only depends on the size $L$ of the leafset, which is a
constant, and $O(log(N))$. In the best case, 

    $$
      n_{total} = O(log_{2b}(N)) + (r+3)(L+1)
    $$
    Therefore, the cost of a transaction between $A$ and service $S$ remains
scalable when the size $N$ of the DHT increases.

  \end{enumerate}


\section{Account registration}
\label{sec:eval_account_registration}
  % TODO Recalculate table
  \begin{table}
    \centering
    \footnotesize
    \begin{tabular}{|c|c|c|}
      \cline{2-3}
      \multicolumn{1}{c|}{}&  \multicolumn{2}{c|}{\textbf{Probability to fail}} \\ \cline{2-3}
      \hline
      \textbf{Size of Trusted Set (L)} & \textbf{p = 0.3} & \textbf{p = 0.05} \\
      \hline \hline
      %8 &  $0.0988087$ & $3.32222 \times 10^{-5}$ \\
      %\hline
      %16 & $0.0402769$ & $3.28714 \times 10^{-8}$  \\
      %\hline
      %32 & $0.00782066$ & $4.11893 \times 10^{-14}$  \\

      8 &  $0.098$ & $3.322 \times 10^{-5}$ \\
      \hline
      16 & $0.040$ & $3.287 \times 10^{-8}$  \\
      \hline
      32 & $0.007$ & $4.118 \times 10^{-14}$  \\
      \hline
    \end{tabular}
    \caption{Probability of failure when registering a new user account}
    \label{tab:p_account_registration}
  \end{table}
  
  \subsection{\textit{Probability of failure}}
    The user registration between a node $A$ and $I$ can fail if $I$ does
not respond (or responds false information) to $A$ during the registration request or during the registration
progress. In all cases, this corresponds to the probability that more than
$L/2$ nodes of $I$ are malicious. The probability of encountering exactly $k$
malicious nodes among $L +1$ is given by the binomial distribution

    \begin{equation} \label{eq:p_k_malicious_nodes}
      P_{K malicious} = \begin{pmatrix} L+1 \\ k\end{pmatrix} p^k (1-p)^{L+1-k}
    \end{equation}

    where $p$ is the probability that a single node is malicious. Hence the
probability of facing at most k malicious nodes is 

    \begin{equation}
      P_{\leq k} = \sum_{i=1}^{k} \begin{pmatrix} L+1 \\ k\end{pmatrix} p^i (1-p)^{L+1-i}
    \end{equation}

    Therefore, the probability of facing more than $k$ malicious nodes among
$n$ is given by

    \begin{equation} \label{eq:p_malicious_ge_k}
      P_{\ge k} = P_{\leq n} - P_{\leq k}
    \end{equation}

    and the probability that $I$ will not respond is

    %% for wolfram alpha
    %% P_{\leq L+1} =  sum( ((L+1) choose i)  * p^i * (1-p)^(L+1-i), from i=1 to L+1) where L=8, p=0.3
    %% P_{\ge L/2}  =  sum( ((L+1) choose i)  * p^i * (1-p)^(L+1-i), from i=1 to L+1) - sum( ((L+1) choose i)  * p^i * (1-p)^(L+1-i), from i=1 to L/2)  where L=8, p=0.3

    \begin{align} \label{eq:p_malicious_ge_L_2}
      P_{\ge L/2} &= P_{\leq L+1} - P_{\leq L/2} \\
      &= \sum_{i=1}^{L+1} \begin{pmatrix} L+1 \\ i\end{pmatrix} p^i (1-p)^{L+1-i}
      - \sum_{i=1}^{L/2} \begin{pmatrix} L+1 \\ i\end{pmatrix} p^i (1-p)^{L+1-i}
    \end{align}

    %% TODO: figure with the likeliness of dealing with an increains gnumber k of malicious nodes among a leafset comprising 32 nodes.

    Then the probability that $I$ will not respond to the user registration
request or responds with a fake registration response is

    \begin{align}
      P_{AI} &= P_{\ge L/2} 
    \end{align}


    Table~\eqref{tab:p_account_registration} shows the probability of failure
between $A$ and $I$, for a leafset size of $L = \{8,16,32\}$.

    
  \subsection{\textit{Message complexity}}
    First, $A$ must get the leafset of $I$. The associated cost is $n = 5
\times O(log_{2b}(N)) + Q + 4$ (as seen in section~\ref{sec:eval_leafset}).
    The number $n$ of message inherent to the transaction itself is given by

    \begin{align}
      n &= \underbrace{2(L+1)}_\text{Init} +
           \underbrace{L+1}_\text{Registration Data} +
           \underbrace{(L+2)(L+1)}_\text{Challenge} +
           \underbrace{(L+1)L}_\text{I registration confirmations} +
           \underbrace{L+1}_\text{ACKs}\\
      n &= (2L+6)(L+1)\\
      n &= 2L^2+ 8L + 6
    \end{align}

     The total cost is then
    $$
      n_{total} = 5 \times O(log_{2b}(N)) + Q + 4 + 2L^2 + 8L + 6
    $$    

    The total cost only depends on the size $L$ of the leafset, which is a
constant, and $O(log(N))$. In the best case, 
    $$
      n_{total} = O(log_{2b}(N)) + 2L^2 + 8L + 6
    $$
    Therefore, the cost user registration operation remains scalable when the
    size $N$ of the DHT increases, but there is a high constant cost that
    depends of the number of $L$. Still, this operation is not made frequently
    in the network and should not affect the performance of the
    service.

\section{Lazy User's Information Store Maintenance}
\label{sec:eval_lazy_maintenance}

      %8 &  $0.0988087$ & $3.32222 \times 10^{-5}$ \\
      %\hline
      %16 & $0.0402769$ & $3.28714 \times 10^{-8}$  \\
      %\hline
      %32 & $0.00782066$ & $4.11893 \times 10^{-14}$  \\

  \begin{table}
    \centering
    \footnotesize
    \begin{tabular}{|c|c|c|c|}
      \cline{3-4}
      \multicolumn{2}{c|}{} &  \multicolumn{2}{c|}{\textbf{Probability to fail}} \\ \cline{2-3}
      \cline{2-4}
      \multicolumn{1}{c|}{} & \textbf{L} & \textbf{p = 0.3} & \textbf{p = 0.05} \\
      \hline
      \multirow{3}{*}{\rotatebox[origin=c]{90}{\textbf{Case 1}}} & 8 & $0.106108$ & $3.9472 \times 10^{-5}$\\
      \cline{2-4}
      \multicolumn{1}{|c|}{} & 16 & $0.0403399$ & $3.29105 \times 10^{-8}$ \\
      \cline{2-4}
      \multicolumn{1}{|c|}{} & 32 & $0.00782066$ & $4.11893 \times 10^{-14}$ \\
      \hline
      \multirow{3}{*}{\rotatebox[origin=c]{90}{\textbf{Case 2}}} & 8 & $0.123141$ & $1.58218 \times 10^{-4}$ \\
      \cline{2-4}
      \multicolumn{1}{|c|}{} & 16 & $0.0404868$ & $3.36526 \times 10^{-8}$ \\
      \cline{2-4}
      \multicolumn{1}{|c|}{} & 32 & $0.00782067$ & $4.11893 \times 10^{-14}$ \\
      \hline
    \end{tabular}
    \caption{Probability of failure when performing the user's information store maintenance}
    \label{tab:p_lazy_maintenance}
  \end{table}
  
  \subsection{\textit{Probability of failure}}

    We must consider two cases for certificate log maintenance: when a new node
enters the leafset (case 1) and when a node leaves the leafset (case 2).

    \paragraph{\textbf{Case 1}} A new node enters the leafset. The maintenance
of the user information fails if it encounters $\frac{L}{2}$ consecutive
malicious nodes when building the node interval to retrieve the logs, or if
more than $\frac{L}{2}$ nodes are malicious (impossibility to have more than
$\frac{L}{2} +1$ identical answers).\\
    The probability $P_1$ of having $\frac{L}{2}$ consecutive malicious nodes
is given by equation~\ref{eq:p_k_malicious_nodes}: $P_1 = p^{L/2}$, and the
probability $P_2$ of not being able to retrieve at least $\frac{L}{2} +1$
identical answers is given by equation~\ref{eq:p_malicious_ge_L_2}: $P_2 =
P_{\ge L/2}$.\\
 Hence the total probability that this maintenance operation will
fail is

$$
  P_{total} = P_1 + (1-P_1) P_2
$$


    \paragraph{\textbf{Case 2}} A node is leaving the leafset. The user
information cannot be repaired if the node cannot repair its leafset (adding a
node on the left or right side), or if it is impossible to find at least
$\frac{L}{2} +1$ identical answers to retrieve the user information.

A node cannot repair its leafset if $\frac{L}{2} -1$ consecutive nodes are
malicious with a probability $P_1 = p^{L/2 - 1}$ and the user information
retrieval fails with a probability $P_2 = P_{\ge L/2}$.\\
Hence the total probability of failure is
$$
  P_{total} = P_1 + (1-P_1)P_2
$$

Table~\ref{tab:p_lazy_maintenance} computes the probability that a maintenance
operation will fail, given increasing leafset sizes $L = {8,16,32}$.


\section{User private key recovery}
\label{sec:eval_private_key_recovery}
% TODO Recalculate table
  \begin{table}
    \centering
    \footnotesize
    \begin{tabular}{|c|c|c|}
      \cline{2-3}
      \multicolumn{1}{c|}{}&  \multicolumn{2}{c|}{\textbf{Probability to fail}} \\ \cline{2-3}
      \hline
      \textbf{Size of Trusted Set (L)} & \textbf{p = 0.3} & \textbf{p = 0.05} \\
      \hline \hline
      8 &  $0.098$ & $3.322 \times 10^{-5}$ \\
      \hline
      16 & $0.040$ & $3.287 \times 10^{-8}$  \\
      \hline
      32 & $0.007$ & $4.118 \times 10^{-14}$  \\
      \hline
    \end{tabular}
    \caption{Probability of failure when recovering user private key}
    \label{tab:p_private_key_recovery}
  \end{table}

  \subsection{\textit{Probability of failure}}

    The user private key recovery between a node $A$ and $I$ can fail (a) if $I$ does
not respond (or responds fake information) to $A$ during the key recovery
request. While a response with fake information (like a salt forged by malicious nodes) can be more dangerous for the
user, the probability remains the same, corresponding to the
probability that more than $L/2$ nodes of $I$ are malicious.\\
    The probability that $I$ will not respond to the user private key recovery
request (or that $I$ responds fake user information) is the probability of
facing more than $\frac{L}{2} +1$ malicious nodes in $I$. This probability is given by the
equation~\ref{eq:p_malicious_ge_L_2} and hence the probability that the user
private key recovery fails is:

\begin{equation} \label{eq:L_2_malicious_A_I}
 P_{AI} = P_{\ge L/2}
\end{equation}

    Table~\eqref{tab:p_private_key_recovery} shows the probability of failure
between $A$ and $I$, for a leafset size of $L = \{8,16,32\}$.

    
  \subsection{\textit{Message complexity}}
    First, $A$ must get the leafset of $I$. The associated cost is $n = 5
\times O(log_{2b}(N)) + Q + 4$ (as seen in section~\ref{sec:eval_leafset}).
    The number $n$ of message inherent to the transaction itself is given by

    \begin{align}
      n &= \underbrace{2(L+1)}_\text{Init} + \underbrace{L+1}_\text{Data} +  \underbrace{L+1}_\text{ACKs}\\
      n &= 4(L+1)
    \end{align}
     The total cost is then

    $$
      n_{total} = 5 \times O(log_{2b}(N)) + Q + 4 + 4(L+1)
    $$    
    The total cost only depends on the size $L$ of the leafset, which is a
constant, and $O(log(N))$. In the best case, 

    $$
      n_{total} = O(log_{2b}(N)) + 4(L+1)
    $$
    Therefore, the cost of recovering a user private key remains
scalable when the size $N$ of the DHT increases.


\section{User Sign-in}
\label{sec:eval_sign_in}
  \begin{table}
    \centering
    \footnotesize
    \begin{tabular}{|c|c|c|}
      \cline{2-3}
      \multicolumn{1}{c|}{}&  \multicolumn{2}{c|}{\textbf{Probability to fail}} \\ \cline{2-3}
      \hline
      \textbf{Size of Trusted Set (L)} & \textbf{p = 0.3} & \textbf{p = 0.05} \\
      \hline \hline
      8 &  $0.188$ & $6.64 \times 10^{-5}$ \\
      \hline
      16 & $0.079$ & $6.57 \times 10^{-8}$  \\
      \hline
      32 & $0.016$ & $8.24 \times 10^{-14}$  \\
      \hline
    \end{tabular}
    \caption{Probability of failure in user sign-in}
    \label{tab:p_sign_in}
  \end{table}
  
  \subsection{\textit{Probability of failure}}
    A sign-in request between a node $A$ and the service $S$, can fail (a) if the identification service $I$ does
not respond (or responds fake information) to $S$ during the user information recovery
request, or (b) if $S$ does not respond (or responds fake information) to $A$
during the challenge or the session keys generation. While a response with fake information can be more dangerous for the
user, the probability remains the same, corresponding to the
probability that more than $L/2$ nodes of $I$ (or $S$) are malicious.
    As seen before, the probability of facing more than $k$ malicious nodes among
$n$ is given by the equation~\ref{eq:p_k_malicious_nodes}.

    Then the probability that $I$ will not respond to the service user
information recovery request, or that $S$ not respond to the node $A$ is

    \begin{align}
      P_{AI} &= (1- P_{\ge L/2}) P_{\ge L/2} +  P_{\ge L/2} \\
      P_{AI} &= 2P_{\ge L/2} - P^2_{\ge L/2}
    \end{align}


    Table~\eqref{tab:p_sign_in} shows the probability of failure
between $A$, the service $S$ and $I$, for a leafset size of $L = \{8,16,32\}$.

  \subsection{\textit{Message complexity}}
    First, $A$ must get the leafset of $S$. The associated cost is $n = 5
\times O(log_{2b}(N)) + Q + 4$ (as seen in section~\ref{sec:eval_leafset}).
Then, every node in $S$ must get the leafset of $I$, with each leafset request
having the same associated cost as before $n = (L+1)(5 \times O(log_{2b}(N)) + Q + 4)$~\ref{sec:eval_leafset}.
    The number $n$ of message inherent to the transaction itself is given by

    \begin{align}
      % TODO: + ommited for space, fix this with other way to express this
      n &= \underbrace{2(L+1)}_\text{Init A with S} +
           \underbrace{2(L+1)^2}_\text{Init S with I} +
           \underbrace{(L+1)^2}_\text{Data from I to S} +
           \underbrace{(L+1)L}_\text{Init Challenge} +
           \underbrace{2(L+1)}_\text{S challenges A} +
           \underbrace{L+1}_\text{ACKs}\\
      n &= 5(L+1) + 3(L+1)(L+1)+ (L+1)L\\
      n &= 4L^2 +12L + 8 
    \end{align}
     The total cost is then

    $$
      n_{total} = (L +2)(5 \times O(log_{2b}(N)) + Q + 4) + 4L^2 +12L + 8 
    $$    
    The total cost only depends on the size $L$ of the leafset, which is a
constant, and $O(log(N))$. In the best case, 

    $$
      n_{total} = (L +2)O(log_{2b}(N)) + 4L^2 +12L + 8 
    $$
    Therefore, the cost of a user sign in operation remains
scalable when the size $N$ of the DHT increases.

\section{Logout}
  \label{sec:eval_logout}
  \begin{table}
    \centering
    \footnotesize
    \begin{tabular}{|c|c|c|}
      \cline{2-3}
      \multicolumn{1}{c|}{}&  \multicolumn{2}{c|}{\textbf{Probability to fail}} \\ \cline{2-3}
      \hline
      \textbf{Size of Trusted Set (L)} & \textbf{p = 0.3} & \textbf{p = 0.05} \\
      \hline \hline
      8 &  $0.098$ & $3.322 \times 10^{-5}$ \\
      \hline
      16 & $0.040$ & $3.287 \times 10^{-8}$  \\
      \hline
      32 & $0.007$ & $4.118 \times 10^{-14}$  \\
      \hline
    \end{tabular}
    \caption{Probability of failure in user logout}
    \label{tab:p_logout}
  \end{table}
  
  \subsection{\textit{Probability of failure}}
    The user logout between a node $A$ and $S$ can fail if $S$ does
not respond (or responds false information) to $A$ during the logout request.
  In all cases, this corresponds to the probability that more than
$L/2$ nodes of $I$ are malicious.
    As seen before, the probability of facing more than $k$ malicious nodes among
$n$ is given by the equation~\ref{eq:p_k_malicious_nodes}.//

    Then the probability that $S$ will not respond to the user logout 
request or responds with a fake logout response is

    \begin{align}
      P_{AI} &= P_{\ge L/2} 
    \end{align}


    Table~\eqref{tab:p_logout} shows the probability of failure
between $A$ and $S$, for a leafset size of $L = \{8,16,32\}$.

    
  \subsection{\textit{Message complexity}}
    First, $A$ must get the leafset of $S$. The associated cost is $n = 5
\times O(log_{2b}(N)) + Q + 4$ (as seen in section~\ref{sec:eval_leafset}).
    The number $n$ of message inherent to the transaction itself is given by

    \begin{align}
      n &= \underbrace{2(L+1)}_\text{Init} +  \underbrace{L+1}_\text{Logout Request} + \underbrace{L+1}_\text{ACKs}\\
      n &= 4(L+1)
    \end{align}
     The total cost is then

    $$
      n_{total} = 5 \times O(log_{2b}(N)) + Q + 4 + 4(L+1)
    $$    
    The total cost only depends on the size $L$ of the leafset, which is a
constant, and $O(log(N))$. In the best case, 

    $$
      n_{total} = O(log_{2b}(N)) + 4(L+1)
    $$
    Therefore, the cost of a user logout operation remains scalable when the size $N$ of the DHT increases.

\section{Password Change}
  \label{sec:eval_password_change}
  \begin{table}
    \centering
    \footnotesize
    \begin{tabular}{|c|c|c|}
      \cline{2-3}
      \multicolumn{1}{c|}{}&  \multicolumn{2}{c|}{\textbf{Probability to fail}} \\ \cline{2-3}
      \hline
      \textbf{Size of Trusted Set (L)} & \textbf{p = 0.3} & \textbf{p = 0.05} \\
      \hline \hline
      8 &  $0.098$ & $3.322 \times 10^{-5}$ \\
      \hline
      16 & $0.040$ & $3.287 \times 10^{-8}$  \\
      \hline
      32 & $0.007$ & $4.118 \times 10^{-14}$  \\
      \hline
    \end{tabular}
    \caption{Probability of failure when changing the user password}
    \label{tab:p_password_change}
  \end{table}
  
  \subsection{\textit{Probability of failure}}
    A password change request between a node $A$, the identification service
$I$ can fail (a) if $L/2 + 1$ nodes in $I$ does not responds (or responds fake
information) correctly the password change confirmation message to the nodes 
$I_i \in I$ during the password change request, or (b) if $S$ does not respond (or responds fake information) to $A$
during the password change request. While a response with fake information can be more dangerous for the
user, the event of that happening is the same as when there is no response:
when more than $L/2$ nodes of $I$ are malicious.

    As seen before, the probability of facing more than $k$ malicious nodes among
$n$ is given by the equation~\ref{eq:p_k_malicious_nodes}.//

    Then the probability that $I$ will not respond to the service user
information recovery request is

    \begin{align}
      P_{AI} &= P_{\ge L/2} \\
    \end{align}


    Table~\eqref{tab:p_password_change} shows the probability of failure
between $A$, the service $S$ and $I$, for a leafset size of $L = \{8,16,32\}$.

  \subsection{\textit{Message complexity}}
    First, $A$ must get the leafset of $I$. The associated cost is $n = 5
\times O(log_{2b}(N)) + Q + 4$ (as seen in section~\ref{sec:eval_leafset}).
%%% not needed, this has to be done in the leafset maintenance
%Then, every node in $I$ must get the leafset of $I$, with each leafset request
%having the same associated cost as before $n = (L+1)(5 \times O(log_{2b}(N)) + Q + 4)$~\ref{sec:eval_leafset}.
    The number $n$ of message inherent to the transaction itself is given by

    \begin{align}
      n &= \underbrace{2(L+1)}_\text{Init A with I} +
           \underbrace{(L+1)L}_\text{Data from I to I} +
           \underbrace{L+1}_\text{ACKs}\\
      n &= 3(L+1) + L(L+1)\\
      n &= L^2 + 4L + 3
    \end{align}
     The total cost is then

    $$
      n_{total} = 5 \times O(log_{2b}(N)) + Q + 4 +  L^2 + 4L + 3
    $$    
    The total cost only depends on the size $L$ of the leafset, which is a
constant, and $O(log(N))$. In the best case, 

    $$
      n_{total} = O(log_{2b}(N)) + L^2 + 4L + 3
    $$
    Therefore, the cost of a password change operation remains
scalable when the size $N$ of the DHT increases.

%We suppose in the following that the underlying
%reputation system makes an error $\varepsilon$ when classifying a
%node $X$ with a reputation $R(x) \geq \rho$, where $\rho \in [ 0 \cdots 1 ]$,
%and $ \varepsilon = f ( \rho )$. In other words, classifying a node $X$ as
%honest because its reputation is greater than $\rho$ has a
%probability of error $\varepsilon$.
%Let $n$ be the size of the Trusted Ring. The probability
%to have $k$ misclassified nodes in the Trusted Ring, that
%is $k$ malicious nodes is:
%
%$$
%P_{k_{malicious}} = \left(\!
%                          \begin{array}{c}
%                            n\\
%                            k
%                          \end{array}
%                    \!\right)              
%                    \varepsilon^{n-k} ( 1 - \varepsilon )^k
%$$
%
%Then, the probability to have at most $k$ malicious
%nodes in a Trusted Ring of size $n$ is:
%
%$$
%P_{\leq k} = \sum^{k}_{i=1} \left(\!
%                                \begin{array}{c}
%                                    n\\
%                                    k
%                                  \end{array}
%                            \!\right)              
%                    \varepsilon^{n-i} ( 1 - \varepsilon )^i
%$$
%
%Therefore, the probability to have $k$ or more malicious
%nodes in a Trusted Ring of size $n$ is:
%
%$$
%P_{\leq k} = \sum^{n}_{i=k} \left(\!
%                                \begin{array}{c}
%                                    n\\
%                                    i
%                                  \end{array}
%                            \!\right)              
%                    \varepsilon^{n-i} ( 1 - \varepsilon )^i
%$$
%
%The user identification fails when:
%\begin{enumerate}
%  \item The user cannot retrieve his own PKI from the \textit{trustset}.
%  \item or when the public key of the user fails to be retrieved.
%\end{enumerate}
%
%These failures can happen when the \textit{trustsets} storing the PKI or the public
%key have more malicious nodes than normal nodes.
%
%The probability that a \textit{trustset} has half or more malicious nodes, assuming a maximum
%classification error for the underlying reputation system
%of $5\%$, is .% FILL HERE
%%Hence the probability for having a fully erroneous trustset is theoretically possible, but
%%practically infeasible.
%
%Considering a maximum error rate of $5\%$ is a typical
%value for a reputation system. In some cases it may be
%over-estimated (for more details, please refer the results
%obtained for the WTR reputation system\cite{wrt_reputation_system}). This
%error hardly depends on the total number of malicious
%nodes in the network, and decreases when the ratio
%of malicious node decreases. The less malicious nodes
%there are in the system, the easier it is to discriminate
%against them.
