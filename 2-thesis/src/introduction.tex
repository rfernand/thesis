% old stuff
The subject of securely establishing stable identities in P2P
systems has been previously studied, for instance by Aberer,
Datta and Hauswirth [5]. The need for identities mainly arose
from technical concerns, such as handling dynamic IP address
assignment, or avoiding Sybil attacks [6]. Authentication of a
node is done via a signature key, automatically generated and
stored on the node.

Traditionally, P2P networks identified the different nodes that compose the
system, but not the user behind each one of them as a different being.
As P2P systems grow in funciontalities, the need to identify users inside the
network arose.


P2P networks are conformed by nodes with unique identifiers that map to their
own IP address. That way is easy to differenciate them between each others, but
it is not enough to handle user identities. To let an user log in to the
network using different nodes, a user identity \textit{proof} is needed. That
is because all nodes the P2P networks shares the same functionalities, so
anybode can identify as an user if it can prove it for the other peers. For
this to work, the proof of identity need to be:
\begin{description}
  \item[Unique] The proof and identity has to be unique for each user in the network.

    %Unique identifiers can be obtained 
    % HASHES(username) maps to a trusted ring of nodes that determines if the
    % username is taken or not
  \item[Verificable for everyone]  Each node has to be able to verify the proof
    handled by other node, making the relation with the corresponding user if it
    exists.
    % The information to verify the proof has to be in the p2p network. 
    % Trusted ring of nodes that store the information and give it to the nodes
    % that wants to verify the user.
  \item[Impossible to guess or break] They way to obtain the proof should not
    be guessed by other node or obtained without specific knowledge of the
    corresponding user.
    % this part is hard. It cannot be impossible, but it can be very very very
    % hard to break without having the right private user knowledge.  Cryptographic proof.
\end{description}




As P2P systems began providing more complex functional-
ity [2], [3], [4], [7], the need to authenticate users, rather than
nodes, arose. It seems that often, authentication via a signature
key has been carried over to this problem. While a solution
of automatic identification of a node is preferable as long as
users use a single device, equating a node with a user fails as
users increasingly access services from multiple devices.
Illustrative is the case of backup systems, where an impor-
tant use case is to restore data on a different system from where
it was backed up. Here, two different approaches to authen-
tication have been taken. All approaches build on encrypting
backed up content, and the approaches vary in whether the
keys are randomly derived [7], or derived from a password [8].
In the former case, a user must manually back the keys up,
as these keys are required to restore the backup. The systems
deriving a key from a password are related to our proposed
protocol, and use some related techniques. However, to the
best of our knowledge, they do not consider the additional
protocols required surrounding password authentication, such
as remembered logins, and recovering lost passwords.
Some P2P storage systems also use techniques which are
related to ours. For example, the DHT-based systems GNUnet
and Freenet use keyword strings to derive a public-private key
pair whose private key is used to sign data and the hash of
the public key to identify the data in the storage. Both of
these systems use a keyword string as a seed to a pseudo-
random number generator that produces the key pair [9], [10].
Knowing only the memorable keyword string the user can
store and retrieve information.
Related to forgotten passwords, recovery of information in a
P2P scenario has been studied by Vu et al. [11] who proposed
a combination of threshold-based secret sharing with delegate
selection and encrypting shares with passwords.
Frykholm and Juels [12] proposed a password-recovery
mechanism based on security questions very similar to our
protocol for the same task. They offer better, information-
theoretic security properties, something not applicable to our
scenario. We treat the subject of password change, which is not applicable to
their scenario, although their proposal could be extended to support password
change using our techniques.



