P2P networks are form by nodes with unique identifiers that map to their
own IP address making it easy to differenciate each node from another, but
impossible to verify a particular user.
 To let a user sign in to the
network using different nodes, a user identity \textit{proof} is needed.

% TODO: REWRITE THIS
%An user identity proof is something known only by the user what diferenciates one user from another in the network
 %That %is because all nodes the P2P networks shares the same functionalities, so
%anybody can identify as an user if it can prove it for the other peers.

For this to work, the proof of identity needs to be:
\begin{enumerate}[a)]
  \item \textbf{Unique:} The proof and identity has to be unique for each user in the network.

    %Unique identifiers can be obtained 
    % HASHES(username) maps to a trusted ring of nodes that determines if the
    % username is taken or not
  \item \textbf{Verifiable for everyone:}  Each node has to be able to verify proofs
    handled by other nodes, identifying the relationship with the corresponding user if it
    exists.
    % The information to verify the proof has to be in the p2p network. 
    % Trusted ring of nodes that store the information and give it to the nodes
    % that wants to verify the user.
  \item \textbf{Impossible to guess or break:} They way to obtain the proof should not
    be able to be guessed by other nodes or obtained without specific knowledge
    provided by the corresponding user.
    % this part is hard. It cannot be impossible, but it can be very very very
    % hard to break without having the right private user knowledge.  Cryptographic proof.
\end{enumerate}

Traditionally, P2P networks identified the different nodes that composed the
system, but not the user behind each one of them as a different being.
As P2P systems grew in functionalities, the need to identify users inside the
network arose.\\

While a solution of automatic identification of a node is preferable, equating a node with a user fails as users
increasingly access services from multiple devices.
%example
Illustrative is the case of backup systems, where an important-
  use case is to restore data on a different system from where
it was backed up. \\

%%%% This was moved from formalization to the introduction. Seems better placed here.
Existing systems of user identification in P2P networks only consider the
use of pre-shared keys to identify the user in the network. While that can
be easily implemented, it does not provide the flexibility that a
username-password based identification provides when 
users sign in to the system from different devices.
However, when users manually transfer their keys from one device to another
many security issues arise, such as the keys being handled carelessly, or the
devices, like a cellphone, being lost.\\

The use of a username and a password means that the user keys need to: a) be
secured inside the identification system or b) be derived from information
stored in the identification system. The first option
needs trusted storage with additional use of encryption to secure the
private keys, and would have some issues with offline brute force
attacks~\cite{kreitz2012passwords}. The latter option let us build the identification system with the same desired properties without adding
additional complexity, but does support a password recovery mechanism for the
user identification system.\\

Still, to handle the user keys without compromising the user's identity, additional
security layers need to be placed inside the P2P network.
To secure the stored keys, the proposed system uses encryption, indirection and
rings of trust inside the network. The system goal is to offer a secure means to
identify a user using only his username/password knowledge, taking into
consideration the presence of Byzantine nodes. The following work will only
focus on the registration, sign-in, logout and password change protocols,
leaving aside the issue of password recovery mechanisms in the system. If
a user loses his password and private keys used to sign in to the system, the
user's registered identity is lost and a new identity will need to be registered.\\

As shown in \cite{the_sybil_attack}, it is impossible to get a 100\% reliable
system over a DHT where malicious nodes are present. 
The goal of our approach is to drastically reduce the probability of a
malicious node impersonating another user who is using the identification
system.\\
%%%% This was moved from formalization to the introduction. Seems better placed here.


This present work will have the following structure:\\

In Chapter~\ref{sec:soa_p2p} we explain the
main characteristics of a P2P system and choose the main base for the
construction of our identification system. In
Section~\ref{sec:soa_p2p_user_identification}
we define a user identification system, and discuss proposed solutions for
improving their security and those solution's main problems.
 In Section~\ref{sec:soa_p2p_trust} we see the solutions that can
develop a trusted ring inside a P2P network, allowing the maintenance of
quasi-trusted services in the system.  
In Chapter~\ref{sec:formalization} we define the main assumptions and building
blocks for the
identification system, formalizing the notations used for our proposed system's protocols.
In Chapter~\ref{sec:system}  we describe the different parts of the system and how the
identification protocols will work.
In Chapter~\ref{sec:evaluation} we present a probability assessment of
the algorithms used in our identification system, and show that it is highly
improbable to fail.
In Chapter~\ref{sec:validation} we discuss some security issues and the portability of our
solution.
In Chapter~\ref{sec:conclusion} we conclude with the main achievements of our work
and depict the next steps for the future.
