P2P networks are conformed by nodes with unique identifiers that map to their
own IP address. That way is easy to differentiate them between each others, but
it is not enough to handle user identities. To let an user log in to the
network using different nodes, a user identity \textit{proof} is needed. That
is because all nodes the P2P networks shares the same functionalities, so
anybody can identify as an user if it can prove it for the other peers. For
this to work, the proof of identity need to be:
\begin{enumerate}[a)]
  \item \textbf{Unique:} The proof and identity has to be unique for each user in the network.

    %Unique identifiers can be obtained 
    % HASHES(username) maps to a trusted ring of nodes that determines if the
    % username is taken or not
  \item \textbf{Verifiable for everyone:}  Each node has to be able to verify the proof
    handled by other node, making the relation with the corresponding user if it
    exists.
    % The information to verify the proof has to be in the p2p network. 
    % Trusted ring of nodes that store the information and give it to the nodes
    % that wants to verify the user.
  \item \textbf{Impossible to guess or break:} They way to obtain the proof should not
    be guessed by other node or obtained without specific knowledge of the
    corresponding user.
    % this part is hard. It cannot be impossible, but it can be very very very
    % hard to break without having the right private user knowledge.  Cryptographic proof.
\end{enumerate}

Traditionally, P2P networks identified the different nodes that compose the
system, but not the user behind each one of them as a different being.
As P2P systems grow in functionalities, the need to identify users inside the
network arose.\\

While a solution of automatic identification of a node is preferable as long as
users use a single device, equating a node with a user fails as users
increasingly access services from multiple devices.
%example
Illustrative is the case of backup systems, where an important-
  use case is to restore data on a different system from where
it was backed up. \\

%%%% This was moved from formalization to the introduction. Seems better placed here.
Existing systems for the user identification in P2P networks only consider the
use of pre-shared keys to identify the user in the network. While that can
be easily implemented, does not provide to the users the flexibility that a
username-password based identification provides when using different devices to
log in in the system. As the user needs to transfer manually his keys from one
device to another, there are many security issues when they are handled without
care or the devices (like a cellphone) are lost. \\

The use of a username and a password means that the user keys needs to: a) be
secured inside the identification system or b) be derived from information
stored in the identification system. Between  the two options, the first one
needs a trusted storage with additional use of encryption to secure the
private keys, and have some issues with offline brute force
attacks~\cite{kreitz2012passwords}. In the other hand deriving the user private
keys let us build the identification system with the same desired properties without adding an
additional complexity, but does support recovery password mechanism for the
user identification system.\\
Still, to handle the user keys without compromising the users identity, additional
security layers needs to be placed inside the P2P network.
To secure the stored keys, the proposed system uses encryption, indirection and
rings of trust inside the network. The system goal is to offer a secure mean to
identify an user using only his username/password knowledge taking in
consideration the presence of byzantine nodes. The following work will only
focus in the registration, sign-in, logout and password change protocols,
leaving aside the inclusion of password recovery mechanisms in the system. If
an user lose his password and private key used to sign in the system, the user
registered identity is lost and a new identity will need to be registered.\\

As shown in \cite{the_sybil_attack}, it is impossible to get a 100\% reliable
system over a DHT where malicious nodes are present. The goal of our approach
is to reduce drastically the probability of a malicious node to impersonate
another user using the identification system.\\
%%%% This was moved from formalization to the introduction. Seems better placed here.


This present work will have the following structure:\\

In Chapter~\ref{sec:soa_p2p} we show the definition of P2P networks, explaining the
main characteristics of a P2P system, choosing the main base for the
construction of the identification system. In
Section~\ref{sec:soa_p2p_user_identification} we will see what we define as a
user identification system, identifying the existing solutions and their main
problems. In Section~\ref{sec:soa_p2p_trust} we see the solutions that can
develop a trusted ring inside a P2P network, allowing the maintenance of
quasi-trusted services in the system.  
In Chapter~\ref{sec:formalization} we define the main assumptions and building
blocks for the
identification system, formalizing the notations used for the following system's protocols.
In Chapter~\ref{sec:system}  we will describe the different parts of the system and how the
identification protocols will work.
In Chapter~\ref{sec:evaluation} we present a probability assessment of
the algorithms used in our identification system, and show that it is highly
improbable to fail.
In Chapter~\ref{sec:validation} we will discuss some security issues and the portability of our
solution.
In Chapter~\ref{sec:conclusion} we conclude the main achievements of our work
and depict the next steps for the future.
