In this chapter we will discuss some security issues and the portability of our
solution.% and our decision about not having a password recovery mechanism

\section{Security issues}
%Regarding P2P networks and identification systems, there are many issues that
%hand that need to be accounted for when building the system. 
% objectives achieved?
% what L? (L=32)
\subsection{Trustworthiness}
In comparison with centralized approaches, we know that the reliability of a
centralized identification service depends mostly on the
supporting host: its availability, its security and its capacity  to handle
significant amounts of concurrent requests. 
In practice this entails that a centralized identification service is not
foolproof: it is as trustworthy as the group of servers it runs on.
Well-established centralized identification services have been breached. %%%%%%%%% TODO: ADD REFERENCES
In a decentralized approach, the main issue is that the supporting hosts are
not accountable for their decisions and therefore cannot be trusted.\\

Because of this, as other P2P systems, a 100\% secure implementation cannot be achieved
(section~\ref{sec:pseudo-secure}).
Having account of that, because even a trusted node is not accountable on its
own, we rely in sets of nodes to cooperate for the verification of the
integrity of the operation.

With our solution, with a leafset size of $L = 32$ nodes and a
$5\%$ proportion of malicious nodes distributed uniformly in the P2P system, the resulting probability of a false
positive, where a decision made by malicious nodes is adopted, is of the order
of $10^{-14}$. 
This value is far lower than any we have encountered in the
literature about decentralized identification systems.Also the system maintains a cost scalable when the size of nodes $N$ of the DHT
increases, and remains very affordable in terms of communications overhead. 
We think that probability is acceptable for a user identification system to be
used in real life applications.


\subsection{Sybil attacks}
\label{sec:sybil_attacks}

The Sybil attack~\cite{the_sybil_attack} is a major concern if obtaining
network identities is cheap (NodeIDs). Such an attack allows to tamper the ratio of
malicious nodes in the system as a single node acts under multiple identities:
the ratio can rise above the $30\%$ upper threshold. \\

A common practise of the Sybil attack has a single malicious user create
several identities with numerically close nodeIDs to be able to control a
substantial part of the DHT. We assume that the generation of the nodeIDs can
be verified by any node in the network, for instance the bootstrapping node or
the node with the closest nodeID. Our proposition assumes that nodeIDs are
computed from a public key, generated with a well-known public/private key
algorithm. We also suppose that the number of bits of these keys is sufficient
to avoid a brute force attack to reverse the SHA-1 function. During the join
procedure to the DHT, a node $A$ must provide several pieces of information to
the verifier node $X$: its public key $key_{pub}$, and the associated nodeID computed using
$nodeID = SHA1(key_{pub}$. $X$ can easily verify the correctness of the nodeID
with the public key. IF they do not match, then $X$ prevents $A$ from joining
the ring.\\

Due to the mathematical properties of the SHA-1 hash function, it is
practically impossible for a node to find two keys $key_{pub1}$ and
$key_{pub2}$ such that $SHA1(key_{pub1}) = SHA1(key_{pub2}$. Therefore, it is
also practically impossible for a node to find several public keys that give
numerically close nodeIDs in the ring. Thus it is practically unfeasible for a
node to control a substantial sector of the ring.\\

However, a single node can still generate several nodeIDs distributed over the
ring. This kind of Sybil attack does not allow to control a sector of the ring,
but will increase failures in the DHT (Routing failures, etc.).  Reputation
systems can efficiently mitigate the Sybil attack, but will collapse if the
ratio of malicious nodes is too high ($> 30\%$). To our knowledge, no existing reputation system can control a large
number of Sybil attacks in a DHT, and this is still a difficult topic within
the P2P research community.\\

As seen before in section~\ref{sec:challenges_puzzles}, other solutions to
mitigate the presence of sybils is to make obtaining nodeIDs
exponentially difficult when a node needs a new
one. Existing solutions usually
rely on a set of servers to provide the nodeIDs, and are therefore not very
scalable.\\

The impact of a Sybil attack over the Trusted ring of the reputation system
would mainly be at the level of its \textit{trustsets}. The main idea is to be
able to control various nodeIDs in a \textit{trustset}. This is not an easy
task; as shown above is practically unfeasible to obtain numerically close
nodeIDs that belong to the same \textit{trustset}. Moreover, controlling a very
large number of trustsets is practically impossible, as it will require to
control a very large number of nodeIDs. This is unfeasible due to the number of
transactions required from an attacker in order to establish a good reputation
for all of its nodeIDs, and then insert them into the Trusted
Ring~\cite{rosas2011corps}.\\

Analog to the problem of Sybil attacks to control multiple nodeIDs, an attacker
would like to try to control a large amount of user accounts in the
identification system. Having registered a large amount of usernames used, a new
user could have difficulties to find an unused username when registering in the
network. Also, having a large amount of identities through the services in the
network can be used to exploit some of their functionalities.
In our solution, we use computational challenges to mitigate the problem of a
node registering multiple amount of users among the system. Other solutions to
this problem is the use of CAPTCHA~\cite{von2003captcha} as an additional
challenge for the user registration. While this type of challenge is not
without faults, it adds an additional layer of security against Sybil attacks.


\subsection{Collusion of nodes}

A collusion of nodes consists in an agreement made among various nodes to make
a combined attack, like artificially increase or decrease the reputation of a
set of nodes by colluding nodes of the reputation system. Some existing
reputations systems like WTR can efficiently mitigate the effect of this kind
of attack. Moreover, colluded nodes should be able to sustain the attack
permanently in order to maintain a good reputation and remain the Trusted Ring.
This will have a significant cost for the attackers, and being able to manage a
great number of attackers is very unlikely. 

\subsection{Offline attacks}
As our solution relies in a public/private key scheme to identify the user,
using derived keys from a public salt stored in the network, a malicious node
could try to guess the password of an user. If the function use to derive the
private key has a low computation cost, a brute force attack that tries
different passwords and test the derived private key with the user public key
can succeed if the attacker has the computational resources to do so. We
assumed that the key derivation function (e.g., bcrypt) is slow enough to make
unfeasible for a computer to guess the password of an user in a certain
amount of time (like more than 20 years).

\subsection{Passwords}
The traditional view is that passwords should be replaced by some better
mechanism, but despite significant research efforts into dislodging passwords,
they are still by far the most common authentication mechanism today.
Reasons for their prevalence include simplicity, price, and very strong user
familiarity.
It is known that the average user tends to use the
same password in more than one service, presenting serious security problems
for the user identity integrity. A definitive solution to this problem have not
been found yet, but a strict password creation policy can seems to mitigate this
problem (but at the same time this can increase the cases of forgotten passwords)

%\section{Password recovery}
%While the system does not allow password recovery operations...



%there can a time
%when the group nodes cannot decide over something in common, with the
%When commiting a transaction that needs a group of $n$ nodes
%
%
%This causes that after a transaction that
%involves a group of nodes, there is some amount of time where the nodes are not
%syncronized.

\subsection{Node data inconsistency}
Synchronization between nodes does not happen instantly. Also, thanks to
the different routes used in the network, simultaneous messages can reach a
destination at different times. So, differently to centralized systems, P2P
networks have a certain time where the data in some nodes are inconsistent.
This is relevant for functions that need a
group of nodes to be synchronized at a certain time, because this ``temporal
inconsistency''  can cause fake positives or fake negatives in the protocols of
the network. For example, after a successful password change request, there can
be a time when some nodes will have the old user information. If another
request involving the user credentials is issued while these nodes have not
synchronized their data, depending in what nodes are used in the request, the
validation that needs $\frac{L+1}{2}$ identical answers can fail or succeed.
While the timeframe where the nodes have not undergone synchronization is small,
it still can lead to problems if is it not considered in the protocols
construction. In consequence, an added time span is considered for the
successful completion the user registration and password change operations.

%To mitigate this problem, we propose the use of an active user information store
%maitenance.  and add consider a timespan of 1 to 2 seconds  


\section{Portability}

Although our implementation relies on Pastry as its DHT, our approach can rely
on other building block implementations. For example it is very simple to
switch the underlying DHT from Pastry to Chord, as it maintains structures
that are equivalent to the leafset and the routing table in Pastry. An
underlying reputation system with the same properties explained
in Section~\ref{sec:building_blocks}  is also required, but can be easily
supported by any type of DHT, as long as it provides some type of key lookup
service and.
