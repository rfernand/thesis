\input{src/intros/5-problemasysoluciones}

\section{Definición de problemas actuales}
\label{sec:problemas}

Como vimos anteriormente, para el desarrollo de una red social distribuida, se necesitan
soluciones para sus funcionalidades básicas~\ref{sec:conclusiones_soa}. Ahora,
nos enfocaremos en presentar una arquitectura de prueba del sistema, por lo que
se abstendrá la presentación de implementaciones específicas de
éstos problemas, para pasar a tratar instancias representativas de una o más
funcionalidades.

Para ello se analizará el cómo se estructurará, organizará y
almacenará la información dentro del sistema, entre otros puntos transversales a la
implementación de éstas funcionalidades, lo cual será organizado bajo las
siguientes aristas principales:


\begin{itemize}
    \item Almacenamiento de la información.
    \item Búsquedas de la información.
    %\begin{itemize}
        %\item {Perfil de usuario}
        %\item {Publicación de contenido}
        %\item Búsqueda e identificación dentro de la red.
        %\begin{itemize}
            %\item {Lista de contactos}
            %\item {Organización de los contactos del usuario}
            %\item {Búsqueda de usuarios, grupos y contenido publicado}
        %\end{itemize}
        %\item Conectividad y disponibilidad de la información.
    %\end{itemize}
    \item Seguridad.
    %\begin{itemize}
        %\item {Configuraciones de privacidad}
    %\end{itemize}
    \item Integración con otros servicios y aplicaciones.

\end{itemize}

\section{Problemas y soluciones del almacenamiento de la información}
\label{sec:almacenamiento}
\input{src/almacenamiento}

\section{Problemas y soluciones de la búsqueda de la información}
\label{sec:busqueda}
\input{src/busqueda}

\section{Problemas y soluciones de seguridad}
\label{sec:seguridad}
\input{src/seguridad}

\section{Integración con otras aplicaciones}
\label{sec:conectividad}
\input{src/conectividad}


\section{Definición de problemas actuales}
\label{sec:problemas}

Como vimos anteriormente, para el desarrollo de una red social distribuida, se necesitan
soluciones para sus funcionalidades básicas~\ref{sec:conclusiones_soa}. Ahora,
nos enfocaremos en presentar una arquitectura de prueba del sistema, por lo que
se abstendrá la presentación de implementaciones específicas de
éstos problemas, para pasar a tratar instancias representativas de una o más
funcionalidades.

Para ello se analizará el cómo se estructurará, organizará y
almacenará la información dentro del sistema, entre otros puntos transversales a la
implementación de éstas funcionalidades, lo cual será organizado bajo las
siguientes aristas principales:


\begin{itemize}
    \item Almacenamiento de la información.
    \item Búsquedas de la información.
    %\begin{itemize}
        %\item {Perfil de usuario}
        %\item {Publicación de contenido}
        %\item Búsqueda e identificación dentro de la red.
        %\begin{itemize}
            %\item {Lista de contactos}
            %\item {Organización de los contactos del usuario}
            %\item {Búsqueda de usuarios, grupos y contenido publicado}
        %\end{itemize}
        %\item Conectividad y disponibilidad de la información.
    %\end{itemize}
    \item Seguridad.
    %\begin{itemize}
        %\item {Configuraciones de privacidad}
    %\end{itemize}
    \item Integración con otros servicios y aplicaciones.

\end{itemize}

\section{Problemas y soluciones del almacenamiento de la información}
\label{sec:almacenamiento}
\input{src/almacenamiento}

\section{Problemas y soluciones de la búsqueda de la información}
\label{sec:busqueda}
\input{src/busqueda}

\section{Problemas y soluciones de seguridad}
\label{sec:seguridad}
\input{src/seguridad}

\section{Integración con otras aplicaciones}
\label{sec:conectividad}
\input{src/conectividad}


\section{Definición de problemas actuales}
\label{sec:problemas}

Como vimos anteriormente, para el desarrollo de una red social distribuida, se necesitan
soluciones para sus funcionalidades básicas~\ref{sec:conclusiones_soa}. Ahora,
nos enfocaremos en presentar una arquitectura de prueba del sistema, por lo que
se abstendrá la presentación de implementaciones específicas de
éstos problemas, para pasar a tratar instancias representativas de una o más
funcionalidades.

Para ello se analizará el cómo se estructurará, organizará y
almacenará la información dentro del sistema, entre otros puntos transversales a la
implementación de éstas funcionalidades, lo cual será organizado bajo las
siguientes aristas principales:


\begin{itemize}
    \item Almacenamiento de la información.
    \item Búsquedas de la información.
    %\begin{itemize}
        %\item {Perfil de usuario}
        %\item {Publicación de contenido}
        %\item Búsqueda e identificación dentro de la red.
        %\begin{itemize}
            %\item {Lista de contactos}
            %\item {Organización de los contactos del usuario}
            %\item {Búsqueda de usuarios, grupos y contenido publicado}
        %\end{itemize}
        %\item Conectividad y disponibilidad de la información.
    %\end{itemize}
    \item Seguridad.
    %\begin{itemize}
        %\item {Configuraciones de privacidad}
    %\end{itemize}
    \item Integración con otros servicios y aplicaciones.

\end{itemize}

\section{Problemas y soluciones del almacenamiento de la información}
\label{sec:almacenamiento}
\input{src/almacenamiento}

\section{Problemas y soluciones de la búsqueda de la información}
\label{sec:busqueda}
\input{src/busqueda}

\section{Problemas y soluciones de seguridad}
\label{sec:seguridad}
\input{src/seguridad}

\section{Integración con otras aplicaciones}
\label{sec:conectividad}
\input{src/conectividad}


\section{Definición de problemas actuales}
\label{sec:problemas}

Como vimos anteriormente, para el desarrollo de una red social distribuida, se necesitan
soluciones para sus funcionalidades básicas~\ref{sec:conclusiones_soa}. Ahora,
nos enfocaremos en presentar una arquitectura de prueba del sistema, por lo que
se abstendrá la presentación de implementaciones específicas de
éstos problemas, para pasar a tratar instancias representativas de una o más
funcionalidades.

Para ello se analizará el cómo se estructurará, organizará y
almacenará la información dentro del sistema, entre otros puntos transversales a la
implementación de éstas funcionalidades, lo cual será organizado bajo las
siguientes aristas principales:


\begin{itemize}
    \item Almacenamiento de la información.
    \item Búsquedas de la información.
    %\begin{itemize}
        %\item {Perfil de usuario}
        %\item {Publicación de contenido}
        %\item Búsqueda e identificación dentro de la red.
        %\begin{itemize}
            %\item {Lista de contactos}
            %\item {Organización de los contactos del usuario}
            %\item {Búsqueda de usuarios, grupos y contenido publicado}
        %\end{itemize}
        %\item Conectividad y disponibilidad de la información.
    %\end{itemize}
    \item Seguridad.
    %\begin{itemize}
        %\item {Configuraciones de privacidad}
    %\end{itemize}
    \item Integración con otros servicios y aplicaciones.

\end{itemize}

\section{Problemas y soluciones del almacenamiento de la información}
\label{sec:almacenamiento}
\input{src/almacenamiento}

\section{Problemas y soluciones de la búsqueda de la información}
\label{sec:busqueda}
\input{src/busqueda}

\section{Problemas y soluciones de seguridad}
\label{sec:seguridad}
\input{src/seguridad}

\section{Integración con otras aplicaciones}
\label{sec:conectividad}
\input{src/conectividad}
