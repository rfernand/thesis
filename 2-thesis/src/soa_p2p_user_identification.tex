\section{P2P Identification Schemes}
%\subsection{P2P Networks and User Identification systems}
%
%The work will be based in structured P2P networks, because unstructured
%systems does not have the desired properties for the implementation of a
%identification system. 
%
%
%\paragraph{PeerSoN}
%PeerSoN presenta un prototipo de estructura para redes sociales
%Peer-to-Peer, buscando asegurar la privacidad de los usuarios y potenciar la
%comunicación directa entre cada usuario.
%Su arquitectura está basada en 2 capas, una para la búsqueda y otra para el almacenamiento de la
%información. Considera la asignación de identificadores únicos para cada
%usuario, procedimientos de entrada, envío y obtención de archivos y el manejo
%de mensajes asincrónicos.
%
%La capa de almacenamiento consiste en los peers en sí, los cuales, una vez
%encontrados por la capa de búsqueda, pasan a juntar y enviar la información
%directamente entre sí, distribuyendo réplicas de éstos
%en la red para aumentar la disponibilidad de los datos. 
%PeerSoN~\cite{buchegger:peerson} basa su seguridad en cifrado simétrico y asimétrico. La
%información primero es cifrada usando una llave simétrica, y luego esta
%llave es cifrada con las llaves públicas de los recipientes. Los
%identificadores de los usuarios son cifrados junto con las llaves
%simétricas, siendo todo enviado con la información cifrada.
%
% Por último, para la capa de búsqueda usan las capacidades de almacenamiento,
%ruteo y recuperación de información de
%OpenDHT~\cite{Rhea:2005:OPD:1080091.1080102}.
%
%%#peerson
%Dentro de sus debilidades se encuentran un sistema de seguridad y de búsquedas
%muy básicos, sin abordar las problemáticas que pueda tener en una
%implementación real del sistema. Su sistema de cifrado requiriere para poder compartir
%un dato a grupo una cantidad de procesamiento proporcional a la
%cantidad de usuarios. Además, en caso de modificación del grupo, se requeriría
%el recifrado de todos los archivos compartidos entre sí, con todos los
%costos que esto significaría. En el caso del sistema de búsqueda, al basarse en
%un DHT sin mecanismos adicionales, no permitiría búsquedas complejas o de más
%de una dimensión, dificultando el proceso de establecimiento conexiones entre
%la red.
%
%\paragraph{Safebook}
%Safebook~\cite{conf_wowmom_CutilloMO11} es un propuesta la formación para una
%red social que utiliza dos principios de diseño: descentralización y uso de la
%confianza con los usuarios de la vida real. Su arquitectura la organiza
%separándola  en 3 niveles: red social (\textit{social network}, SN), aplicación
%(\textit{application services}, AS) y transporte y comunicación
%(\textit{communication and transport}, CT). El primer nivel es provisto por la
%representación digital de los miembros y sus relaciones, la capa da aplicación
%corresponde a la infraestructura de la red social y por último la capa de
%comunicación que corresponde al Internet. 
%De ésta forma, un usuario es representado como un \textit{host node} en la
%Internet, un \textit{peer node} en la capa P2P, y un miembro o usuario en la
%red social.
%Uno de sus objetivos es ser resistente a gran parte de los posibles ataques a
%redes sociales~\ref{sec:ataques}.
%
% Confía en matryoshkas que proveen el almacenamiento de la información, perfiles de la obtención de la información
%y ofuscación de la comunicación. Una matryoshka consiste en un set de nodos
%agrupados en varios anillos concéntricos, los cuales se organizan acorde al nivel de confianza que el
%nodo, asociado con la matryoshka, tiene hacia los nodos de cada anillo. El
%anillo más interior es el más confiado y estaría formado por los ``amigos'' del
%nodo. Este sería el responsable  de almacenar la información replicada del nodo
%asociado con la matryoshka. Las capas más cercanas almacenan la información
%publicada de forma encriptada y no-encriptada, pero la información privada es
%almacenada por el mismo dueño y no es replicada hacia los anillos. 
%
%Para la autenticación del usuario, utiliza un servicio de identificación
%confiable para proveer a cada nodo identificadores únicos: un \textit{identificador del
%nodo} y un \textit{seudónimo}.
%Acuerdo a~\cite{springerlink_10.1007_978-3-642-14282-6_7}, un esquema simple de encriptación grupal es usada para la
%encriptación, y usuarios obtienen llaves oportunamente para desencriptar la
%información publicada. El dueño debe explícitamente autorizar y volver a publicar al
%anillo interior cada mensaje escrito por otros usuarios.
%La anonimidad en Safebook es obtenida usando un ruteo de múltiples saltos. Una
%tabla de hash distribuida mantiene punteros a los nodos pertenecientes al
%anillo más lejano de la matryoshka. Las peticiones entrantes son ruteadas desde
%el anillo más lejano hacia el centro de la matryoshka. Los mensajes son
%encriptados en cada salto usando criptografía asimétrica.
%
%La búsqueda de información se realiza a través de consultas recursivas en el
%sistema P2P a través del DHT. Para ello, el nodo realiza una consulta
%por la información del usuario objetivo, requiriendo el conocimiento previo de
%los identificadores de éste. De ésta forma, consulta al nodo responsable del
%identificador de ese usuario por la lista de nodos de entrada de la
%capa más lejana de la matryoshka del usuario objetivo. Por último, la consulta
%prosigue reenviandose a través de los nodos de cada capa de la matryoshka, hasta
%llegar al centro de la misma. La información requerida llega de vuelta a través
%invirtiendo el camino de llegada de la consulta.
%No contempla un sistema de búsqueda adicional al provisto por el DHT.
%
%Los problemas principales detectados de este sistema se encuentran en la
%disponibilidad de la información, problemas de escalabilidad al depender de
%sistemas externos para el ingreso de nuevos usuarios a la red y carencias en
%los métodos de búsqueda y recuperación de la información.
%Esto es debido a que el almacenar réplicas de los datos personales entre los amigos más cercanos
%puede generar problemas de disponibilidad de los datos en los momentos en que
%éstos no se encuentren en linea. Para la identificación requiere de un servicio
%externo centralizado  que pueda ser confiado para la asignación de ID únicas.
%Por último, la transmisión de mensajes a través de la red se realiza
%anonimizando al que envía y recibe el mensaje, aumentando el costo y tiempo de
%envío de cada dato por cada salto que de entre nodo y nodo.

%% introduction of identification schemes. Par 2: proposed username/password identification schemes. Part 3: securing the network protocols. Part 4: Pending issues.


\section{Keys generation}
\subsection{Randomly derived}
  manual backup of the keys
\subsection{Keys derived from a password}

\section{Resilience against malicious nodes}

 \subsection{Reputation systems}
Reputation systems mitigate the problem of malicious nodes in
P2P networks, trying to build trust among the nodes. The key
idea of a reputation system is to predict the future behaviour
of nodes based on feedback about their past transactions [1]. A
transaction is application dependent, for example forwarding a
message in the network, buying an item in e-commerce services,
share or store files, etc. After a transaction, the client node emits
a recommendation that evaluates the behaviour of the other peer.
The aggregation of these recommendations leads to a reputation
value.
A reputation system built on top of a DHT has the ability
to compute a global reputation value for every node. Indeed
all the recommendations about a single node can be handled
consistently at a common location: either by a specific node
or by a set of nodes. Among existing reputation systems for
DHTs, we can cite: PeerTrust~\cite{peertrust}, WTR~\cite{wtr},
Eigentrust~\cite{eigentrust},
PowerTrust~\cite{powertrust} and CORPS~\cite{corps}

Reputation systems have to deal with malicious nodes that:
do not participate, collude with other malicious nodes and
emit false recommendations. There are techniques to mitigate
the impact of these attacks, such as the ones presented in
TrustGuard~\cite{trustguard} and WTR~\cite{wtr}. Nevertheless, to our knowledge,
none of the existing solutions to promote trust in P2P can be
$100\%$ effective in detecting and blocking these attacks.

It's assumed an $5\%$ error in the clasification of trusted nodes.


% revise this
\paragraph{CORPS trust model}
\label{sec:corps}
They consider a probabilistic model of trust based on reputation.
The reputation value $R(X)$ is the probability that node $X$ will
be honest in the future. This reputation value is computed
according to a list of recommendations emitted by nodes that
have already carried out transactions with $X$.
After each transaction, a node emits a recommendation
about its peer. A node may lie: it may emit negative
recommendations about a peer that behaves correctly, or
positive recommendations about a malicious peer. Several nodes
may collude to increase or decrease the reputation value of
another node. These problems generate a deviation between
the computed reputation of the node and its real behaviour. It is
considered that this deviation depends on the function used to
compute the reputation value and on the percentage of malicious
nodes within the system.

It assumes a reputation system in the overlay structure with the following properties:
\begin{enumerate}
  \item Every node $X$ has an associated reputation value $R(X)$
  which represents the probability that $X$ is an honest node.
  \item $R(X)$ is computed using the recommendations emitted
  by nodes that have completed a transaction with $X$. Bad
  recommendations have a stronger effect on $R(X)$ than
  good ones. It should be more difficult for node to
  increase its reputation value than to decrease it.
  \item For every node $X$, $R(X)$ is highly available in the DHT.
\end{enumerate}

To avoid nodes that lie about a reputation value, it considers a reputation
system that computes the reputation of nodes concurrently on different nodes.
They decide individually if that node is reputable or not, using a voting
scheme in case of disagreement. On the whole, assuming that there is a smaller
percentage of malicious nodes in the network, the result avoids
false statements about reputation.

To maintain the same nomenclature used in CORPS, in all the following sections,
we call a node $n_i \in TS$ a trusted node, even if there still remains a
probability that this node's actual reputation is smaller than the threshold
$\rho$.
 
 
\subsection{Self-Certification}
A trusted authority issues identity certificates in a
centralized system. P. Dewan proposed a self-certification
mechanism [12] that splits the trusted entity among the
peers and enables them to generate their own identities in a
decentralized reputation system. Certified Authority (CA)
is run by each peer so as to issue an identity certificate(s)
for itself. These self certified certificates are similar to
SDSI certificates [9]. Each peer has its own reputation and
the reputations of all peers collectively form the reputation
of a CA.
In Self-certification mechanism there is no need for
centralized trusted entity which issues identities in a
centralized system. There is no way to map the identity of
a peer in the system to its real-life identity when they use
self-certified identities. They remain pseudonymous in the
system. The idea of making peers anonymous or
pseudonymous is desirable in P2P networks, but it can also
backfire sometimes.
In Self-certification mechanism the anonymity of the peers
is preserved by grouping of peers. The combination of self
certification and anonymity limits the possibility of a Sybil
attack. In contrast to the traditional CA-based approach,
neither the group authority nor the transacting peers can
establish the identity of the peer. In addition, certificate
revocations are not needed in the group-based approach as
the group authority only vouches for the real-life existence
of the peer, unlike the traditional certificate-based
approaches where various certificate attributes are attested
by the authority and necessitate revocation if any of those
attributes mutate in time. If a highly reputed identity is
compromised, its misuse would be self-destructive as its
reputation will go down if misused.
