Los servicios P2P son robustos, escalables y auto-organizados por naturaleza,
pero su arquitectura diferente trae nuevos problemas y requerimientos.
Tradicionalmente, las redes P2P identifican sólo los nodos que componen el
sistema, sin diferenciar a los usuarios detrás de cada uno de ellos.
Hoy en día, las personas usan más de un dispositivo para conectarse a la red.
Este cambio de comportamiento hace que una identificación por usuario sea
necesaria. Además, el usuario normal está acostumbrado al uso de nombres de
usuarios y contraseñas para identificarse en éstos sistemas.
Mientras que existen propuestas para la implementación de un sistema de
identificación por nombre de usuario y contraseña, éstos no contemplan la
existencia de nodos maliciosos, faltando los mecanismos adecuados para
asegurar el proceso de identificación.

El trabajo a continuación investiga profundamente los requerimientos y
características necesarias para un sistema de identificación seguro, junto con
los desafíos encontrados para su implementación en redes P2P.

%El objetivo principal de este trabajo es desarrollar un sistema de
%identificación de usuarios basado en nombre de usuario y contraseña en redes
%estructuradas P2P, usando sistemas de reputación y administración de nodos
%confiables para protegerse en contra de nodos maliciosos y implementar protocolos seguros
%en el sistema.

%Se realizará un análisis teórico del sistema para asegurar que los protocolos
%son seguros y no presentan riesgos en ambientes reales.
%
%La arquitectura para desarrollar el sistema será en redes P2P basadas en  Distributed Hash
%Tables (DHT).\\

{\bf Keywords:} P2P, identificación de usuario, sistemas distribuidos.
