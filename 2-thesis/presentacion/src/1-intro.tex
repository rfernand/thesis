\begin{frame}
\frametitle{P2P Networks}
\framesubtitle{Characteristics}
\begin{table}
\begin{tabular}{p{7cm}p{3cm}}
\begin{itemize}
  \item Scalable
  \item Decentralized
  \item Self-maintained
  \item Robust
\end{itemize}
&
\vspace{1.5cm}
\includegraphics[width=4cm]{img/p2p-unstructured}\\
\end{tabular}
\end{table}
\end{frame}

\begin{frame}
\frametitle{P2P Networks}
\framesubtitle{Overlay structure}
\begin{table}
\begin{tabular}{p{7cm}p{3cm}}
\begin{itemize}
    \item Structured networks (CAN, CHORD)
    \item Unstructured networks (Gnutella, Bittorrent)
\end{itemize}
&
\vspace{1.5cm}
\includegraphics[width=4cm]{img/p2p-structured}\\
\end{tabular}
\end{table}
\end{frame}

\begin{frame}
\frametitle{P2P Networks}
\framesubtitle{Overlay structure}
\begin{table}
\begin{tabular}{p{7cm}p{3cm}}
\begin{itemize}
    \item Images of structured/unstructures networks
    \item Differences
\end{itemize}
&
\vspace{1.5cm}
\includegraphics[width=4cm]{img/example}\\
\end{tabular}
\end{table}
\end{frame}

\begin{frame}
\frametitle{Username / password identification}
\framesubtitle{Why?}
\begin{table}
\begin{tabular}{p{7cm}p{3cm}}
\begin{itemize}
  \item Many \b{complex systems} require authenticated users to work.
  \item Most of the users has \b{more than one device}, so identification through
    multiple devices is needed.
  \item \b{User are accostumed} to username/password solutions.
\end{itemize}
&
\vspace{1.5cm}
\includegraphics[width=4cm]{img/users}\\
\end{tabular}
\end{table}
\end{frame}

\begin{frame}
\frametitle{Username / password identification}
\framesubtitle{P2P Identification Schemes}

  Descentralized schemes distributes the task of public key auth to all
  participants.
\begin{table}
\begin{tabular}{p{7cm}p{3cm}}
\begin{itemize}
  \item PGP-like scheme\\ Creates web of trust to auth public keys based on
    their acquaintances opinions.
  \item Quorum-based scheme\\ Multiple independent participants replicate
    public keys.
\end{itemize}
&
\vspace{1.5cm}
\includegraphics[width=4cm]{img/password}\\
\end{tabular}
\end{table}
\end{frame}

\begin{frame}
\frametitle{Identification schemes}
\framesubtitle{In need of a third party}
\begin{table}
\begin{tabular}{p{7cm}p{3cm}}
\begin{itemize}
    \item PKI
    \item Username/password pair
\end{itemize}

Poor scaling, single point of failure, heavy administration overhead.
&
\vspace{1.5cm}
\includegraphics[width=4cm]{img/keyboard_key}\\
\end{tabular}
\end{table}
\end{frame}

\begin{frame}
\frametitle{Trust in P2P networks}
\framesubtitle{Bizantine nodes}
\begin{table}
\begin{tabular}{p{7cm}p{3cm}}
Are all nodes that not behave as expected
%Multiples causes: connection problems, viruses, modified software, etc.\\
%\b{\*Nobody can be trusted}
\begin{enumerate}
    \item Faulty nodes
    \item Malicious nodes
    \item Infected nodes
\end{enumerate}
&
\vspace{1.5cm}
\includegraphics[width=4cm]{img/malicious}\\
\end{tabular}
\end{table}
\end{frame}

\begin{frame}
\frametitle{Trust in P2P networks}
\framesubtitle{Bizantine node tolerance}
\begin{table}
\begin{tabular}{p{7cm}p{3cm}}
  P2P networks can achieve byzantine fault tolerance under $1/3$ byzantine
  peers\\
  A peer is honest only if the peers executes the protocol faithfully;
  otherwise, the peer is faulty.
&
\vspace{1.5cm}
\includegraphics[width=4cm]{img/bizantine_generals_problem}\\
\end{tabular}
\end{table}
\end{frame}

