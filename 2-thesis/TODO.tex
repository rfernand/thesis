src/seguridad.tex-maliciosos dentro de la red y la reducción de usuarios \textit{free-riders},
                  en~\cite{Altmann:2009:PFS:1719850.1719891}
                  proponen la  utilización de relaciones de confianza como fórmula para reducir
                  los incentivos de éstos comportamientos y ofrecer una mayor seguridad frente a
                  éste tipo de amenazas. % TODO: Pero....... problem?
                  
                  %% TODO: PASAR A CONCLUSIONES
                  En fin, los sistemas de confianza son una solución factible a muchos de los
                  problemas que puedan surgir por nodos maliciosos en éstos tipos de sistemas
                  distribuidos, ofreciendo una capa de seguridad que actualmente no se
                  encuentra presente en las redes sociales centralizadas, permitiendo que
--
src/5-problemas.tex-    acceso y escritura de forma diferenciada para los miembros y grupos de miembros
                        de la red social, representando un problema real en la implementación de una
                        arquitectura completamente distribuida. 
                        
                        %%%% TODO: ORDENAR, nose donde poner esto :@
                        %%Para ello, la red social puede contar con un servicio o capa de almacenamiento de la
                        %%información. En redes P2P estructuradas es común la utilización de DHTs para la mantención del servicio de
                        %%almacenamiento de datos. 
                        %%Al basar el sistema de almacenamiento en un DHT, se obtienen múltiples
--
src/5-problemas.tex-    %%implementación.
                        
                        
                        
                        % TODO: Problema de control
                        %En un sistema P2P, cada nodo tiene autoridad dentro de si mismo, pero no ........
                        %Por otro lado, redes P2P permiten que cada usuario pueda almacenar localmente
                        %información relevante sobre los contactos que posee, junto con otra información
                        %privada. Ahora, si sólo eso se utiEl problema radica en que sin réplicas, esta información puede
--
src/5-problemas.tex-
                    \subsection{Integración con otros servicios y aplicaciones.}
                    \label{sec:conectividad}
                        % OAUTH
                        % TODO: Pasar a conclusiones
                        %A pesar de no ser un punto crucial para el desarrollo de una red social, la
                        %integración con aplicaciones externas facilita el crecimiento de la red.
                        Las aplicaciones externas le otorgan a la red nuevas
                        funcionalidades, otorgándole la flexibilidad para la generación de nuevos
--
src/5-problemas.tex-    %\end{itemize}
                        
                        
                        
                        %% TODO: Sacar bien el problema del paper de LIGHT
                        %% El problema está en que todavía los métodos de indexación siguen siendo muy
                        %% costosos para su implementación, creciendo exponencialmente su 
                        
                        %  En~\cite{5345647} se presenta un algoritmo de búsqueda Peer-to-Peer que utiliza el
--
src/5-problemas.tex-%    abiertamente a todos los suscriptores, siendo ellos los que pasan a filtrar el
                    %    contenido, según ciertos tópicos definidos por cada uno de los receptores de la
                    %    información.
                    %    
                    %    % TODO:
                    %    % mencionar porque estos sistemas son importantes (sistema de notificaciones y
                    %    % seguimiento de actualizaciones entre los usuarios)
                    %    % y los problemas involucrados en una red social p2p
                    %    
--
src/5-problemas.tex-    transacciones de forma correcta. Los sistemas de reputación sólo proveen la
                        habilidad para un nodo de encontrar la reputación de un nodo dado, en orden de
                        decidir si hacer o no una transacción con él.
                        
                        %% TODO: redactar mejor esto -->
                        %%Los requisitos para la implementación de éste sistema es que las estructuras
                        %%sociales que se definan tengan:
                        %%\begin{itemize}
                        %%    \item Confianza mutua entre los usuarios de la estructura.
--
src/5-problemas.tex-     Si bien se pueden construir sistemas de reputación utilizando de base las
                        relaciones entre los usuarios de la red social, dándole
                        mayor importancia a las recomendaciones realizadas por usuarios relacionados
                        con uno que con los que no mantiene ninguna relación, la confianza entre los
                        nodos no puede basarse únicamente en ello. % TODO: Completar falencias de las
                        %relaciones de confianza en el establecimiento de un sistema de reputación
                        
                        %%A continuación se pasan a analizar subproblemas que nacen de ésta problemática,
                        %%siendo analizados por separado según corresponda:
src/5-problemas.tex-    
                        %% %% TODO: sistemas de confianza
                        %%
                        %%\paragraph{Identificación de usuarios}
                        %%%diferencias con sistemas centralizados
                        %%El problema de la identificación del usuario reside en que no existen entes
--
src/6-soluciones.tex-
                     %%%%%%% reputacion
                     
                     %Entre los sistemas de reputación para redes P2P existentes, también existen
                     %algoritmos para la contrucción de  TODO
                     %CORPS~\cite{rosas2011corps} construye una comunidad de nodos con reputación y permite la
                     %implementación de servicios pseudo-confiables de forma escalable y a un bajo
                     %costo en comparación con las técnicas tradicionales. Otros sistemas de
                     %reputación dinámicos como The Buddy System~\cite{fahnrich2004buddy} aprovechan las relaciones de
--
src/1-introduccion.tex:%% TODO:
                       %% La introducción está mal organizada. Se trata de explicar lo siguiente:
                       %% - Cúal es el problema?
                       
                       Actualmente, las redes sociales han llegado a ser una de las actividades más populares en el
--
src/1-introduccion.tex-%    \item{Conclusiones y trabajo futuro}%~\ref{sec:conclusiones}
                       %\end{enumerate}
                       %% 
                       %%
                       %% TODO: Mal escrito. Ver comentario a pricipio del capítulo.
                       %%
                       %Por ello, se requiere de un diseño y arquitectura adecuada para su correcta
                       %implementación, por lo que es necesario analizar a fondo los diferentes
                       %requerimientos~\ref{sec:conclusiones_soa} de una red social,
--
src/conectividad.tex-% OAUTH
                     % TODO: Pasar a conclusiones
                     %A pesar de no ser un punto crucial para el desarrollo de una red social, la
                     %integración con aplicaciones externas facilita el crecimiento de la red.
                     Las aplicaciones externas le otorgan a la red nuevas
                     funcionalidades, otorgandole la flexibilidad para la generación de nuevos
--
src/2.2-case_studies.tex-opciones de privacidad del mismo para adecuarlas a sus necesidades. 
                             \item \textit{Configuración de privacidad} \\
                             Existen configuraciones de privacidad que permiten determinar que tan privados
                         serán la información que uno sube a la red. 
                         %% TODO: mover a seccion de privacidad de los datos de los usuarios (caracteristicas)
                         %%En los comienzos de la red social, las configuraciones por defecto
                         %%intentaban a dejar la mayor cantidad de información pública, lo que llevó a
                         %%conflictos con algunos usuarios de la red, haciendo que Facebook cambiara
                         %%varias veces sus opciones de privacidad a lo largo del tiempo, ofreciendo
--
src/2.2-case_studies.tex-ella. 
                         \end{itemize}
                         
                         \subsection{LinkedIn}
                         %% TODO:
                         %% Idem, presentar con una lista las características.
                         %% El crecimiento es exponencial?
                         %% Hay que almacenar tantos datos que en Facebook?
                         \paragraph{Historia}
--
src/almacenamiento.tex-acceso y escritura de forma diferenciada para los miembros y grupos de miembros
                       de la red social, representando un problema real en la implementación de una
                       arquitectura completamente distribuída. 
                       
                       %%%% TODO: ORDENAR, nose donde poner esto :@
                       %%Para ello, la red social puede contar con un servicio o capa de almacenamiento de la
                       %%información. En redes P2P estructuradas es común la utilización de DHTs para la mantención del servicio de
                       %%almacenamiento de datos. 
                       %%Al basar el sistema de almacenamiento en un DHT, se obtienen múltiples
--
src/2-soa_redes_sociales.tex-puede publicar de parte del mundo.
                             
                                 \item \textbf{Publicación de contenido.}
                             
                             %% TODO:
                             %% Será interesante saber en lo que sigua, si realmente hay una diferencia al
                             %% nivel del almacenamiento de estos tipos diferentes de datos...
                             Un usuario registrado comparte diferentes tipos de contenido con
                             los contactos que posee. A continuación se detallan los tipos de contenido que
--
src/2-soa_redes_sociales.tex-El contenido compartido puede ser modificado y/o eliminado por el usuario
                             que lo publicó, según sea el caso.
                             
                                 \item \textbf{Búsqueda de usuarios, grupos y contenido publicado.}
                             %% TODO:
                             %% Eso será seguramente uno de los problemas más complicados en una versión P2P.
                             Un usuario registrado puede buscar y encontrar:
                                 \begin{itemize}
                                     \item Nuevos contactos dentro de la red social.
--
src/2-soa_redes_sociales.tex-        \item Contenido público que halla sido publicado por otros usuarios.
                                 \end{itemize}
                             
                                 \item \textbf{Configuraciones de privacidad.}
                             %% TODO:
                             %% En una versión P2P, aparecerá un problema adicional de Integridad de los datos.
                             Todo contenido publicado por el usuario debe poder definirse como privado,
                             compartido con otros usuarios, o público. Si es privado, sólo el usuario debe poder tener acceso al
                             mismo; si es compartido, sólo el dueño y otros usuarios definidos previamente
