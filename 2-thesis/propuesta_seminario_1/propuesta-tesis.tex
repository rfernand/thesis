\documentclass[12pt,spanish]{article}
\usepackage[utf8]{inputenc}
\usepackage{fancyhdr}
\usepackage{graphicx}
\usepackage{amsmath}
\usepackage{hyperref}
\usepackage{wasysym}
\usepackage{boxedminipage}
\usepackage{array}
\usepackage{enumitem}
\usepackage{color}
\usepackage{cite}
%\usepackage{enumerate}
\usepackage{enumitem}

\newcommand{\pau}[1]{\textcolor{cyan}{{\bf Pau:} ``#1.''}}
\renewcommand{\figurename}{Figura}
% No space in the lists
\setlist{noitemsep}
\setlist{nolistsep}

% Marks
\newcommand{\todo}[1]{\textcolor{red}{{\bf To do:} ``#1.''}}
\newcommand{\nbody}{$N-$cuerpos }

% Page adjustement
\pdfpagewidth 8.5in
\pdfpageheight 11in
\setlength{\textwidth}{15.5cm}
\renewcommand{\familydefault}{\sfdefault}

% Fancy Header and Footer
\pagestyle{fancy}
\renewcommand{\headrulewidth}{0mm} %Para eliminar barra de header
\renewcommand{\footrulewidth}{0.4pt}
\renewcommand{\arraystretch}{1.5}
\headheight 30pt
\lfoot{Proyecto de Tesis}
\rfoot{\thepage}
\cfoot{}
\lhead{}
\rhead{}
% Header banner
\chead{\setlength{\unitlength}{1mm}
\begin{picture}(0,0)
    \put(-60,0){\includegraphics[width=100mm]{logo.jpg}}
\end{picture}}

% Abreviations
\newcommand{\bs}{\boldsymbol}
\newcommand{\tnhl}{\tabularnewline\hline}
\newcommand{\nl}{\newline}

% Thesis information
\newcommand{\thesis}{\emph{``User identification with username and password in structured P2P networks''}}
\newcommand{\name}{Rodrigo Germán Fernández Gaete}
\newcommand{\tel}{+56 9 84193413}
\newcommand{\email}{\texttt{rfernand@csrg.inf.utfsm.cl}}
\newcommand{\idate}{Primer semestre de 2013}
\newcommand{\prof}{Xavier Bonnaire}
\newcommand{\degree}{Ingeniería Civil Informática,
                     Universidad Técnica Federico Santa María, 2012}
\newcommand{\pdate}{}
\newcommand{\adate}{}
\newcommand{\tdate}{}
\newcommand{\comi}{Comite académico programa MII}

\begin{document}

%%%%%%%%%%%%%%%%%%%%%%%%%%%%%%%%%%%%%%%%%%%%%%%%%%%%%%%%%%%%%%%%%%%%%%%%%%%%%%%%%%%
\section*{PROYECTO DE TESIS}
{\huge \Square} Doctorado en Ingeniería Informática
\nl\nl\noindent
{\huge \XBox} Magíster en Ciencias de la Ingeniería Informática

\begin{center}
\begin{tabular}{|r p{0.4\textwidth}|p{0.5\textwidth}|}
\hline
  1. & {\bf Título del Proyecto de Tesis    }  & \thesis  \tnhl
  2. & {\bf Nombre del Alumno               }  & \name    \tnhl
  3. & {\bf Número de Teléfono - Celular    }  & \tel     \tnhl
  4. & {\bf Correo electrónico              }  & \email   \tnhl
  5. & {\bf Fecha de Ingreso al Programa    }  & \idate   \tnhl
  6. & {\bf Pregrado                        }  & \degree  \tnhl
  7. & {\bf Profesor Guía de Tesis          }  & \prof    \tnhl
  8. & {\bf Fecha Presentación Tema de Tesis}  & \pdate   \tnhl
  9. & {\bf Fecha Aprobación Tema de Tesis  }  & \adate   \tnhl
 10. & {\bf Fecha Tentativa de Término      }  & \tdate   \tnhl
 11. & {\bf Comisión Interna de graduación  }  & \comi    \tnhl
\end{tabular}
\end{center}
\vfill
%%%%%%%%%%%%%%%%%%%%%%%%%%%%%%%%%%%%%%%%%%%%%%%%%%%%%%%%%%%%%%%%%%%%%%%%%%%%%%%%%%%
\section{Resumen}
%Debe ser suficientemente informativo, y contener una síntesis
%del proyecto, sus objetivos, resultados esperados y palabras claves.

\fbox{
\begin{minipage}[t]{0.9\textwidth}
\setlength{\parindent}{10mm}



{\bf Keywords:} P2P, identificación de usuario, sistemas distribuidos.

\end{minipage}
}
\vfill
%%%%%%%%%%%%%%%%%%%%%%%%%%%%%%%%%%%%%%%%%%%%%%%%%%%%%%%%%%%%%%%%%%%%%%%%%%%%%%%%%%%
\section*{Abstract}
\fbox{
\begin{minipage}[b]{0.9\textwidth}
\setlength{\parindent}{10mm}

P2P services are robust, scalable and self-organized by nature, but have a
complete different architecture with new problems and unique requirements.
There are proposals to implement different user identification system, but does
not have the capabilite that the users are accustom to have today in commonly
view services; user identification based in username and password. While an
early approach exist to solve the problem, it does not contemplate the existance
of malicious nodes, lacking the mechanism to secure the identification
proccess.

 We thoroughly investigate the requirements and features of a secure
identification scheme, along with the challenges facing a P2P implementation.

The main goal of this work is to...
develop a secure user identification system based in username/password in
structured P2P systems.

Later,...

... The final goal...

 %The main features discussed are (a)
%the users login, (b) users posts, (c) users contacts list, (d) groups and organizations
%among users, (e) privacy settings and (f) integration with other services and
%applications.


%With more than 1,5 billion active users, social networks are one of the most
%popular Internet services.
%The  problem is that social networks services of today are not scalable, 
%having very high maintenance costs.
%On the other hand, P2P services are robust, scalable and self-organized by nature, but have a
%complete different architecture with new problems and unique requirements. 
% Therefore, this document focus on finding the existing problems in the
%implementation of a P2P social networking service.
% We thoroughly investigate the requirements and features of social networks, along with the
%challenges facing a P2P implementation. The main features discussed are (a)
%the users login, (b) users posts, (c) users contacts list, (d) groups and organizations
%among users, (e) privacy settings and (f) integration with other services and
%applications.
%After the analyzis of these features with their current solutions,
%we discuss the problems that hinder the satisfaction of the system requirements, ending with the identification of today's main challenges.
%Among the problems identified are (1) deficiencies in the
%implementation of complex search, which do not allow real-time
%feedback and need to improve the quality and accuracy of the results obtained,
%(2) lack of proposals for enhanced security system for these systems,
%especially related to selfish users control and encryption systems,
%and (3) vulnerabilities in the entry mechanisms of the system and (4) management of
%selfish users.\\


%\textbf{Keywords:}
{\bf Keywords:} P2P, user identification, distributed systems.

\end{minipage}
}

\vfill
%%%%%%%%%%%%%%%%%%%%%%%%%%%%%%%%%%%%%%%%%%%%%%%%%%%%%%%%%%%%%%%%%%%%%%%%%%%%%%%%%%%
\section{Formulación general de la problemática y propuesta de tesis}

%Debe contener la exposición general del problema, identificando
%claramente qué aspectos relacionados con la informática son los más
%relevantes.  Además, deberá contener el marco teórico, la discusión
%bibliográfica con sus referencias y, finalmente, su propuesta de
%tesis.  (La extensión máxima de esta sección es de hasta 5 páginas.
%En hojas adicionales incluya la lista de referencias bibliográficas
%citadas)


\subsection{Contexto y Motivación}


\subsection{Planteamiento del Problema}


\subsection{Propuesta de tesis}


\subsubsection{Algoritmo}


%\vfill

%%%%%%%%%%%%%%%%%%%%%%%%%%%%%%%%%%%
%\bibliographystyle{mn}            %
%\bibliographystyle{mn2e}            %
\bibliographystyle{ieeetr}            %
\bibliography{../../bib/article,../../bib/paper,../../bib/url}    %
%%%%%%%%%%%%%%%%%%%%%%%%%%%%%%%%%%%
\section{Hipothesis}

\fbox{
\begin{minipage}[t]{0.9\textwidth}
\setlength{\parindent}{10mm}

% here goes the hipotesis
The next work is based in the following hipostesis, under the assumption that 
P2P networks are capable of maintain highly secure services.
\begin{itemize}
    % more hipotesis/facts that the thesis is based
    \item  It is assumed that is possible to develop a secure username/password
based  user identification scheme in P2P networks. 
    \item It is assumed that trust between nodes can be achieved in P2P
networks.
    \item It is assumed that that encryption schemes available today are
    sufficient to secure user sensible information.

\end{itemize}

\vfill
\end{minipage}
}

%\section{Hipótesis de Trabajo}
%
%\fbox{
%\begin{minipage}[t]{0.9\textwidth}
%\setlength{\parindent}{10mm}
%
%% here goes the hipotesis
%The next work is based in the following hipostesis, under the assumption that 
%P2P networks are capable of maintain highly secure services.
%\begin{itemize}
%    % more hipotesis/facts that the thesis is based
%    \item  It is assumed that is possible to develop a secure username/password
%based  user identification scheme in P2P networks. 
%    \item It is assumed that trust between nodes can be achieved in P2P
%networks.
%    \item It is assumed that that encryption schemes available today are
%    sufficient to secure user sensible information.
%
%\end{itemize}
%
%\vfill
%\end{minipage}
%}
\newpage
%%%%%%%%%%%%%%%%%%%%%%%%%%%%%%%%%%%%%%%%%%%%%%%%%%%%%%%%%%%%%%%%%%%%%%%%%%%%%%%%%%%
\section{Goals}
\subsection{Main Goals}

\fbox{
\begin{minipage}[t]{0.9\textwidth}
\setlength{\parindent}{10mm}

% main objective
The implementation of a secure username/password based user identification scheme in structured P2P
networks using secure routing, building of trust between nodes and encryption
techniques.

\vspace{0.5cm}
\noindent

In particular, the main goals are:


\begin{enumerate}
    \item Have a minimal posibility of error in the identification process.
    \item Use a layer developed scheme that can easily adapt to most commonly
          used P2P networks.
\end{enumerate}
\vspace{0.5cm}

All this is fundamental to the identification protocol to maintain his desirable properties.
\end{minipage}
}

%\section{Objetivos}
%\subsection{Objetivos Generales}
%
%\fbox{
%\begin{minipage}[t]{0.9\textwidth}
%\setlength{\parindent}{10mm}
%
%% main objective
%The implementation of a secure username/password based user identification scheme in structured P2P
%networks using secure routing, building of trust between nodes and encryption
%techniques.
%
%\vspace{0.5cm}
%\noindent
%
%In particular, the main goals are:
%
%
%\begin{enumerate}
%    \item Have a minimal posibility of error in the identification process.
%    \item Use a layer developed scheme that can easily adapt to most commonly
%          used P2P networks.
%\end{enumerate}
%\vspace{0.5cm}
%
%All this is fundamental to the identification protocol to maintain his desirable properties.
%\end{minipage}
%}


\subsection{Specifics Goals}

\fbox{
\begin{minipage}[t][70mm][t]{0.9\textwidth}
\setlength{\parindent}{10mm}

In the development of the present work the following specifics goals are taken in consideration:
\vspace{0.5cm}
\begin{itemize}
    \item Study the posibility of password recovery mechanisms in the proposed
          identification scheme.
    \item Study and use of development of trust between nodes to maintain a
          trusted layer inside the P2P network.
    \item Use bizantine tolerant algorithm to verify and maintian the
          algorithms consistency in the presence of malicious nodes.
    \item Study and use of secure routing, search and storage mechanisms a 
          structured P2P network.
\end{itemize}
\end{minipage}
}

%\subsection{Objetivos Específicos}
%
%\fbox{
%\begin{minipage}[t][70mm][t]{0.9\textwidth}
%\setlength{\parindent}{10mm}
%%El desarrollo del presente trabajo de tesis considera la realización de los
%%siguientes objetivos específicos:
%In the development of the present work the following specifics goals are taken in consideration:
%\vspace{0.5cm}
%\begin{itemize}
%    \item Study the posibility of password recovery mechanisms in the proposed
%          identification scheme.
%    \item Study and use of development of trust between nodes to maintain a
%          trusted layer inside the P2P network.
%    \item Use bizantine tolerant algorithm to verify and maintian the
%          algorithms consistency in the presence of malicious nodes.
%    \item Study and use of secure routing, search and storage mechanisms a 
%          structured P2P network.
%\end{itemize}
%\end{minipage}
%}

\vfill
\section*{Metodología y Plan de Trabajo}


\fbox{
\begin{minipage}[t]{0.9\textwidth}
\setlength{\parindent}{10mm}

The working plan for the thesis development consists in three stages.

% here goes the planning!

\noindent {\bf Stage I:} Problem definition
\begin{enumerate}
    \item Study of P2P network systems and P2P search and storage mechanisms.
          \emph{(September 2012)}
    \item Study the P2P networks capabilities to implement complex systems as
          seen in centralized systems.
          \emph{(Octover 2012 - November 2012)}
    \item Born of the idea.
          \emph{(December 2013)}
    \item Problem specification, hipothesis and project objectives.
          \emph{(February 2013)}
\end{enumerate}

\noindent {\bf Stage II:} P2P Systems definition
\begin{enumerate}
    \item State of the art of P2P network systems and P2P search and storage mechanisms.
          \emph{(May 2013)}
    \item State of the art of building of trust between nodes in P2P networks.
          \emph{(April 2013 - June 2013)}
    \item State of the art of user identification schemes in P2P networks and
          how to secure the different system protocols.
          \emph{(July 2013)}
\end{enumerate}

\noindent {\bf Stage III:} Solution proposal
\begin{enumerate}
    \item User identification system design for structured P2P networks.
          \emph{(August 2013 - September 2013)}
    \item Teoric evaluation of the user identification proposal.
          \emph{(October 2013)}
    \item Final thesis report development.
          \emph{(November 2013 - December 2013 2013)}
    \item Paper development.
          \emph{(January 2014)}
\end{enumerate}
\end{minipage}
}

%\section*{Metodología y Plan de Trabajo}
%
%
%\fbox{
%\begin{minipage}[t]{0.9\textwidth}
%\setlength{\parindent}{10mm}
%El plan de trabajo para el desarrollo del presente tema de tesis consta de tres
%etapas.
%\noindent {\bf Etapa I:} Definición del problema
%\begin{enumerate}
%    \item Nacimiento de la idea.
%    \emph{(Mayo 2012)}
%    \item Estudio del problema de los \nbody, diferentes métodos de
%          integración, propiedades generales, etc.
%          \emph{(Junio 2012 - Julio 2012)}
%    \item Estudio de sistemas keplerianos, pruebas severas del integrador.
%    \emph{(Agosto 2012)}
%    \item Especificaciones del problema, hipótesis de trabajo y objetivos del
%          proyecto. \emph{(Septiembre 2012)}
%\end{enumerate}
%
%\noindent {\bf Etapa II:} Definición de la solución
%\begin{enumerate}
%    \item Estado del arte de esquemas de integración para el problema
%          de los \nbody.
%          \emph{(Octubre 2012)}
%    \item Estado del arte de integradores de sistemas de \nbody.
%          \emph{(Noviembre 2012)}
%    \item Estado del arte de la computación basada en tarjetas gráficas (GPU).
%          \emph{(Diciembre 2012)}
%    \item Diseño del algoritmo de integración que considere la diferenciación
%          en el cálculo de la fuerza sobre las estrellas.
%          \emph{(Noviembre 2012 - Diciembre 2012)}
%\end{enumerate}
%
%\noindent {\bf Etapa III:} Implementación de la solución
%\begin{enumerate}
%    \item Diseño del algoritmo de integración que considere la diferenciación
%          en el cálculo de la fuerza sobre las estrellas.
%          \emph{(August 2013 - September 2013)}
%    \item Desarrollo de experimentos y comparaciones en el desempeño de
%          ambos algoritmos.
%          \emph{(October 2013 - November 2013)}
%    \item Desarrollo del informe final de la tesis.
%          \emph{(December 2013 2013)}
%    \item Desarrollo de publicación.
%          \emph{(January 2014)}
%\end{enumerate}
%\end{minipage}
%}


\vfill
\section*{Results}
\subsection{Contributions and Expected Results}

\fbox{
%\begin{minipage}[t][70mm][t]{0.9\textwidth}
\begin{minipage}[t]{0.9\textwidth}
\setlength{\parindent}{10mm}

%expected results
%Los aportes y resultados esperados con la realización
%del presente proyecto de tesis son:

\begin{itemize}
    \item Design of a secure and modern user identification system for
          structured P2P networks.
    \item Generate a base system to develop complex projects in P2P networks.
    \item Reassure that P2P distributed systems have the capabilities to offer
          complex and high level services.
    \item Development of an article to be sent to a distributer systems
          publisher. The paper will show the results of the user identification system
          designed in this project.
\end{itemize}

\end{minipage}
}

%\vfill
%\section*{Resultados}
%\subsection{Aportes y Resultados Esperados}
%
%\fbox{
%%\begin{minipage}[t][70mm][t]{0.9\textwidth}
%\begin{minipage}[t]{0.9\textwidth}
%\setlength{\parindent}{10mm}

%\begin{itemize}
%    \item Desarrollar una solución moderna,
%          tanto computacionalmente, como en términos de ingeniería para la
%          identificación de usuarios en sistemas P2P.
%    \item Generar el sistema base para futuros proyectos
%          basados en redes P2P.
%    \item Reafirmar que los sistemas distribuidos P2P poseen las capacidades
%          necesarias para ofrecer servicios complejos y de alto nivel.
%    \item Desarrollo de un artículo para ser enviado a una revista de computación distribuida.
%          Dicho trabajo presentará los resultados obtenidos por el
%          el sistema señalado en el presente proyecto.
%\end{itemize}
%\end{minipage}
%}

\subsection{Validation procedures}

\fbox{
%\begin{minipage}[t][70mm][t]{0.9\textwidth}
\begin{minipage}[t]{0.9\textwidth}
\setlength{\parindent}{10mm}

% translate and change this
P2P networks presents a big difficulty to be tested in a real enviroment
because of the high number of nodes needed to try it out.
Therefore, instead of going after an empiric validation of the proposed
identification system, only a  theorical evaluation will be presented.
The proposed system will be compared with the other systems available at
the moment with a throughfully analisis of the security of the algorithm used.

Taking that in consideration, the theorical evaluation will be focused in:

\begin{itemize}
    \item Proving that the system will have a minimal probability of error.
    \item Proving that the system will  maintain his consistency in networks with atmost 30\% of bizantine nodes. 
\end{itemize}

\end{minipage}
}

%\subsection{Formas de Validación}
%
%\fbox{
%%\begin{minipage}[t][70mm][t]{0.9\textwidth}
%\begin{minipage}[t]{0.9\textwidth}
%\setlength{\parindent}{10mm}
%
%Ya que los algoritmos existen desde
%hace más de cuarenta años, y que hay problemas teóricos cuya solución se conoce
%con bastante exactitud, se podrá generar un mecanismo de validación directa,
%verificando el comportamiento del algoritmo para una situación determinada.
%
%Por otro lado, es necesario además comparar los resultados de nuestro
%algoritmo con...
%
%Dentro de las pruebas para validar el sistema,
%las más importantes serán:
%\begin{itemize}
%    \item %blah blah
%\end{itemize}
%
%\end{minipage}
%}


\vfill
\section[]{Resources}
\subsection{Available Resources}

\fbox{
\begin{minipage}[t][50mm][t]{0.9\textwidth}
\setlength{\parindent}{10mm}
\begin{itemize}
    \item The books and publications related to distributed systems and P2P networks provided by the \emph{UTFSM Library} will be used.
\end{itemize}
\end{minipage}
}

%\section[]{Recursos}
%\subsection{Recursos Disponibles}
%
%\fbox{
%\begin{minipage}[t][50mm][t]{0.9\textwidth}
%\setlength{\parindent}{10mm}
%\begin{itemize}
%    \item Se hará uso de los libros que la \emph{Biblioteca UTFSM} posee
%          relacionado a sistemas distribuidos y redes P2P.
%\end{itemize}
%\end{minipage}
%}

\subsection{Resources Required}

\fbox{
\begin{minipage}[t][50mm][t]{0.9\textwidth}
\setlength{\parindent}{10mm}
No more extra resources are needed.
\end{minipage}
}

%\subsection{Recursos Solicitados}
%
%\fbox{
%\begin{minipage}[t][50mm][t]{0.9\textwidth}
%\setlength{\parindent}{10mm}
%No se solicitan recursos extra.
%\end{minipage}
%}


\end{document}
