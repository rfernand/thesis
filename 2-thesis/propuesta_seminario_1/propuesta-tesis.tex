\documentclass[12pt,spanish]{article}
\usepackage[utf8]{inputenc}
\usepackage{fancyhdr}
\usepackage{graphicx}
\usepackage{amsmath}
\usepackage{hyperref}
\usepackage{wasysym}
\usepackage{boxedminipage}
\usepackage{array}
\usepackage{enumitem}
\usepackage{color}
\usepackage{cite}
%\usepackage{enumerate}
\usepackage{enumitem}

\newcommand{\pau}[1]{\textcolor{cyan}{{\bf Pau:} ``#1.''}}
\renewcommand{\figurename}{Figura}
% No space in the lists
\setlist{noitemsep}
\setlist{nolistsep}

% Marks
\newcommand{\todo}[1]{\textcolor{red}{{\bf To do:} ``#1.''}}
\newcommand{\nbody}{$N-$cuerpos }

% Page adjustement
\pdfpagewidth 8.5in
\pdfpageheight 11in
\setlength{\textwidth}{15.5cm}
\renewcommand{\familydefault}{\sfdefault}

% Fancy Header and Footer
\pagestyle{fancy}
\renewcommand{\headrulewidth}{0mm} %Para eliminar barra de header
\renewcommand{\footrulewidth}{0.4pt}
\renewcommand{\arraystretch}{1.5}
\headheight 30pt
\lfoot{Proyecto de Tesis}
\rfoot{\thepage}
\cfoot{}
\lhead{}
\rhead{}
% Header banner
\chead{\setlength{\unitlength}{1mm}
\begin{picture}(0,0)
    \put(-60,0){\includegraphics[width=100mm]{logo.jpg}}
\end{picture}}

% Abreviations
\newcommand{\bs}{\boldsymbol}
\newcommand{\tnhl}{\tabularnewline\hline}
\newcommand{\nl}{\newline}

% Thesis information
\newcommand{\thesis}{\emph{``A modular direct-summation $N$-body integrator
                             based on GPU Computing''}}
\newcommand{\name}{Rodrigo Germán Fernández Gaete}
\newcommand{\tel}{+56 9 84193413}
\newcommand{\email}{\texttt{rfernand@csrg.inf.utfsm.cl}}
\newcommand{\idate}{Primer semestre de 2013}
\newcommand{\prof}{Xavier Bonnaire}
\newcommand{\degree}{Ingeniería Civil Informática,
                     Universidad Técnica Federico Santa María, 2012}
\newcommand{\pdate}{}
\newcommand{\adate}{}
\newcommand{\tdate}{}
\newcommand{\comi}{Comite académico programa MII}

\begin{document}

%%%%%%%%%%%%%%%%%%%%%%%%%%%%%%%%%%%%%%%%%%%%%%%%%%%%%%%%%%%%%%%%%%%%%%%%%%%%%%%%%%%
\section*{PROYECTO DE TESIS}
{\huge \Square} Doctorado en Ingeniería Informática
\nl\nl\noindent
{\huge \XBox} Magíster en Ciencias de la Ingeniería Informática

\begin{center}
\begin{tabular}{|r p{0.4\textwidth}|p{0.5\textwidth}|}
\hline
  1. & {\bf Título del Proyecto de Tesis    }  & \thesis  \tnhl
  2. & {\bf Nombre del Alumno               }  & \name    \tnhl
  3. & {\bf Número de Teléfono - Celular    }  & \tel     \tnhl
  4. & {\bf Correo electrónico              }  & \email   \tnhl
  5. & {\bf Fecha de Ingreso al Programa    }  & \idate   \tnhl
  6. & {\bf Pregrado                        }  & \degree  \tnhl
  7. & {\bf Profesor Guía de Tesis          }  & \prof    \tnhl
  8. & {\bf Fecha Presentación Tema de Tesis}  & \pdate   \tnhl
  9. & {\bf Fecha Aprobación Tema de Tesis  }  & \adate   \tnhl
 10. & {\bf Fecha Tentativa de Término      }  & \tdate   \tnhl
 11. & {\bf Comisión Interna de graduación  }  & \comi    \tnhl
\end{tabular}
\end{center}
\vfill
%%%%%%%%%%%%%%%%%%%%%%%%%%%%%%%%%%%%%%%%%%%%%%%%%%%%%%%%%%%%%%%%%%%%%%%%%%%%%%%%%%%
\section{Resumen}
%Debe ser suficientemente informativo, y contener una síntesis
%del proyecto, sus objetivos, resultados esperados y palabras claves.

\fbox{
\begin{minipage}[t]{0.9\textwidth}
\setlength{\parindent}{10mm}

{\bf Palabras claves:} P2P, identificación de usuario, sistemas distribuidos.
\end{minipage}
}
\vfill
%%%%%%%%%%%%%%%%%%%%%%%%%%%%%%%%%%%%%%%%%%%%%%%%%%%%%%%%%%%%%%%%%%%%%%%%%%%%%%%%%%%
\section*{Abstract}
\fbox{
\begin{minipage}[b]{0.9\textwidth}
\setlength{\parindent}{10mm}

{\bf Keywords:} P2P, user identification, distributed systems.
\end{minipage}
}

\vfill
%%%%%%%%%%%%%%%%%%%%%%%%%%%%%%%%%%%%%%%%%%%%%%%%%%%%%%%%%%%%%%%%%%%%%%%%%%%%%%%%%%%
\section{Formulación general de la problemática y propuesta de tesis}

%Debe contener la exposición general del problema, identificando
%claramente qué aspectos relacionados con la informática son los más
%relevantes.  Además, deberá contener el marco teórico, la discusión
%bibliográfica con sus referencias y, finalmente, su propuesta de
%tesis.  (La extensión máxima de esta sección es de hasta 5 páginas.
%En hojas adicionales incluya la lista de referencias bibliográficas
%citadas)

\subsection{Contexto y Motivación}


\subsection{Planteamiento del Problema}


\subsection{Propuesta de tesis}


\subsubsection{Algoritmo}


%\vfill

%%%%%%%%%%%%%%%%%%%%%%%%%%%%%%%%%%%
%\bibliographystyle{mn}            %
%\bibliographystyle{mn2e}            %
\bibliographystyle{ieeetr}            %
\bibliography{../../bib/article,../../bib/paper,../../bib/url}    %
%%%%%%%%%%%%%%%%%%%%%%%%%%%%%%%%%%%
\section{Hipótesis de Trabajo}

\fbox{
\begin{minipage}[t]{0.9\textwidth}
\setlength{\parindent}{10mm}

% here goes the hipotesis

\begin{itemize}
    \item % more hipotesis/facts that the thesis is based
\end{itemize}

\vfill
\end{minipage}
}
\newpage
%%%%%%%%%%%%%%%%%%%%%%%%%%%%%%%%%%%%%%%%%%%%%%%%%%%%%%%%%%%%%%%%%%%%%%%%%%%%%%%%%%%
\section{Objetivos}
\subsection{Objetivos Generales}

\fbox{
\begin{minipage}[t]{0.9\textwidth}
\setlength{\parindent}{10mm}

% main objective

\vspace{0.5cm}
\noindent

% in particular, the main goals are:

\begin{enumerate}
    \item % goal 1, 2, 3...
\end{enumerate}
\vspace{0.5cm}

% all this is fundamental to the identification protocol to maintain his desirable properties.
\end{minipage}
}


\subsection{Objetivos Específicos}

\fbox{
\begin{minipage}[t][70mm][t]{0.9\textwidth}
\setlength{\parindent}{10mm}

% Specifics objectives
%El desarrollo del presente trabajo de tesis considera la realización de los
%siguientes objetivos específicos:
\vspace{0.5cm}
\begin{itemize}
    \item % goal 1, 2, 3 ...
\end{itemize}
\end{minipage}
}

\vfill
\section*{Metodología y Plan de Trabajo}


\fbox{
\begin{minipage}[t]{0.9\textwidth}
\setlength{\parindent}{10mm}
%El plan de trabajo para el desarrollo del presente tema de tesis consta de tres
%etapas.

% here goes the planning!

%\noindent {\bf Etapa I:} Definición del problema
%\begin{enumerate}
%    \item Nacimiento de la idea.
%    \emph{(Mayo 2012)}
%    \item Estudio del problema de los \nbody, diferentes métodos de
%          integración, propiedades generales, etc.
%          \emph{(Junio 2012 - Julio 2012)}
%    \item Estudio de sistemas keplerianos, pruebas severas del integrador.
%    \emph{(Agosto 2012)}
%    \item Especificaciones del problema, hipótesis de trabajo y objetivos del
%          proyecto. \emph{(Septiembre 2012)}
%\end{enumerate}
%
%\noindent {\bf Etapa II:} Definición de la solución
%\begin{enumerate}
%    \item Estado del arte de esquemas de integración para el problema
%          de los \nbody.
%          \emph{(Octubre 2012)}
%    \item Estado del arte de integradores de sistemas de \nbody.
%          \emph{(Noviembre 2012)}
%    \item Estado del arte de la computación basada en tarjetas gráficas (GPU).
%          \emph{(Diciembre 2012)}
%    \item Diseño del algoritmo de integración que considere la diferenciación
%          en el cálculo de la fuerza sobre las estrellas.
%          \emph{(Noviembre 2012 - Diciembre 2012)}
%\end{enumerate}
%
%\noindent {\bf Etapa III:} Implementación de la solución
%\begin{enumerate}
%    \item Programación serial del algoritmo diseñado utilizando C/C++.
%          \emph{(Noviembre 2012 - Diciembre 2012)}
%    \item Programación paralela del algoritmo, basado en la implementación
%          anterior, utilizando CUDA para acelerar el cálculo.
%          \emph{(Enero 2013 - Febrero 2013)}
%    \item Desarrollo de experimentos y comparaciones en el desempeño de
%          ambos algoritmos.
%          \emph{(Marzo 2013 - Abril 2013)}
%    \item Desarrollo del informe final de la tesis.
%          \emph{(Marzo 2013 - Mayo 2013)}
%    \item Desarrollo de publicación.
%          \emph{(Marzo 2013 - Mayo 2013)}
%\end{enumerate}
\end{minipage}
}


\vfill
\section*{Resultados}
\subsection{Aportes y Resultados Esperados}

\fbox{
%\begin{minipage}[t][70mm][t]{0.9\textwidth}
\begin{minipage}[t]{0.9\textwidth}
\setlength{\parindent}{10mm}

%expected results
%Los aportes y resultados esperados con la realización
%del presente proyecto de tesis son:

\begin{itemize}
    \item Desarrollar una solución moderna,
          tanto computacionalmente, como en términos de ingeniería para la
          identificación de usuarios en sistemas P2P.
    \item Generar el sistema base para futuros proyectos
          basados en redes P2P.
    \item Reafirmar que los sistemas distribuidos P2P poseen las capacidades
          necesarias para ofrecer servicios complejos y de alto nivel.
    \item Desarrollo de un artículo para ser enviado a una revista de computación distribuida.
          Dicho trabajo presentará los resultados obtenidos por el
          el sistema señalado en el presente proyecto.
\end{itemize}



\end{minipage}
}

\subsection{Formas de Validación}

\fbox{
%\begin{minipage}[t][70mm][t]{0.9\textwidth}
\begin{minipage}[t]{0.9\textwidth}
\setlength{\parindent}{10mm}

% translate and change this
Ya que los algoritmos existen desde
hace más de cuarenta años, y que hay problemas teóricos cuya solución se conoce
con bastante exactitud, se podrá generar un mecanismo de validación directa,
verificando el comportamiento del algoritmo para una situación determinada.

Por otro lado, es necesario además comparar los resultados de nuestro
algoritmo con...

Dentro de las pruebas para validar el sistema,
las más importantes serán:
\begin{itemize}
    \item %blah blah
\end{itemize}

\end{minipage}
}


\vfill
\section[]{Recursos}
\subsection{Recursos Disponibles}

\fbox{
\begin{minipage}[t][50mm][t]{0.9\textwidth}
\setlength{\parindent}{10mm}
\begin{itemize}
    \item Se hará uso de los libros que la \emph{Biblioteca UTFSM} posee
          relacionado a sistemas distribuidos y redes P2P.
\end{itemize}
\end{minipage}
}

\subsection{Recursos Solicitados}

\fbox{
\begin{minipage}[t][50mm][t]{0.9\textwidth}
\setlength{\parindent}{10mm}
No se solicitan recursos extra.
\end{minipage}
}


\end{document}
