\documentclass[12pt,spanish]{article}
\usepackage[utf8]{inputenc}
\usepackage{fancyhdr}
\usepackage{graphicx}
\usepackage{amsmath}
\usepackage{hyperref}
\usepackage{wasysym}
\usepackage{boxedminipage}
\usepackage{array}
\usepackage{enumitem}
\usepackage{color}
\usepackage{cite}
%\usepackage{enumerate}
\usepackage{enumitem}

\newcommand{\pau}[1]{\textcolor{cyan}{{\bf Pau:} ``#1.''}}
\renewcommand{\figurename}{Figura}
% No space in the lists
\setlist{noitemsep}
\setlist{nolistsep}

% Marks
\newcommand{\todo}[1]{\textcolor{red}{{\bf To do:} ``#1.''}}
\newcommand{\nbody}{$N-$cuerpos }

% Page adjustement
\pdfpagewidth 8.5in
\pdfpageheight 11in
\setlength{\textwidth}{15.5cm}
\renewcommand{\familydefault}{\sfdefault}

% Fancy Header and Footer
\pagestyle{fancy}
\renewcommand{\headrulewidth}{0mm} %Para eliminar barra de header
\renewcommand{\footrulewidth}{0.4pt}
\renewcommand{\arraystretch}{1.5}
\headheight 30pt
\lfoot{Proyecto de Tesis}
\rfoot{\thepage}
\cfoot{}
\lhead{}
\rhead{}
% Header banner
\chead{\setlength{\unitlength}{1mm}
\begin{picture}(0,0)
    \put(-60,0){\includegraphics[width=100mm]{logo.jpg}}
\end{picture}}

% Abreviations
\newcommand{\bs}{\boldsymbol}
\newcommand{\tnhl}{\tabularnewline\hline}
\newcommand{\nl}{\newline}

% Thesis information
\newcommand{\thesis}{\emph{``User identification with username and password in structured P2P networks''}}
\newcommand{\name}{Rodrigo Germán Fernández Gaete}
\newcommand{\tel}{+56 9 84193413}
\newcommand{\email}{\texttt{rfernand@csrg.inf.utfsm.cl}}
\newcommand{\idate}{Primer semestre de 2013}
\newcommand{\prof}{Xavier Bonnaire}
\newcommand{\degree}{Ingeniería Civil Informática,
                     Universidad Técnica Federico Santa María, 2012}
\newcommand{\pdate}{}
\newcommand{\adate}{}
\newcommand{\tdate}{}
\newcommand{\comi}{Comite académico programa MII}

\begin{document}

%%%%%%%%%%%%%%%%%%%%%%%%%%%%%%%%%%%%%%%%%%%%%%%%%%%%%%%%%%%%%%%%%%%%%%%%%%%%%%%%%%%
\section*{PROYECTO DE TESIS}
{\huge \Square} Doctorado en Ingeniería Informática
\nl\nl\noindent
{\huge \XBox} Magíster en Ciencias de la Ingeniería Informática

\begin{center}
\begin{tabular}{|r p{0.4\textwidth}|p{0.5\textwidth}|}
\hline
  1. & {\bf Título del Proyecto de Tesis    }  & \thesis  \tnhl
  2. & {\bf Nombre del Alumno               }  & \name    \tnhl
  3. & {\bf Número de Teléfono - Celular    }  & \tel     \tnhl
  4. & {\bf Correo electrónico              }  & \email   \tnhl
  5. & {\bf Fecha de Ingreso al Programa    }  & \idate   \tnhl
  6. & {\bf Pregrado                        }  & \degree  \tnhl
  7. & {\bf Profesor Guía de Tesis          }  & \prof    \tnhl
  8. & {\bf Fecha Presentación Tema de Tesis}  & \pdate   \tnhl
  9. & {\bf Fecha Aprobación Tema de Tesis  }  & \adate   \tnhl
 10. & {\bf Fecha Tentativa de Término      }  & \tdate   \tnhl
 11. & {\bf Comisión Interna de graduación  }  & \comi    \tnhl
\end{tabular}
\end{center}
\vfill
%%%%%%%%%%%%%%%%%%%%%%%%%%%%%%%%%%%%%%%%%%%%%%%%%%%%%%%%%%%%%%%%%%%%%%%%%%%%%%%%%%%
\section{Resumen}
%Debe ser suficientemente informativo, y contener una síntesis
%del proyecto, sus objetivos, resultados esperados y palabras claves.

\fbox{
\begin{minipage}[t]{0.9\textwidth}
\setlength{\parindent}{10mm}



{\bf Keywords:} P2P, identificación de usuario, sistemas distribuidos.

\end{minipage}
}
\vfill
%%%%%%%%%%%%%%%%%%%%%%%%%%%%%%%%%%%%%%%%%%%%%%%%%%%%%%%%%%%%%%%%%%%%%%%%%%%%%%%%%%%
\section*{Abstract}
\fbox{
\begin{minipage}[b]{0.9\textwidth}
\setlength{\parindent}{10mm}

P2P services are robust, scalable and self-organized by nature, but have a
complete different architecture with new problems and unique requirements.
There are proposals to implement different user identification system, but does
not have the capabilite that the users are accustom to have today in commonly
view services; user identification based in username and password. While an
early approach exist to solve the problem, it does not contemplate the existance
of malicious nodes, lacking the mechanism to secure the identification
proccess.

 We thoroughly investigate the requirements and features of a secure
identification scheme, along with the challenges facing a P2P implementation.

The main goal of this work is to...
develop a secure user identification system based in username/password in
structured P2P systems.

Later,...

... The final goal...

 %The main features discussed are (a)
%the users sign in, (b) users posts, (c) users contacts list, (d) groups and organizations
%among users, (e) privacy settings and (f) integration with other services and
%applications.


%With more than 1,5 billion active users, social networks are one of the most
%popular Internet services.
%The  problem is that social networks services of today are not scalable, 
%having very high maintenance costs.
%On the other hand, P2P services are robust, scalable and self-organized by nature, but have a
%complete different architecture with new problems and unique requirements. 
% Therefore, this document focus on finding the existing problems in the
%implementation of a P2P social networking service.
% We thoroughly investigate the requirements and features of social networks, along with the
%challenges facing a P2P implementation. The main features discussed are (a)
%the users sign in, (b) users posts, (c) users contacts list, (d) groups and organizations
%among users, (e) privacy settings and (f) integration with other services and
%applications.
%After the analyzis of these features with their current solutions,
%we discuss the problems that hinder the satisfaction of the system requirements, ending with the identification of today's main challenges.
%Among the problems identified are (1) deficiencies in the
%implementation of complex search, which do not allow real-time
%feedback and need to improve the quality and accuracy of the results obtained,
%(2) lack of proposals for enhanced security system for these systems,
%especially related to selfish users control and encryption systems,
%and (3) vulnerabilities in the entry mechanisms of the system and (4) management of
%selfish users.\\


%\textbf{Keywords:}
{\bf Keywords:} P2P, user identification, distributed systems.

\end{minipage}
}

\vfill
%%%%%%%%%%%%%%%%%%%%%%%%%%%%%%%%%%%%%%%%%%%%%%%%%%%%%%%%%%%%%%%%%%%%%%%%%%%%%%%%%%%
\section{General formulation of the problem and thesis proposal}

%Debe contener la exposición general del problema, identificando
%claramente qué aspectos relacionados con la informática son los más
%relevantes.  Además, deberá contener el marco teórico, la discusión
%bibliográfica con sus referencias y, finalmente, su propuesta de
%tesis.  (La extensión máxima de esta sección es de hasta 5 páginas.
%En hojas adicionales incluya la lista de referencias bibliográficas
%citadas)


\subsection{Context and Motivation}

A Peer-to-Peer network (onwards called \textit{P2P}) is a distributed system of big scale. His participants
are called \textit{nodes} and they directly share resources and data, acting
like clients and servers. They are called big scale because they are made to
contain millions of nodes through the Internet. P2P systems are scalable,
decentralized, robust and self-organized.

P2P systems are characterized by do not having a central coordination. Each
peer is independient and has a local view of the system. The global behavior
emerges from the local interaction of its members~\cite{Aberer:2001:PIS:503271.503268}.
P2P networks natural properties make non-profitable services capable of running
with the help of the same users that use it. 

The most basic P2P systems only provide the structure to anyone in the network
to store and retrieve data in it. To provide more complex functionalities,
additional logic has to be implemented to them.
As P2P systems grow in funciontalities, the need to identify users inside the
network arose.

The subject of securely establishing stable identities in P2P
systems has been previously studied, for instance by Aberer,
Datta and Hauswirth~\cite{1318567}. The need for identities mainly arose
from technical concerns, such as handling dynamic IP address
assignment, or avoiding Sybil attacks~\cite{the_sybil_attack}. Authentication of a
node is done via a signature key, automatically generated and
stored on the node. Traditionally, P2P networks identified the different nodes that compose the
system, but not the user behind each one of them as a different being.

The increase of number of devices the people have and use to join the network
today makes the per node identification obsolete when trying to identify the
unique user behind them. An example of when this would be needed, is the
backups systems. They need to store important data in the network and then
restore it on a different system from where it was backed up.

%example
To store the data safely all approaches build on encrypting backed up content.
The schemes to identify the user are basically two: the ones that
use keys randomly derived~\cite{Lillibridge:2003:CIB:1247340.1247343} and the schemes that derive
keys from a user defined password~\cite{Cox:2002:PMB:844128.844155}. While
approachs that derive randomly the keys does not have the risk of someone
guessing the user keys, they require that the user manually back up the
keys. 
% TODO: check this system
P2P storage systems that implements the use of keyword strings to derive a
public-private key pair whose private key is used to sign data and the hash of
the public key to identify the data in the storage. Both of
these systems use a keyword string as a seed to a pseudo-
random number generator that produces the key pair ~\cite{clarke2010private},~\cite{Bennett03anencoding}.
Knowing only the memorable keyword string the user can
store and retrieve information.

%TODO: check vu et al. 11 and Frykholm and Juels AND CHANGE THIS
Related to forgotten passwords functionalities for the system, recovery of
information in a P2P scenario has been studied by Vu et al.~\cite{5380695} who proposed
a combination of threshold-based secret sharing with delegate
selection and encrypting shares with passwords.
Frykholm and Juels~\cite{Frykholm:2001:EPR:501983.501985} proposed a password-recovery
mechanism based on security questions very similar to our
protocol for the same task. They offer better, information-
theoretic security properties, something not applicable to our
scenario. We treat the subject of password change, which is not applicable to
their scenario, although their proposal could be extended to support password
change using our techniques.
% passwords in p2p networks
Also, approachs to handle
remembered sign ins, and recovering lost passwords has been
made~\cite{kreitz2012passwords}.

\subsection{Problem statement}

Most of existing systems for the user identification in P2P networks only consider the
use of preshared keys to identificate the user in the network. While that can
be easily implemented, does not provide to the users the flexibility that a
username-password based identification provides when using different devices to
sign in in the system. As the user needs to transfer manually his keys from one
device to another, there are many security issues when they are handled without
care or the devices (like a cellphone) are lost. 

The use of a username and a password means that the user keys needs to be secured inside the identification system.
To handle the user keys without compromising the users identity, aditional
security layers needs to be placed inside the P2P network.

While a solution based in username and password has been proposed
before~\cite{kreitz2012passwords}, it does not take in consideration the
precense of malicious nodes. A malicious or bizantine node is any node that does
not behave as expected by the protocol of the system. The presence of this type
of nodes can easily break the security and functioning of the whole system.


\subsection{Thesis proposal}

Before developing an P2P system with the desired functionalities, an adecuate
system architecture is needed. We will throughfully evaluate a new user
identification system that can work in the precense of bizantine nodes, with
the hopes to reach the desirable functionalities with a minimun probability of
failure to make the system secure in real scenarios.

\subsubsection{System architecture}

For the design of the system's algorithms our work will be based on structured
networks based on Distributed Hash Tables (DHT), which provide efficient key
lookups, high data availability and persistence.

To mitigate the problem of malicious nodes, we can use reputation systems to
build trust among the nodes. The key
idea of a reputation system is to predict the future behaviour
of nodes based on feedback about their past
transactions~\cite{Resnick:2000:RS:355112.355122}. A
transaction is application dependent, for example forwarding a
message in the network, buying an item in e-commerce services,
share or store files, etc. After a transaction, the client node emits
a recommendation that evaluates the behaviour of the other peer.
The aggregation of these recommendations leads to a reputation
value.
A reputation system built on top of a DHT has the ability
to compute a global reputation value for every node. Indeed
all the recommendations about a single node can be handled
consistently at a common location: either by a specific node
or by a set of nodes.

% Among existing reputation systems for
%DHTs, we can cite: PeerTrust~\cite{peertrust}, WTR~\cite{wtr},
%Eigentrust~\cite{eigentrust}, PowerTrust~\cite{powertrust}.
To build a group of trusted nodes, the CORPS~\cite{rosas2011corps} algorithm
presents an efficient solution to to build a scalable
trusted ring within a DHT that allows to find reputable
peers.


To secure the stored keys, the proposed system uses encryption, indirection and
rings of trust inside the network. The system goal is to offer a secure mean to
identify a user using only his username/password knowledge taking in
consideration the precense of bizantine nodes.

\subsubsection{Protocols}
For the user identification system, the following protocols will be used as a
base for his development, as further security mechanisms will be needed to be
added to secure each one from malicious nodes.

\paragraph{Account registration}

To register a new user account, the user first
has to choose a \textit{username} and a \textit{password}.

Considering a key-based authentication, the user creates a \textit{key store file}, containing all the
keys used by the P2P application the user wants to sign in to.
The user generates a cryptographic key to authenticate the write operations
that will be made in the file, and store this key along with the others in the
\textit{key store file}.


% encryption and store of the key store file
The user then creates a \textit{symmetric key $K_{KS}$} ,
encrypts the file content with this key and puts the ciphertext
into the storage, obtaining a \textit{file name $f_{KS}$} . Now, the user
creates a \textit{sign in information file} by creating a random
byte string \textit{salt}, deriving a \textit{symmetric key $K_{LI}$} from the user
\textit{password} and the \textit{salt}.
Using the new \textit{ symmetric key $K_{LI}$}, the user encrypts the
\textit{file name $f_{KS}$},
the \textit{symmetric key $K_{KS}$} and the \textit{cryptographic key to
authenticate the write operations $K_W$}.
 The salt and the three encrypted values are put
into the storage, obtaining a file name $f_{LI}$ . The salt is stored
in plaintext, so that the user later can derive the decryption
key $K_{LI}$ by only providing the password. Finally, the user
performs the write-once operation put on the DHT with
uname as key and $f_{LI}$ as value.

%unique username
If the username was taken,
the user is prompted for a new username.

%finish
Once all operations
have succeeded, the user is registered in the system.



\paragraph{Sign in}
The user uses his username to find and retrieve his \textit{sign in information
file}. Then, using his \textit{password} and the \textit{salt} included in the
\textit{sign in information file}, obtains the \textit{file name $f_{KS}$} used to
route back to where the \textit{key store file} is stored.  Lastly, uses the
\textit{symmectric key $K_{KS}$} to dencrypt the \textit{key store file} and recover
his user keys.

\paragraph{Logout}
The system does not have something like a ``session'' to maintain; the only way
to identify a user is by his keys that are obtained by the identification
process.


\paragraph{Password Change}
To change the password, the user has to rewrite his \textit{sign in information
file}.

Before the user can change the password, she must sign in using her password to
obtain $K_{LI}$ . With this information, the password change can be accomplished:
the user is asked for a new password and a new salt is
generated. The key-derivation function is used to generate a new key
$K_{LInew}$
for the sign in information file. Then, the content of the key-store file is
fetched and decrypted (with the old key). A new key $K_{KSnew}$ is generated and
used for encrypting the key-store content again before it is saved to the
storage system, obtaining a new filename $f_{KSnew}$.
Finally, the sign in information file
is updated: $f_{KSnew}$, $K_{KSnew}$, the write credential KW as well as a new empty
device mapping devmapnew are encrypted with the new key $K_{LInew}$.
  Together with the new salt, this ciphertext is written to the distributed
storage, using the reference $f_{LI}$ and the credential $K_W$, to authenticate the
write operation. Lastly, the keys stored in the key store should be updated by
the application using our P2P protocol.



%\section{Formulación general de la problemática y propuesta de tesis}
%
%%Debe contener la exposición general del problema, identificando
%%claramente qué aspectos relacionados con la informática son los más
%%relevantes.  Además, deberá contener el marco teórico, la discusión
%%bibliográfica con sus referencias y, finalmente, su propuesta de
%%tesis.  (La extensión máxima de esta sección es de hasta 5 páginas.
%%En hojas adicionales incluya la lista de referencias bibliográficas
%%citadas)
%
%
%\subsection{Contexto y Motivación}
%
%
%\subsection{Planteamiento del Problema}
%
%
%\subsection{Propuesta de tesis}
%
%


%\vfill

%%%%%%%%%%%%%%%%%%%%%%%%%%%%%%%%%%%
%\bibliographystyle{mn}            %
%\bibliographystyle{mn2e}            %
\bibliographystyle{ieeetr}            %
\bibliography{../../bib/article,../../bib/paper,../../bib/url}    %
%%%%%%%%%%%%%%%%%%%%%%%%%%%%%%%%%%%
\section{Working Hypothesis}

\fbox{
\begin{minipage}[t]{0.9\textwidth}
\setlength{\parindent}{10mm}

% here goes the hypotesis
The next work is based in the following hypothesis, under the assumption that 
P2P networks are capable of maintain highly secure services.
\begin{itemize}
    % more hypotesis/facts that the thesis is based
    \item  It is assumed that is possible to develop a secure username/password
based  user identification scheme in P2P networks. 
    \item It is assumed that a certain amount of trust between nodes can be achieved in P2P
networks.
    \item It is assumed that that encryption schemes available today are
    sufficient to secure user sensible information.

\end{itemize}

\vfill
\end{minipage}
}

%\section{Hipótesis de Trabajo}
%
%\fbox{
%\begin{minipage}[t]{0.9\textwidth}
%\setlength{\parindent}{10mm}
%
%The next work is based in the following hipostesis, under the assumption that 
%P2P networks are capable of maintain highly secure services.
%\begin{itemize}
%    \item  It is assumed that is possible to develop a secure username/password
%based  user identification scheme in P2P networks. 
%    \item It is assumed that trust between nodes can be achieved in P2P
%networks.
%    \item It is assumed that that encryption schemes available today are
%    sufficient to secure user sensible information.
%
%\end{itemize}
%
%\vfill
%\end{minipage}
%}
\newpage
%%%%%%%%%%%%%%%%%%%%%%%%%%%%%%%%%%%%%%%%%%%%%%%%%%%%%%%%%%%%%%%%%%%%%%%%%%%%%%%%%%%
\section{Goals}
\subsection{Main Goals}

\fbox{
\begin{minipage}[t]{0.9\textwidth}
\setlength{\parindent}{10mm}

% main objective
The implementation of a secure username/password based user identification scheme in structured P2P
networks using secure routing, building of trust between nodes and encryption
techniques.

\vspace{0.5cm}
\noindent

In particular, the main goals are:


\begin{enumerate}
    \item Have a minimal posibility of error in the identification process.
    \item Use a layer developed scheme that can easily adapt to most commonly
          used P2P networks.
\end{enumerate}
\vspace{0.5cm}

All this is fundamental to the identification protocol to maintain his desirable properties.
\end{minipage}
}

%\section{Objetivos}
%\subsection{Objetivos Generales}
%
%\fbox{
%\begin{minipage}[t]{0.9\textwidth}
%\setlength{\parindent}{10mm}
%
%% main objective
%The implementation of a secure username/password based user identification scheme in structured P2P
%networks using secure routing, building of trust between nodes and encryption
%techniques.
%
%\vspace{0.5cm}
%\noindent
%
%In particular, the main goals are:
%
%
%\begin{enumerate}
%    \item Have a minimal posibility of error in the identification process.
%    \item Use a layer developed scheme that can easily adapt to most commonly
%          used P2P networks.
%\end{enumerate}
%\vspace{0.5cm}
%
%All this is fundamental to the identification protocol to maintain his desirable properties.
%\end{minipage}
%}


\subsection{Specifics Goals}

\fbox{
\begin{minipage}[t][70mm][t]{0.9\textwidth}
\setlength{\parindent}{10mm}

In the development of the present work the following specifics goals are taken in consideration:
\vspace{0.5cm}
\begin{itemize}
    \item Study the posibility of password recovery mechanisms in the proposed
          identification scheme.
    \item Study and use of development of trust between nodes to maintain a
          trusted layer inside the P2P network.
    \item Use bizantine tolerant algorithm to verify and maintian the
          algorithms consistency in the presence of malicious nodes.
    \item Study and use of secure routing, search and storage mechanisms a 
          structured P2P network.
\end{itemize}
\end{minipage}
}

%\subsection{Objetivos Específicos}
%
%\fbox{
%\begin{minipage}[t][70mm][t]{0.9\textwidth}
%\setlength{\parindent}{10mm}
%%El desarrollo del presente trabajo de tesis considera la realización de los
%%siguientes objetivos específicos:
%In the development of the present work the following specifics goals are taken in consideration:
%\vspace{0.5cm}
%\begin{itemize}
%    \item Study the posibility of password recovery mechanisms in the proposed
%          identification scheme.
%    \item Study and use of development of trust between nodes to maintain a
%          trusted layer inside the P2P network.
%    \item Use bizantine tolerant algorithm to verify and maintian the
%          algorithms consistency in the presence of malicious nodes.
%    \item Study and use of secure routing, search and storage mechanisms a 
%          structured P2P network.
%\end{itemize}
%\end{minipage}
%}

\vfill
\section*{Metodología y Plan de Trabajo}


\fbox{
\begin{minipage}[t]{0.9\textwidth}
\setlength{\parindent}{10mm}

The working plan for the thesis development consists in three stages.

% here goes the planning!

\noindent {\bf Stage I:} Problem definition
\begin{enumerate}
    \item Study of P2P network systems and P2P search and storage mechanisms.
          \emph{(September 2012)}
    \item Study the P2P networks capabilities to implement complex systems as
          seen in centralized systems.
          \emph{(Octover 2012 - November 2012)}
    \item Born of the idea.
          \emph{(December 2013)}
    \item Problem specification, hipothesis and project objectives.
          \emph{(February 2013)}
\end{enumerate}

\noindent {\bf Stage II:} P2P Systems definition
\begin{enumerate}
    \item State of the art of P2P network systems and P2P search and storage mechanisms.
          \emph{(May 2013)}
    \item State of the art of building of trust between nodes in P2P networks.
          \emph{(April 2013 - June 2013)}
    \item State of the art of user identification schemes in P2P networks and
          how to secure the different system protocols.
          \emph{(July 2013)}
\end{enumerate}

\noindent {\bf Stage III:} Solution proposal
\begin{enumerate}
    \item User identification system design for structured P2P networks.
          \emph{(August 2013 - September 2013)}
    \item Teoric evaluation of the user identification proposal.
          \emph{(October 2013)}
    \item Final thesis report development.
          \emph{(November 2013 - December 2013 2013)}
    \item Paper development.
          \emph{(January 2014)}
\end{enumerate}
\end{minipage}
}

%\section*{Metodología y Plan de Trabajo}
%
%
%\fbox{
%\begin{minipage}[t]{0.9\textwidth}
%\setlength{\parindent}{10mm}
%El plan de trabajo para el desarrollo del presente tema de tesis consta de tres
%etapas.
%\noindent {\bf Etapa I:} Definición del problema
%\begin{enumerate}
%    \item Nacimiento de la idea.
%    \emph{(Mayo 2012)}
%    \item Estudio del problema de los \nbody, diferentes métodos de
%          integración, propiedades generales, etc.
%          \emph{(Junio 2012 - Julio 2012)}
%    \item Estudio de sistemas keplerianos, pruebas severas del integrador.
%    \emph{(Agosto 2012)}
%    \item Especificaciones del problema, hipótesis de trabajo y objetivos del
%          proyecto. \emph{(Septiembre 2012)}
%\end{enumerate}
%
%\noindent {\bf Etapa II:} Definición de la solución
%\begin{enumerate}
%    \item Estado del arte de esquemas de integración para el problema
%          de los \nbody.
%          \emph{(Octubre 2012)}
%    \item Estado del arte de integradores de sistemas de \nbody.
%          \emph{(Noviembre 2012)}
%    \item Estado del arte de la computación basada en tarjetas gráficas (GPU).
%          \emph{(Diciembre 2012)}
%    \item Diseño del algoritmo de integración que considere la diferenciación
%          en el cálculo de la fuerza sobre las estrellas.
%          \emph{(Noviembre 2012 - Diciembre 2012)}
%\end{enumerate}
%
%\noindent {\bf Etapa III:} Implementación de la solución
%\begin{enumerate}
%    \item Diseño del algoritmo de integración que considere la diferenciación
%          en el cálculo de la fuerza sobre las estrellas.
%          \emph{(August 2013 - September 2013)}
%    \item Desarrollo de experimentos y comparaciones en el desempeño de
%          ambos algoritmos.
%          \emph{(October 2013 - November 2013)}
%    \item Desarrollo del informe final de la tesis.
%          \emph{(December 2013 2013)}
%    \item Desarrollo de publicación.
%          \emph{(January 2014)}
%\end{enumerate}
%\end{minipage}
%}


\vfill
\section*{Results}
\subsection{Contributions and Expected Results}

\fbox{
%\begin{minipage}[t][70mm][t]{0.9\textwidth}
\begin{minipage}[t]{0.9\textwidth}
\setlength{\parindent}{10mm}

%expected results
%Los aportes y resultados esperados con la realización
%del presente proyecto de tesis son:

\begin{itemize}
    \item Design of a secure and modern user identification system for
          structured P2P networks.
    \item Generate a base system to develop complex projects in P2P networks.
    \item Reassure that P2P distributed systems have the capabilities to offer
          complex and high level services.
    \item Development of an article to be sent to a distributer systems
          publisher. The paper will show the results of the user identification system
          designed in this project.
\end{itemize}

\end{minipage}
}

%\vfill
%\section*{Resultados}
%\subsection{Aportes y Resultados Esperados}
%
%\fbox{
%%\begin{minipage}[t][70mm][t]{0.9\textwidth}
%\begin{minipage}[t]{0.9\textwidth}
%\setlength{\parindent}{10mm}

%\begin{itemize}
%    \item Desarrollar una solución moderna,
%          tanto computacionalmente, como en términos de ingeniería para la
%          identificación de usuarios en sistemas P2P.
%    \item Generar el sistema base para futuros proyectos
%          basados en redes P2P.
%    \item Reafirmar que los sistemas distribuidos P2P poseen las capacidades
%          necesarias para ofrecer servicios complejos y de alto nivel.
%    \item Desarrollo de un artículo para ser enviado a una revista de computación distribuida.
%          Dicho trabajo presentará los resultados obtenidos por el
%          el sistema señalado en el presente proyecto.
%\end{itemize}
%\end{minipage}
%}

\subsection{Validation procedures}

\fbox{
%\begin{minipage}[t][70mm][t]{0.9\textwidth}
\begin{minipage}[t]{0.9\textwidth}
\setlength{\parindent}{10mm}

% translate and change this
P2P networks presents a big difficulty to be tested in a real enviroment
because of the high number of nodes needed to try it out.
Therefore, instead of going after an empiric validation of the proposed
identification system, only a  theorical evaluation will be presented.
The proposed system will be compared with the other systems available at
the moment with a throughfully analisis of the security of the algorithm used.

Taking that in consideration, the theorical evaluation will be focused in:

\begin{itemize}
    \item Proving that the system will have a minimal probability of error.
    \item Proving that the system will  maintain his consistency in networks with atmost 30\% of bizantine nodes. 
\end{itemize}

\end{minipage}
}

%\subsection{Formas de Validación}
%
%\fbox{
%%\begin{minipage}[t][70mm][t]{0.9\textwidth}
%\begin{minipage}[t]{0.9\textwidth}
%\setlength{\parindent}{10mm}
%
%Ya que los algoritmos existen desde
%hace más de cuarenta años, y que hay problemas teóricos cuya solución se conoce
%con bastante exactitud, se podrá generar un mecanismo de validación directa,
%verificando el comportamiento del algoritmo para una situación determinada.
%
%Por otro lado, es necesario además comparar los resultados de nuestro
%algoritmo con...
%
%Dentro de las pruebas para validar el sistema,
%las más importantes serán:
%\begin{itemize}
%    \item %blah blah
%\end{itemize}
%
%\end{minipage}
%}


\vfill
\section[]{Resources}
\subsection{Available Resources}

\fbox{
\begin{minipage}[t][50mm][t]{0.9\textwidth}
\setlength{\parindent}{10mm}
\begin{itemize}
    \item The books and publications related to distributed systems and P2P networks provided by the \emph{UTFSM Library} will be used.
\end{itemize}
\end{minipage}
}

%\section[]{Recursos}
%\subsection{Recursos Disponibles}
%
%\fbox{
%\begin{minipage}[t][50mm][t]{0.9\textwidth}
%\setlength{\parindent}{10mm}
%\begin{itemize}
%    \item Se hará uso de los libros que la \emph{Biblioteca UTFSM} posee
%          relacionado a sistemas distribuidos y redes P2P.
%\end{itemize}
%\end{minipage}
%}

\subsection{Resources Required}

\fbox{
\begin{minipage}[t][50mm][t]{0.9\textwidth}
\setlength{\parindent}{10mm}
No more extra resources are needed.
\end{minipage}
}

%\subsection{Recursos Solicitados}
%
%\fbox{
%\begin{minipage}[t][50mm][t]{0.9\textwidth}
%\setlength{\parindent}{10mm}
%No se solicitan recursos extra.
%\end{minipage}
%}


\end{document}
