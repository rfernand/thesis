

%4.2.3. Limiting the Number of Results to Process with Top k Approaches. Processing only a
%subset of items during the search process can yield performance benefits: less data
%processing and lower latency. Various algorithms, discussed shortly, can be used to
%retrieve the top items for a particular query without having to calculate the scores for
%all the items. Retrieving top items makes sense as it has been shown that users of
%Web search engines prefer quality over quantity with respect to search results: more
%precision and less recall [Oulasvirta et al. 2009]. Top k approaches have been applied
%to various architectures and at various stages in peer-to-peer information retrieval:

Al realizar una búsqueda, los usuarios prefieren calidad por sobre cantidad de
resultados. Por ello, en vez de entregar millones de resultados para una
consulta, lo que se busca es obtener los objetos más relevantes para el usuario~\citealt{oulasvirta2009more}. %[Oulasvirta et al. 2009].
Para limitar la cantidad de resultados obtenidos y mejorar la calidad de éstos,
existen propuestas para obtener sólo los $k$ resultados mejor evaluados.
Esto trae múltiples beneficios para el sistema. El procesar sólo un subconjunto
de los objetos durante una búsqueda mejora el rendimiento al reducir la
cantidad de información que se debe manejar, otorgando una menor latencia para las búsquedas.

Dentro de los algoritmos propuestos para lograr esto, se encuentran:



%
%— Top k results requesting [Cuenca-Acuna et al. 2003]
%A simple way to optimise the system is to only request the top results. Approaches
%that use local indices always apply a variable form of limited result requesting im-
%plicitly by bounding the number of hops made when flooding or by performing a
%random walk that terminates. However, that number can also be explicitly set to
%a constant by the requester as is done for the globally replicated index used by
%[Cuenca-Acuna et al. 2003]. They first obtain a list of k search results and keep con-
%tacting nodes as long as the chance of them contributing to this top k remains high.
%The top results stabilise after a few rounds.

\paragraph{Top k results requesting~\cite{cuenca2003planetp}} % [Cuenca-Acuna et al. 2003]
Una forma simple de optimizar un sistema es haciendo que sólo se pidan los
resultados ``top''. Sistemas que usan índices locales siempre aplican una forma
variable para limitar los resultados. Por ejemplo, al limitar la cantidad de saltos a realizar
cuando se busca por \textit{flooding} o a través de un \textit{random walk}.
Sin embargo, ese número también puede ser explícitamente puesto en una
constante por quien realiza la consulta, como lo hecho por~\citealt{cuenca2003planetp} % [Cuenca-Acuna et al. 2003]
para índices globalmente replicados. Ellos primero obtienen una lista de $k$
resultados de búsqueda y continúan contactando nodos mientras que se mantenga una
probabilidad alta de que éstos aporten a la lista de $k$ resultados ``top''. La lista de
los ``top'' $k$ resultados se estabilizan después de unas cuantas rondas.

%
%— Top k query processing [Suel et al. 2003; Balke et al. 2005; Michel et al. 2005a;
%Zhang and Suel 2005]
%This approach has its roots in the database community, particularly in the work of
%[Fagin et al. 2001]. Several variations exist, all with the same basic idea: we can
%determine the top k documents given several input lists without having to examine
%these lists completely and while not adversely affecting performance. This is often
%used in cases where a distributed global index is used and posting lists have to be
%intersected. The threshold algorithm is the most popular [Michel et al. 2005a; Suel
%et al. 2003]. This algorithm maintains two data structures: a queue with peers to
%contact for obtaining search results and a list with the current top k results. Peers in
%the queue are processed one by one, each returning a limited set of k search results
%of the form (document, score) sorted by score in descending order. For a distributed
%global index these are the top items in the posting list for a particular term. The
%algorithm tracks two scores for each unique document: worst and best. The worst
%score is the sum of the scores for a document d found in all result lists in which
%d appeared. The best score is the worst score plus the lowest score (of some other
%document) encountered in the result lists in which d did not appear. Since all the
%result lists are truncated, this last score forms an upper bound of the best possible
%score that would be achievable for document d. The current top k is formed by the
%highest scoring documents seen so far based on their worst score. If the best score
%of a document is lower than the threshold, which is the worst score of the docu-
%ment at position k in the current top k results, it need not be considered for the
%top k. The algorithm thus bases the final intersection on only the top k results from
%each peer, which provably yields performance equivalent to ‘sequentially’ intersect-
%ing the entire lists. This thus saves both bandwidth and computational costs without
%negatively affecting result quality. A drawback is that looking up document scores
%requires random access to the result lists [Suel et al. 2003]. [Zhang and Suel 2005]
%later investigated the combination of top k query processing with several optimi-
%sation techniques. They draw the important conclusion that different optimisations
%may be appropriate for queries of different lengths. [Balke et al. 2005] show that top
%k query processing can also be effective in peer-to-peer networks with aggregated
%local indices.

\paragraph{Top k query processing~\cite{suel2003odissea, balke2005progressive, michel2005klee, zhang2005efficient}} % [Suel et al. 2003; Balke et al. 2005; Michel et al. 2005a; Zhang and Suel 2005]
Este método tiene sus inicios en la comunidad en torno a las bases de datos,
particularmente en el trabajo de~\citealt{fagin2001optimal} %[Fagin et al. 2001].
Múltiples variaciones existen, todas con la misma idea básica: es posible
determinar los ``top'' $k$ documentos dadas muchas listas de resultados sin tener
que examinarlas por completo y sin tener que afectar de forma negativa al
rendimiento de la búsqueda. Esto es comúnmente usado en presencia de índices
globales distribuidos en casos donde listas con publicaciones deben ser intersectadas.
Dentro de los algoritmos propuestos, el \textit{threshold algorithm} es el más
popular~\citealt{michel2005klee, suel2003odissea}. %[Michel et al. 2005a; Suelet al. 2003]
Este algoritmo mantiene dos estructuras para la información: una cola con
\textit{peers} para contactar y obtener resultados de búsquedas y una lista con
los actuales ``top'' $k$ resultados. Mientras los \textit{Peers} en la cola van
siendo procesados, cada uno va retornando un set limitado de $k$ resultados de búsqueda con
la forma (documento, puntaje), ordenados en orden descendiente según su puntaje.
Para un índice global distribuido estos son los objetos ``top'' de una lista de
publicaciones para un término en particular. El algoritmo mantiene dos puntajes
para cada documento único: el ``mejor'' y el ``peor''. El ``peor'' puntaje es la suma de
los puntajes de un documento $d$ encontrado en todas las listas de resultados en
donde $d$ aparece. El ``mejor'' puntaje es el peor puntaje más el menor puntaje
(de algún otro documento) encontrado en la lista de resultados en donde $d$
\textit{no} aparece. Como cada lista de resultados es cortada, este último
puntaje forma una cota superior del mejor puntaje posible que puede ser obtenido
por el documento $d$. El $k$ `top'' actual  es formado por los documentos con más altos
puntajes vistos según el ``peor'' puntaje de cada uno. Si el ``mejor'' puntaje
de un documento es menor que el umbral definido, siendo este umbral definido por el ``peor'' puntaje del
documento en posición $k$ en la lista de $k$ ``top'' resultados, el documento no
necesita ser considerado para la lista de ``top'' $k$ elementos.

El algoritmo se basa en la intersección final de sólo los ``top'' $k$ resultados
de cada \textit{peer}, lo cual probablemente entrega un rendimiento equivalente
a intersectar secuencialmente las listas completas. De esta forma se ahorra
ancho de banda y costos de cómputo sin afectar negativamente la calidad de los
resultados.

Un inconveniente de esto es que al buscar por puntajes de documentos requiere
de accesos aleatorios a las listas de resultados~\citealt{suel2003odissea}. %[Suel et al. 2003].
%[Zhang and Suel 2005]
\citealt{zhang2005efficient}~
investigó más tarde la combinación de de búsquedas con procesamiento de los
``top'' $k$ resultados junto con diferentes optimizaciones, indicando
conclusiones importantes sobre que optimizaciones pueden ser apropiadas para
consultas de diferentes largos.
%[Balke et al. 2005] 
\citealt{balke2005progressive}~
muestra que el procesamiento de sólo los ``top'' $k$ resultados puede también
ser efectiva en redes P2P con índices locales agregados.

%— Top k result storing [Tang et al. 2002; Tang and Dwarkadas 2004; Skobeltsyn and
%One step further is only storing the top k results for a query, or term, in the index.
%[Skobeltsyn and Aberer 2006] take this approach as a means to further reduce traffic
%consumption. Related to this is the approach of [Tang and Dwarkadas 2004] that
%store postings only for the top terms in a document. They state that while indexing
%only these top terms might degrade the quality of search results, it likely does not
%matter since such documents would not rank high for queries for the other non-top
%terms they contain anyway.

\paragraph{Top k result storing~\cite{tang2004hybrid, skobeltsyn2006distributed, skobeltsyn2009query, skobeltsyn2007query}} % [Tang et al. 2002; Tang and Dwarkadas 2004; Skobeltsyn and %Aberer 2006; Skobeltsyn et al. 2007; Skobeltsyn et al. 2009]
Una forma de adelantarse un paso adelante es sólo almacenar los ``top'' $k$
resultados para una consulta (o término) dentro del índice.
\citealt{skobeltsyn2006distributed}~%[Skobeltsyn and Aberer 2006]
utiliza este método como medio para reducir aún más el tráfico consumido.
Otro método similar es abordado por~\citealt{tang2004hybrid} %[Tang and Dwarkadas 2004] 
el cual almacena publicaciones sólo de los términos ``top'' de un documento.
Ellos suponen que mientras la indexación de sólo los términos ``top'' puede
degradar la calidad de los resultados de búsqueda, es más probable que esto no
importe, ya que tales documentos de todas formas no quedarían dentro del ranking ``top''
dentro de las consultas realizadas por los otros términos no ``top'' que contienen.
