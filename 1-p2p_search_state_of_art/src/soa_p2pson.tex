\input{src/intros/4-soap2pson}

\section{Implementaciones de redes sociales Peer-to-Peer}
\label{sec:implementaciones}

En lo referente a sistemas enfocados principalmente a la mantención de redes
sociales, no son muchos muchos los sistemas que se han implementado. A
continuación veremos las características de redes sociales P2P u otros
servicios con funcionalidades similares que actualmente han salido o se encuentran en desarrollo.

\subsection{PeerSoN}
PeerSoN~\footnote{http://www.peerson.net}~\cite{buchegger:peerson}~\cite{buchegger:2009:pps:1578002.1578010}  es un prototipo de estructura para redes sociales
Peer-to-Peer que busca asegurar la privacidad de los usuarios y potenciar la
comunicación directa entre cada usuario.

Su arquitectura está basada en 2 capas, una para la búsqueda y otra para el almacenamiento de la
información. Considera la asignación de identificadores únicos para cada
usuario, procedimientos de entrada, envío y obtención de archivos y el manejo
de mensajes asincrónicos.

\paragraph{Almacenamiento}
La capa de almacenamiento consiste en los peers en sí, los cuales, una vez
encontrados por la capa de búsqueda, pasan a juntar y enviar la información
directamente entre sí, distribuyendo réplicas de éstos
en la red para aumentar la disponibilidad de los datos. 

\paragraph{Seguridad}
PeerSoN~\cite{buchegger:peerson} basa su seguridad en encriptación simétrica y asimétrica. La
información primero es encriptada usando una llave simétrica, y luego esta
llave es encriptada con las llaves públicas de los recipientes. Los
identificadores de los usuarios son encriptados junto con las llaves
simétricas, siendo todo enviado con la información encriptada.

\paragraph{Búsqueda}
 La capa de búsqueda almacena la información requerida para
encontrar a los usuarios y a la información que éstos tienen, utilizando un
DHT\ref{sec:p2p_estructured} para ello (OpenDHT~\cite{Rhea:2005:OPD:1080091.1080102}).

