%% TODO:
%% La introducción está mal organizada. Se trata de explicar lo siguiente:
%% - Cúal es el problema?
    En un sistema centralizado, todos los algoritmos de búsqueda son completos,
    debido a que la información es indexada bajo bases de datos en donde la
    realización de consultas complejas no son un problema. Como poseen el
    conociento completo de los índices y documentos almacenados en el sistema,
    se puede sin grandes dificultades
    ordenar los resultados obtenidos y presentarlos de una forma adecuada. Ahora, el
    almacenamiento debe ser manejado por el sistema, al igual que
    los costos de ancho banda requerido para contestar todas las consultas
    requeridas, incurriendo a costos proporcionales a la cantidad de recursos
    utilizados. 

    A diferencia de los sistemas centralizados, el tema es especialmente complejo en redes basadas en DHT, principalmente
    porque las tablas de hash uniforme destruyen el todo orden para la
    localización de la información, quedando esta distribuida en cada nodo de
    la red sin existir una relación entre ellos y la información que almacenan.
    Para poder realizar búsquedas de éste estilo, es necesaria la
    implementación de índices sobre los datos almacenados en ellas.

    Por otro lado, como la información se encuentra distribuida entre los
    diferentes nodos que conforman la red, no es fácil implementar un sistema
    que ordene los resultados de cada búsqueda y entregue, por ejemplo, los 20
    más importantes. 
    Normalmente, los sistemas de
    indexación que se proponen para redes P2P no tratan el problema del manejo de
    miles de resultados, dejando ese trabajo al nodo que realizó la consulta.

    Otro punto importante es que sobre la calidad de los resultados que deben
    obtenerse. En el caso de una consulta con miles de resultados, el sistema
    debe saber discriminar cuales son los más relevantes para el usuario, ordenados
    según criterios implícitos o explícitos para ello.

    Además de las funcionalidades básicas esperadas para este tipo de
    búsquedas, se deben considerar que su implementación sea posible según
    las capacidades del sistema. Por ello, debe
    considerarse el almacenamiento requerido para la mantención de los índices
    y los gastos de red requeridos para la realización de la búsqueda
    en sí.

%% - Porque es un problema importante o actual?

    La búsquedas de datos dentro de las redes sociales  son de vital importancia para la formación de la
    red. A través de búsquedas es como los usuarios
    pueden encontrar a nuevos contactos dentro de la red y encontrar contenido
    específico según las diferentes necesidades de cada usuario.  Para ello, es
    necesario contar con un sistema que permita realizar búsquedas complejas.
    Esto considera poder buscar elementos que se encuentren dentro de un cierto
    rando de valores y/o que calcen con cierto conjunto de palabras claves.


%% - Cuál es el objetivo de la memoria?

%% -Al final hay que presentar la estructura de la memoria con los capítulos
%% posteriores.

A continuación, se desarrollará el trabajo con la siguiente estructura:

En la Sección~\ref{sec:soa_indexing}
.......
En la Sección~\ref{sec:soa_ranking} 
......
 concluyendo y viendo el trabajo futuro
en la Sección~\ref{sec:conclusiones}.

