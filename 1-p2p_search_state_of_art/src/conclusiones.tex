%# Intro
%#SOA SN
%# SOA P2P

Las soluciones utilizadas por los sitios de redes sociales
centralizados no es compatible con el esquema P2P, debido a las grandes
diferencias de como se distribuye la información y se mantiene la arquitectura
del sistema.

Los sistemas P2P más conocidos son capaces de proveer una capa de red que no
dependa de servidores centralizados, pero no suplen por si solos las
funcionalidades necesarias para cubrir los requerimientos
especializados de una red social P2P. Subsistemas con servicios especializados
son requeridos para el manejo del almacenamiento, búsqueda de la información,
manejo de permisos de usuario, seguridad contra ataques, etc., agregando una
complejidad adicional al diseño de la arquitectura del sistema, junto con
nuevos costos para la red que todavía no han sido investigados a cabalidad.

%que surgen de la falta de la precondición de confianza entre todas las entidades que mantienen el sistema.

%# SOA P2PSN
Analizando las diferentes propuestas de redes sociales P2P actuales, nos
encontramos que no se encuentran desarrolladas con la madurez  necesaria para hacer
frente a todos los problemas identificados. 

%#peerson
En PeerSoN, el sistema de encriptación utilizado es muy básico, requiriendo para poder compartir
un dato a grupo una cantidad de procesamiento proporcional a la
cantidad de usuarios. Además, en caso de modificación del grupo, se requeriría
la reencriptación de todos los archivos compartidos entre sí, con todos los
costos que esto significaría. En el caso del sistema de búsqueda, al basarse en
un DHT sin mecanismos adicionales, no permitiría búsquedas complejas o de más
de una dimensión, dificultando el proceso de establecimiento conexiones entre
la red.

%# Safebook
Safebook requiere de diferentes niveles de confianza entre un usuario y sus
conexiones dentro de la red social, y almacena réplicas de los datos personales
entre los amigos más cercanos, pudiendo haber problemas de disponibilidad de
los datos en los momentos en que éstos no se encuentren en linea. Para la identificación requiere de un servicio 
externo que pueda ser confiado para la asignación de ID únicas. La transmisión
de mensajes a través de la red se realiza anonimizando al que envía y recibe el
mensaje, aumentando el costo y tiempo de envío de cada dato por cada salto que
de entre nodo y nodo.

%#Diaspora
%Diaspora, no siendo un servicio completamente descentralizado, posee el
%problema de la mantención de los servidores públicos en donde se mantiene la
%información, siendo pocos los que hasta el día de hoy se mantienen en linea, y
%surgen dudas sobre el destino de los datos depositados en ellos. En la realidad
%son muy pocos los que tienen el conocimiento y los recursos para levantar su
%propio \textit{pod} para mantenerse en la red social. Estas son algunas de las
%razones por la cual no vemos un gran crecimiento en la cantidad de usuarios en
%la red social, y a menos que la situación cambie, se ve difícil una 
%implementación exitosa basada en su arquitectura.


%para la generación de X llaves unalmacenamiento
%requerido por dato escala proporcionamente a la cantidad de usuarios con el
%cual el dato se quiere compartir. permite compartir información a grupos
%de usuarios

%%%%%%%%%% PROBLEMATICAS MAS IMPORTANTES


%%Dentro de los problemas que se encuentran sin una solución definitiva y
%%abiertos a una mayor investigación, podemos nombrar:
Analizando las problemáticas encontradas, podemos concluir que quedan varios
desafíos complejos por resolver, entre los cuales se encuentran:
%# Problems?

\begin{enumerate}
    %%\item Límite del espacio de almacenamiento que los usuarios están dispuestos a compartir.
    \item Lograr incentivar a los usuarios a que compartan recursos con el sistema.

    Para lograr que el sistema se mantenga auto-sustentable y escale de forma
    adecuada frente a un crecimiento de la información almacenada, pero no de nodos
    dentro de la misma, sumado a que cada usuario tiene la capacidad de elegir
    que recursos comparte con el sistema, podría llegar a producirse una la falta de recursos de
    almacenamiento en la red si se enfrentan una cantidad considerable de
    usuarios egoístas en el sistema. Esto es un problema complejo de resolver
    debido a que el sistema no cuenta con un control central que pueda penalizar el
    comportamiento de un usuario, dejando a los mismos usuarios la decisión de
    penalizar o no este tipo de comportamientos. 

    %%\item Falta de mecanismos eficientes para la realización de búsquedas complejas.
    \item Elaboración de un subsistema de búsqueda de datos almacenados en la
    red social que sea eficiente tanto como en sus búsquedas como en la mantención
    de los índices.

    Como ya fue mencionado anteriormente, lograr ambas propiedades en redes
    basadas en un DHT es complejo debido a la pérdida de localidad de la
    información, y dependiendo de no sólo la cantidad de dimensiones que se
    requieran consultar, si no también del orden que éstas se consulten, el costo
    de búsqueda de la información crece en peores casos de forma exponencial.
    Considerando la inmensa cantidad de datos a indexar y la frecuencia con que
    son modificados, la mantención del índice de datos no es una tarea trivial,
    y requiere de algoritmos inteligentes que minimicen la cantidad de recursos
    utilizados para ello. El problema que esto conlleva, es que es difícil
    tener búsquedas optimizadas manteniendo un índice simple de los archivos,
    por lo que muchas veces el minimizar el costo de mantención aumenta el costo y
    eficiencia de las búsquedas en el sistema.
    %%\item No existen una precondición de confianza entre los nodos.
    %%\item Falta de un sistema de control que asegure la integridad de las operaciones en la red
    %\item Elaboración de un sistema de control que permita asegurar la integridad de las operaciones en la red
    %%\item No se cuenta con sistemas de control para asegurar la privacidad de los datos compartidos en el sistema.
    \item Elaboración de un subsistema de control para asegurar la privacidad de los datos compartidos en el sistema.
    

    Como vimos anteriormente, las soluciones normalmente utilizadas para asegurar
    los datos almacenados dentro de una aplicación no son aplicables en un ambiente
    distribuido P2P. Es por ello que normalmente se utilizan sistemas basados en la
    encriptación para poder mantener segura la información en estos sistemas.
    Ahora, viendo el caso específico de redes sociales, los esquemas de
    encriptación tradicionales no cuentan con todas las funcionalidades deseables
    para la mantención del sistema. 

    Las soluciones tradicionales de utilizar encriptación y llaves compartidas
    entre las partes involucradas no cumplen actualmente con todas las
    funcionalidades requeridas para tratar de forma eficaz y permitir al mismo
    tiempo funcionalidades requeridas en grupos conformados por varios usuarios de
    la red social. El tema es complejo debido a que no existe un método de
    encriptación que su costo constante frente a diversa cantidad de usuarios y
    que permita el manejo fino de permisos para cada archivo compartido.
    
    % esto se saco, pero podría agregarse una seción como estado del arte
   % %encriptacion
   % Analizando los sistemas de encriptación disponibles en la actualidad, la combinación de criptografía simétrica y asimétrica no tiene
   % la suficiente eficiencia y funcionalidad para una red social P2P. CP-ABE es
   % inferior a BE ya que ninguno de los CP-ABE logran soportar múltiples tópicos o
   % atributos, acceso oculto a las estructuras de acceso y bajo costo de
   % almacenamiento y costo computacional al mismo tiempo. Es por eso, que la
   % propuesta de usar BE para la red social P2P por sobre las demás es prometedora,
   % ya que no posee las características negativas de las demás, a pesar de no
   % soporta la funcionalidad de encriptar para un grupo del cual uno no pertenece y
   % para ``amigos de amigos''. 
   % %\begin{itemize}
   %     %\item .
    %\end{itemize}
    %%Además, si consideramos el problema de manejo de 
    %%%\item Manejo de los diferentes privilegios de acceso y lectura para usuarios y grupos de usuarios.
    %%%\item Mantención de datos actualizados en las diferentes réplicas frente a frecuentes salidas y entradas de usuarios
    %%\item Falta de un sistema que permita la interacción entre agentes externos y la red social
    \item Elaboración de un subsistema que permita interactuar con agentes externos a la red social.

    El tema es complejo debido a que requiere de la existencia de una interfaz
    estable que pueda hacer accesible la red P2P sin ser un nodo dentro de
    ella desde el internet. Debido al alto dinamismo de estas redes y la
    descentralización de la misma, no es posible utilizar un sólo nodo como
    puerta de acceso a la red social. Y aún así, aunque se pudiera utilizar un
    nodo especializado para ello o un conjunto de nodos para que agentes
    externos accedan a la red social, se crearía un cuello de botella para el
    acceso a ella con un costo que los usuarios no querrán pagar.
    Además, se necesita poder monitorear y controlar las acciones que los agentes
    externos realizan en la red social, de tal forma de poder bloquear ataques
    y prevenir abusos de éstos hacia los usuarios de la red P2P.

    % esto se saco, pero podría agregarse una seción como estado del arte
    %%integración con otras apps
    %%\paragraph{Integración con otras aplicaciones}
    %La integración con otras aplicaciones también requiere un mayor desarrollo para
    %el establecimiento de confianza entre los usuarios y los entes que las
    %mantienen. De la misma forma que OAuth requiere de una entidad confiable para
    %la autorización de aplicaciones dentro de un esquema centralizado, en el
    %esquema distribuido cada usuario debe hacerse cargo de la autorización, siendo
    %posible bajo DIBBE, pero abriendo la posibilidad de nuevas vulnerabilidades a
    %ataques que no han sido estudiadas todavía.

\end{enumerate}


%%%%%%%%%%%%%%%%% TRABAJO FUTURO?

%# Solutions?

%identificacion
%Los sistemas centralizados (ejemplos?)
%Los sistemas distribuidos (ejemplos?)...
%# Solutions?

%resistencia ataques
%Las defensas utilizadas para defenderse diferentes ataques en un servidor
%combencional también es muy diferente al de una implementación P2P. Mientras
%que las primeras son elaboradas en un ambien

Otros temas que no fueron vistos anteriormente debido a que no se encuentran
asociados directamente con las funcionalidades de las redes sociales, sino que
nacen de vulnerabilidades y comportamientos no esperados de parte de agentes
maliciosos, requeriría de un análisis especializado una vez que se hallan
definido el funcionamiento de los diferentes subsistemas que mantendrían las
funcionalidades del sistema. De todas formas, vale la pena mencionar que para
el manejo y control de ataques y nodos maliciosos dentro de la red social,
existen propuestas para utilizar las relaciones de confianza y colaboración
entre los usuarios, pero no existen estudios concretos para la solución de éste
tipo de problemáticas.


%# between SOA P2P and SOA P2PSN
Finalmente, podemos concluir que utilizando como base a los sistemas P2P
estructurados, estos sistemas podrían extenderse para soportar funcionalidades
más avanzadas, sin tener que sacrificar las características deseables que
poseen (escalabilidad, robustez e independencia de otros sistemas), pero
primero se necesitará resolver las problemáticas existentes. Mientras algunas de
ellas pueden ser resueltas poniendo en práctica ciertas investigaciones, otras
requerirán de más trabajo para poder llegar a ser resueltas.
Aún así, existe la esperanza de que investigaciones en proceso encuentren soluciones que puedan
enfrentar parte de estos problemas y continuar con el desarrollo de una red
social P2P completamente distribuida.
%, y probablemente podramos encontrar formas
%de implementar una solución deseada si es que se sigue manteniendo con una
%tendencia positiva.

%%%## almacenamiento
%%%\paragraph{Almacenamiento}
%%\begin{enumerate}
%% \item Control de usuarios maliciosos o \textit{free riders} que se niegan compartir espacio de almacenamiento con otros usuarios.
%% \item Mejorar la disponibilidad de datos dentro de la red, especialmente los
%%    relacionados con el perfil y datos privados de los usuarios para el
%%    establecimiento de relaciones entre ellos.\\
%%    %### disponibilidad
%%    Considerando las diferentes estrategias para aumentar la disponibilidad de los
%%    datos en la red, éstas requerirán ajustarse para alcanzar un equilibrio entre
%%    los costos y beneficios que ofrezca. Es importante no aumentar innecesariamente
%%    la cantidad de almacenamiento necesario para mantener la disponibilidad alta de
%%    los datos, teniendo en cuenta que el sistema no debe presentar riesgos para la
%%    privacidad de los datos de las personas.
%% \item Mejorar la velocidad de carga y distribución de los datos, junto con el
%%    costo de ancho de banda asociado a ello.\\
%%    %### banda ancha
%%    Son pocas las soluciones que tratan el problema de los costos de ancho de
%%    banda para la distribución de los archivos por la red, siendo un problema que
%%    variará de intensidad dependiendo mayoritariamente de la arquitectura implementada por el
%%    sistema, especialmente por las políticas de transferencia de datos a través
%%    de los nodos que se encuentren bajo NAT y que requieran de nodos intermediarios
%%    para ser alcanzados.
%%%## busquedas
%%%\paragraph{Búsqueda}
%%%## seguridad
%%%\paragraph{Seguridad}
%% \item Sistema de identificación de usuarios que permita manejar ataques del tipo
%%    robo de identidad y clonación de perfiles.
%% \item Disminución de los costos de encriptación y desencriptación del sistema
%%    de encriptación utilizado, idealmente llegando a un costo constante, sin
%%    depender de la cantidad de usuarios para los cuales se encripte el dato.
%% %\item Que la adición y remoción de nuevos usuarios de un grupo no dependa de la cantidad de objetos compartidos en éste.
%% %\item Tamaño del encabezado de encriptación no debe depender de la cantidad destinatarios, de tal forma que el peso de los archivos sea escalable.
%% \item Implementación de operaciones de unión e intersección entre usuarios pertenecientes a diferentes grupos.
%% % no estoy tan seguro:
%% %\item Que sólo los usuarios autorizados puedan acceder a la lista de acceso (de la cabecera de encriptación). 
%%     %\item Habilidad de encriptar para grupos de los cuales uno no es miembro.
%%     %\item Habilidad de encriptar para ``amigos de amigos''.
%%     %\item Habilidad que permita no revelar las estructuras de acceso en la cabecera de los objetos encriptados.
%% \item Sistema de recuperación de llaves de acceso y otro tipo de información
%%    relevante para la identificación del usuario con la red social.
%%\end{enumerate}


