%%%%\input{src/intros/soa_indexing}

\subsection{Almacenamiento de los índices.}
\label{sec:index_storage}
    

    A la hora de indexar los documentos, dependiendo de la forma en
    que ésto se realice serán los costos de red incurridos en la busqueda.
    Si consideramos que los índices serán realizado por documento, repartiendo
    los documentos entre los usuarios del sistema, cada uno mantendría un índice
    invertido de forma local. Por ello, cada consulta debe ser distribuida
    hacia todos los nodos del sistema generando una inundación innecesaria de paquetes en el sistema.
    
    Si en cambio se particionan los índices según las palabras claves que
    aparecen en cada documento, un DHT puede ser usado para mapear una palabra
    con el nodo responsable de éste.
    El problema de ésto es que si no se ordena y limita los resultados de una búsqueda, una consulta de millones de resultados conllevaría a un enorme
    costo de bando de ancha de la red. Además, una consulta por varias palabras
    claves requeriría del envío sobre la red de los diferentes índices de parte
    de cada nodo responsable, realizando posteriormente el cruze  de la
    información resultante. En el caso de de pocos resultados, esto no es un
    gran problema, pero cruzar índices de billones o millones de resultados entre
    sí es complejo.

    Si vemos las restricciones de almacenamiento de los índices en un DHT y lo
    aplicamos al caso de Twitter, tenemos que, con
    340 millones de tweets por día, en un año tenemos alrededor de 124 Billones de
    documentos con un largo de hasta 140 carácteres cada uno. Si consideramos
    un promedio de 15 palabras por tweet, un índice invertido de estos
    contendría $124 * 10^9 * 15 = 1860 * 10^9$ identificadores, una cifra
    considerable si pensamos que es similar en orden de magnitud al número de
    índices manejados por google en el año 2003.
    Ahora, en un DHT, un identificador, que es una llave con la cual puede
    obtenerse el documento, es normalmente un hash de 20 bytes del contenido del
    documento.
    Si consideramos ese peso por cada índice a almancenar en el sistema,
    tenemos en total alrededor de $4 * 10^{13}$  bytes (40 TB). Ahora, gracias
    a la escalabilidad de este tipo de sistema, los índices pueden ser repartidos
    entre todos los usuarios de la red. Considerando la cantidad de usuarios de
    Twitter (140 millones de usuarios), tenemos que cada uno de ellos sólo
    debería dedicar para la indexación de archivos  alrededor de 30 MB, lo cual
    es un costo ínfimo considerando las capacidades de almacenamiento de hoy en
    día.


\subsection{Problemas encontrados}

\begin{enumerate}
    \item Falta de mecanismos eficientes para la realización de búsquedas complejas.
\end{enumerate}

  \subsubsection{Falta de mecanismos eficientes para la realización de búsquedas complejas}

     Existen un gran rango de sistemas de indexación propuestos, los cuales pueden clasificarse en
    dos tipos: las \textit{sobre DHTs} y las \textit{dependientes del sistema de red}.
    Las primeras permiten una fácil implementación sobre cualquier DHT existente,
    haciendo uso de las interfaces genéricas que éstos proveen, sin hacer modificación
    alguna al sistema de red por sobre el cual está montado. Por el contrario, las
    otras soluciones están estrechamente ligadas con el sistema de red en el cual
    fueron implementados, normalmente haciendo uso de modificaciones a los otros
    sistemas con los que interactúan de forma de hacer posible su funcionamiento
    y/o optimizar el rendimiento de sus funcionalidades.
    Dos puntos son críticos para el rendimiento de un sistema de
    búsquedas sobre un DHT, la \textit{eficiencia de la consulta} y el
    \textit{costo de mantención del índice}. En redes sociales, cada entrada y
    salida de un nodo usualmente resultan en operaciones de inserción o borrado
    en el sistema, a la vez que la cantidad de entradas y salidas dentro del
    sistema ser tan alto como la cantidad de consultas que se realizan. Esto
    significa que cada actualización en el sistema requiere de actualizaciones de
    los índices de los datos, haciendo que el costo de la mantención del índice un
    problema no menor. Por ello, se necesita que el sistema de indexación de datos
    sea eficiente tanto con los tiempo de búsqueda como con el costo de mantención
    del mismo. Sin embargo, muchos de los sistemas existentes se enfocan en bajos
    tiempos de búsqueda a cambio de un mayor costo de mantención, manteniendo un costo prohibitivamente alto
    que vuelve poco escalable al sistema P2P.  % dentro de estas soluciones, podemos mencionar a:

    Dentro de los métodos de indexación que permitan realizar
    búsquedas complejas (LIGHT~\cite{journals_tkde_TangZX10},
    DRing~\cite{hidalgo2011dring}), también existen diversas propuestas para mejorar otros tipos de
    búsquedas dentro de la red (~\cite{5345647}, ~\cite{Ng02peerclustering},
    ~\cite{conf:infocom:SripanidkulchaiMZ03}).
    %%Otros sistemas que proponen eficiencia por ambos lados, como 
    
    %\begin{itemize}
        %\item LIGHT
        %\item ...
        %\item 1- DHT de búsquedas (LIGHT) + CORPS
    %\end{itemize}
    
    
    
    %% TODO: Sacar bien el problema del paper de LIGHT
    %% El problema está en que todavía los métodos de indexación siguen siendo muy
    %% costosos para su implementación, creciendo exponencialmente su 
    
    %  En~\cite{5345647} se presenta un algoritmo de búsqueda Peer-to-Peer que utiliza el
    %conocimiento de los intereses de los usuarios para agruparlos en grupos para
    %encontrar contenidos de características similares. El algoritmo se ajusta
    %dinámicamente a las diferentes entradas y salidas de los usuarios, dividiendo o
    %juntando grupos a medida que los interés vallan cambiando. En comparación con
    %otros algoritmos de búsqueda como el
    %firework method~\cite{Ng02peerclustering}
    %y interest-based shortcut method~\cite{conf:infocom:SripanidkulchaiMZ03}
    % tiene una mayor robustez frente a una
    %red con altas tasas de entrada y salida de usuarios.
    
    
%    \subsubsection{Sistemas de subscripción/publicación}
%    
%    Una alternativa para la obtención del contenido de los usuarios entre sí sin
%    tener que realizar una búsqueda específica por el nuevo contenido compartido,
%    es la implementación de sistemas de subscripción, los cuales permites abstraer
%    al que publica de los problemas de distribución del contenido hacia los
%    subscriptores (o \textit{seguidores}) del mismo. Existen dos tipos de sistemas
%    de subscripción; los basados en tópicos y los basados en el contenido. Los
%    primeros, los generadores de contenido organizan la información publicada en tópicos a los cuales los otros
%    usuarios se suscriben. En cambio, los basados en tópicos publican el contenido
%    abiertamente a todos los suscriptores, siendo ellos los que pasan a filtrar el
%    contenido, según ciertos tópicos definidos por cada uno de los receptores de la
%    información.
%    
%    % TODO:
%    % mencionar porque estos sistemas son importantes (sistema de notificaciones y
%    % seguimiento de actualizaciones entre los usuarios)
%    % y los problemas involucrados en una red social p2p
%    
%    %%\subsection{No existen una precondición de confianza entre los nodos}
    


\subsection{Almacenamiento de los índices.}
\label{sec:index_storage}
    

    A la hora de indexar los documentos, dependiendo de la forma en
    que ésto se realice serán los costos de red incurridos en la busqueda.
    Si consideramos que los índices serán realizado por documento, repartiendo
    los documentos entre los usuarios del sistema, cada uno mantendría un índice
    invertido de forma local. Por ello, cada consulta debe ser distribuida
    hacia todos los nodos del sistema generando una inundación innecesaria de paquetes en el sistema.
    
    Si en cambio se particionan los índices según las palabras claves que
    aparecen en cada documento, un DHT puede ser usado para mapear una palabra
    con el nodo responsable de éste.
    El problema de ésto es que si no se ordena y limita los resultados de una búsqueda, una consulta de millones de resultados conllevaría a un enorme
    costo de bando de ancha de la red. Además, una consulta por varias palabras
    claves requeriría del envío sobre la red de los diferentes índices de parte
    de cada nodo responsable, realizando posteriormente el cruze  de la
    información resultante. En el caso de de pocos resultados, esto no es un
    gran problema, pero cruzar índices de billones o millones de resultados entre
    sí es complejo.

    Si vemos las restricciones de almacenamiento de los índices en un DHT y lo
    aplicamos al caso de Twitter, tenemos que, con
    340 millones de tweets por día, en un año tenemos alrededor de 124 Billones de
    documentos con un largo de hasta 140 carácteres cada uno. Si consideramos
    un promedio de 15 palabras por tweet, un índice invertido de estos
    contendría $124 * 10^9 * 15 = 1860 * 10^9$ identificadores, una cifra
    considerable si pensamos que es similar en orden de magnitud al número de
    índices manejados por google en el año 2003.
    Ahora, en un DHT, un identificador, que es una llave con la cual puede
    obtenerse el documento, es normalmente un hash de 20 bytes del contenido del
    documento.
    Si consideramos ese peso por cada índice a almancenar en el sistema,
    tenemos en total alrededor de $4 * 10^{13}$  bytes (40 TB). Ahora, gracias
    a la escalabilidad de este tipo de sistema, los índices pueden ser repartidos
    entre todos los usuarios de la red. Considerando la cantidad de usuarios de
    Twitter (140 millones de usuarios), tenemos que cada uno de ellos sólo
    debería dedicar para la indexación de archivos  alrededor de 30 MB, lo cual
    es un costo ínfimo considerando las capacidades de almacenamiento de hoy en
    día.


\subsection{Problemas encontrados}

\begin{enumerate}
    \item Falta de mecanismos eficientes para la realización de búsquedas complejas.
\end{enumerate}

  \subsubsection{Falta de mecanismos eficientes para la realización de búsquedas complejas}

     Existen un gran rango de sistemas de indexación propuestos, los cuales pueden clasificarse en
    dos tipos: las \textit{sobre DHTs} y las \textit{dependientes del sistema de red}.
    Las primeras permiten una fácil implementación sobre cualquier DHT existente,
    haciendo uso de las interfaces genéricas que éstos proveen, sin hacer modificación
    alguna al sistema de red por sobre el cual está montado. Por el contrario, las
    otras soluciones están estrechamente ligadas con el sistema de red en el cual
    fueron implementados, normalmente haciendo uso de modificaciones a los otros
    sistemas con los que interactúan de forma de hacer posible su funcionamiento
    y/o optimizar el rendimiento de sus funcionalidades.
    Dos puntos son críticos para el rendimiento de un sistema de
    búsquedas sobre un DHT, la \textit{eficiencia de la consulta} y el
    \textit{costo de mantención del índice}. En redes sociales, cada entrada y
    salida de un nodo usualmente resultan en operaciones de inserción o borrado
    en el sistema, a la vez que la cantidad de entradas y salidas dentro del
    sistema ser tan alto como la cantidad de consultas que se realizan. Esto
    significa que cada actualización en el sistema requiere de actualizaciones de
    los índices de los datos, haciendo que el costo de la mantención del índice un
    problema no menor. Por ello, se necesita que el sistema de indexación de datos
    sea eficiente tanto con los tiempo de búsqueda como con el costo de mantención
    del mismo. Sin embargo, muchos de los sistemas existentes se enfocan en bajos
    tiempos de búsqueda a cambio de un mayor costo de mantención, manteniendo un costo prohibitivamente alto
    que vuelve poco escalable al sistema P2P.  % dentro de estas soluciones, podemos mencionar a:

    Dentro de los métodos de indexación que permitan realizar
    búsquedas complejas (LIGHT~\cite{journals_tkde_TangZX10},
    DRing~\cite{hidalgo2011dring}), también existen diversas propuestas para mejorar otros tipos de
    búsquedas dentro de la red (~\cite{5345647}, ~\cite{Ng02peerclustering},
    ~\cite{conf:infocom:SripanidkulchaiMZ03}).
    %%Otros sistemas que proponen eficiencia por ambos lados, como 
    
    %\begin{itemize}
        %\item LIGHT
        %\item ...
        %\item 1- DHT de búsquedas (LIGHT) + CORPS
    %\end{itemize}
    
    
    
    %% TODO: Sacar bien el problema del paper de LIGHT
    %% El problema está en que todavía los métodos de indexación siguen siendo muy
    %% costosos para su implementación, creciendo exponencialmente su 
    
    %  En~\cite{5345647} se presenta un algoritmo de búsqueda Peer-to-Peer que utiliza el
    %conocimiento de los intereses de los usuarios para agruparlos en grupos para
    %encontrar contenidos de características similares. El algoritmo se ajusta
    %dinámicamente a las diferentes entradas y salidas de los usuarios, dividiendo o
    %juntando grupos a medida que los interés vallan cambiando. En comparación con
    %otros algoritmos de búsqueda como el
    %firework method~\cite{Ng02peerclustering}
    %y interest-based shortcut method~\cite{conf:infocom:SripanidkulchaiMZ03}
    % tiene una mayor robustez frente a una
    %red con altas tasas de entrada y salida de usuarios.
    
    
%    \subsubsection{Sistemas de subscripción/publicación}
%    
%    Una alternativa para la obtención del contenido de los usuarios entre sí sin
%    tener que realizar una búsqueda específica por el nuevo contenido compartido,
%    es la implementación de sistemas de subscripción, los cuales permites abstraer
%    al que publica de los problemas de distribución del contenido hacia los
%    subscriptores (o \textit{seguidores}) del mismo. Existen dos tipos de sistemas
%    de subscripción; los basados en tópicos y los basados en el contenido. Los
%    primeros, los generadores de contenido organizan la información publicada en tópicos a los cuales los otros
%    usuarios se suscriben. En cambio, los basados en tópicos publican el contenido
%    abiertamente a todos los suscriptores, siendo ellos los que pasan a filtrar el
%    contenido, según ciertos tópicos definidos por cada uno de los receptores de la
%    información.
%    
%    % TODO:
%    % mencionar porque estos sistemas son importantes (sistema de notificaciones y
%    % seguimiento de actualizaciones entre los usuarios)
%    % y los problemas involucrados en una red social p2p
%    
%    %%\subsection{No existen una precondición de confianza entre los nodos}
    


\subsection{Almacenamiento de los índices.}
\label{sec:index_storage}
    

    A la hora de indexar los documentos, dependiendo de la forma en
    que ésto se realice serán los costos de red incurridos en la busqueda.
    Si consideramos que los índices serán realizado por documento, repartiendo
    los documentos entre los usuarios del sistema, cada uno mantendría un índice
    invertido de forma local. Por ello, cada consulta debe ser distribuida
    hacia todos los nodos del sistema generando una inundación innecesaria de paquetes en el sistema.
    
    Si en cambio se particionan los índices según las palabras claves que
    aparecen en cada documento, un DHT puede ser usado para mapear una palabra
    con el nodo responsable de éste.
    El problema de ésto es que si no se ordena y limita los resultados de una búsqueda, una consulta de millones de resultados conllevaría a un enorme
    costo de bando de ancha de la red. Además, una consulta por varias palabras
    claves requeriría del envío sobre la red de los diferentes índices de parte
    de cada nodo responsable, realizando posteriormente el cruze  de la
    información resultante. En el caso de de pocos resultados, esto no es un
    gran problema, pero cruzar índices de billones o millones de resultados entre
    sí es complejo.

    Si vemos las restricciones de almacenamiento de los índices en un DHT y lo
    aplicamos al caso de Twitter, tenemos que, con
    340 millones de tweets por día, en un año tenemos alrededor de 124 Billones de
    documentos con un largo de hasta 140 carácteres cada uno. Si consideramos
    un promedio de 15 palabras por tweet, un índice invertido de estos
    contendría $124 * 10^9 * 15 = 1860 * 10^9$ identificadores, una cifra
    considerable si pensamos que es similar en orden de magnitud al número de
    índices manejados por google en el año 2003.
    Ahora, en un DHT, un identificador, que es una llave con la cual puede
    obtenerse el documento, es normalmente un hash de 20 bytes del contenido del
    documento.
    Si consideramos ese peso por cada índice a almancenar en el sistema,
    tenemos en total alrededor de $4 * 10^{13}$  bytes (40 TB). Ahora, gracias
    a la escalabilidad de este tipo de sistema, los índices pueden ser repartidos
    entre todos los usuarios de la red. Considerando la cantidad de usuarios de
    Twitter (140 millones de usuarios), tenemos que cada uno de ellos sólo
    debería dedicar para la indexación de archivos  alrededor de 30 MB, lo cual
    es un costo ínfimo considerando las capacidades de almacenamiento de hoy en
    día.


\subsection{Problemas encontrados}

\begin{enumerate}
    \item Falta de mecanismos eficientes para la realización de búsquedas complejas.
\end{enumerate}

  \subsubsection{Falta de mecanismos eficientes para la realización de búsquedas complejas}

     Existen un gran rango de sistemas de indexación propuestos, los cuales pueden clasificarse en
    dos tipos: las \textit{sobre DHTs} y las \textit{dependientes del sistema de red}.
    Las primeras permiten una fácil implementación sobre cualquier DHT existente,
    haciendo uso de las interfaces genéricas que éstos proveen, sin hacer modificación
    alguna al sistema de red por sobre el cual está montado. Por el contrario, las
    otras soluciones están estrechamente ligadas con el sistema de red en el cual
    fueron implementados, normalmente haciendo uso de modificaciones a los otros
    sistemas con los que interactúan de forma de hacer posible su funcionamiento
    y/o optimizar el rendimiento de sus funcionalidades.
    Dos puntos son críticos para el rendimiento de un sistema de
    búsquedas sobre un DHT, la \textit{eficiencia de la consulta} y el
    \textit{costo de mantención del índice}. En redes sociales, cada entrada y
    salida de un nodo usualmente resultan en operaciones de inserción o borrado
    en el sistema, a la vez que la cantidad de entradas y salidas dentro del
    sistema ser tan alto como la cantidad de consultas que se realizan. Esto
    significa que cada actualización en el sistema requiere de actualizaciones de
    los índices de los datos, haciendo que el costo de la mantención del índice un
    problema no menor. Por ello, se necesita que el sistema de indexación de datos
    sea eficiente tanto con los tiempo de búsqueda como con el costo de mantención
    del mismo. Sin embargo, muchos de los sistemas existentes se enfocan en bajos
    tiempos de búsqueda a cambio de un mayor costo de mantención, manteniendo un costo prohibitivamente alto
    que vuelve poco escalable al sistema P2P.  % dentro de estas soluciones, podemos mencionar a:

    Dentro de los métodos de indexación que permitan realizar
    búsquedas complejas (LIGHT~\cite{journals_tkde_TangZX10},
    DRing~\cite{hidalgo2011dring}), también existen diversas propuestas para mejorar otros tipos de
    búsquedas dentro de la red (~\cite{5345647}, ~\cite{Ng02peerclustering},
    ~\cite{conf:infocom:SripanidkulchaiMZ03}).
    %%Otros sistemas que proponen eficiencia por ambos lados, como 
    
    %\begin{itemize}
        %\item LIGHT
        %\item ...
        %\item 1- DHT de búsquedas (LIGHT) + CORPS
    %\end{itemize}
    
    
    
    %% TODO: Sacar bien el problema del paper de LIGHT
    %% El problema está en que todavía los métodos de indexación siguen siendo muy
    %% costosos para su implementación, creciendo exponencialmente su 
    
    %  En~\cite{5345647} se presenta un algoritmo de búsqueda Peer-to-Peer que utiliza el
    %conocimiento de los intereses de los usuarios para agruparlos en grupos para
    %encontrar contenidos de características similares. El algoritmo se ajusta
    %dinámicamente a las diferentes entradas y salidas de los usuarios, dividiendo o
    %juntando grupos a medida que los interés vallan cambiando. En comparación con
    %otros algoritmos de búsqueda como el
    %firework method~\cite{Ng02peerclustering}
    %y interest-based shortcut method~\cite{conf:infocom:SripanidkulchaiMZ03}
    % tiene una mayor robustez frente a una
    %red con altas tasas de entrada y salida de usuarios.
    
    
%    \subsubsection{Sistemas de subscripción/publicación}
%    
%    Una alternativa para la obtención del contenido de los usuarios entre sí sin
%    tener que realizar una búsqueda específica por el nuevo contenido compartido,
%    es la implementación de sistemas de subscripción, los cuales permites abstraer
%    al que publica de los problemas de distribución del contenido hacia los
%    subscriptores (o \textit{seguidores}) del mismo. Existen dos tipos de sistemas
%    de subscripción; los basados en tópicos y los basados en el contenido. Los
%    primeros, los generadores de contenido organizan la información publicada en tópicos a los cuales los otros
%    usuarios se suscriben. En cambio, los basados en tópicos publican el contenido
%    abiertamente a todos los suscriptores, siendo ellos los que pasan a filtrar el
%    contenido, según ciertos tópicos definidos por cada uno de los receptores de la
%    información.
%    
%    % TODO:
%    % mencionar porque estos sistemas son importantes (sistema de notificaciones y
%    % seguimiento de actualizaciones entre los usuarios)
%    % y los problemas involucrados en una red social p2p
%    
%    %%\subsection{No existen una precondición de confianza entre los nodos}
    


\subsection{Almacenamiento de los índices.}
\label{sec:index_storage}
    

    A la hora de indexar los documentos, dependiendo de la forma en
    que ésto se realice serán los costos de red incurridos en la busqueda.
    Si consideramos que los índices serán realizado por documento, repartiendo
    los documentos entre los usuarios del sistema, cada uno mantendría un índice
    invertido de forma local. Por ello, cada consulta debe ser distribuida
    hacia todos los nodos del sistema generando una inundación innecesaria de paquetes en el sistema.
    
    Si en cambio se particionan los índices según las palabras claves que
    aparecen en cada documento, un DHT puede ser usado para mapear una palabra
    con el nodo responsable de éste.
    El problema de ésto es que si no se ordena y limita los resultados de una búsqueda, una consulta de millones de resultados conllevaría a un enorme
    costo de bando de ancha de la red. Además, una consulta por varias palabras
    claves requeriría del envío sobre la red de los diferentes índices de parte
    de cada nodo responsable, realizando posteriormente el cruze  de la
    información resultante. En el caso de de pocos resultados, esto no es un
    gran problema, pero cruzar índices de billones o millones de resultados entre
    sí es complejo.

    Si vemos las restricciones de almacenamiento de los índices en un DHT y lo
    aplicamos al caso de Twitter, tenemos que, con
    340 millones de tweets por día, en un año tenemos alrededor de 124 Billones de
    documentos con un largo de hasta 140 carácteres cada uno. Si consideramos
    un promedio de 15 palabras por tweet, un índice invertido de estos
    contendría $124 * 10^9 * 15 = 1860 * 10^9$ identificadores, una cifra
    considerable si pensamos que es similar en orden de magnitud al número de
    índices manejados por google en el año 2003.
    Ahora, en un DHT, un identificador, que es una llave con la cual puede
    obtenerse el documento, es normalmente un hash de 20 bytes del contenido del
    documento.
    Si consideramos ese peso por cada índice a almancenar en el sistema,
    tenemos en total alrededor de $4 * 10^{13}$  bytes (40 TB). Ahora, gracias
    a la escalabilidad de este tipo de sistema, los índices pueden ser repartidos
    entre todos los usuarios de la red. Considerando la cantidad de usuarios de
    Twitter (140 millones de usuarios), tenemos que cada uno de ellos sólo
    debería dedicar para la indexación de archivos  alrededor de 30 MB, lo cual
    es un costo ínfimo considerando las capacidades de almacenamiento de hoy en
    día.


\subsection{Problemas encontrados}

\begin{enumerate}
    \item Falta de mecanismos eficientes para la realización de búsquedas complejas.
\end{enumerate}

  \subsubsection{Falta de mecanismos eficientes para la realización de búsquedas complejas}

     Existen un gran rango de sistemas de indexación propuestos, los cuales pueden clasificarse en
    dos tipos: las \textit{sobre DHTs} y las \textit{dependientes del sistema de red}.
    Las primeras permiten una fácil implementación sobre cualquier DHT existente,
    haciendo uso de las interfaces genéricas que éstos proveen, sin hacer modificación
    alguna al sistema de red por sobre el cual está montado. Por el contrario, las
    otras soluciones están estrechamente ligadas con el sistema de red en el cual
    fueron implementados, normalmente haciendo uso de modificaciones a los otros
    sistemas con los que interactúan de forma de hacer posible su funcionamiento
    y/o optimizar el rendimiento de sus funcionalidades.
    Dos puntos son críticos para el rendimiento de un sistema de
    búsquedas sobre un DHT, la \textit{eficiencia de la consulta} y el
    \textit{costo de mantención del índice}. En redes sociales, cada entrada y
    salida de un nodo usualmente resultan en operaciones de inserción o borrado
    en el sistema, a la vez que la cantidad de entradas y salidas dentro del
    sistema ser tan alto como la cantidad de consultas que se realizan. Esto
    significa que cada actualización en el sistema requiere de actualizaciones de
    los índices de los datos, haciendo que el costo de la mantención del índice un
    problema no menor. Por ello, se necesita que el sistema de indexación de datos
    sea eficiente tanto con los tiempo de búsqueda como con el costo de mantención
    del mismo. Sin embargo, muchos de los sistemas existentes se enfocan en bajos
    tiempos de búsqueda a cambio de un mayor costo de mantención, manteniendo un costo prohibitivamente alto
    que vuelve poco escalable al sistema P2P.  % dentro de estas soluciones, podemos mencionar a:

    Dentro de los métodos de indexación que permitan realizar
    búsquedas complejas (LIGHT~\cite{journals_tkde_TangZX10},
    DRing~\cite{hidalgo2011dring}), también existen diversas propuestas para mejorar otros tipos de
    búsquedas dentro de la red (~\cite{5345647}, ~\cite{Ng02peerclustering},
    ~\cite{conf:infocom:SripanidkulchaiMZ03}).
    %%Otros sistemas que proponen eficiencia por ambos lados, como 
    
    %\begin{itemize}
        %\item LIGHT
        %\item ...
        %\item 1- DHT de búsquedas (LIGHT) + CORPS
    %\end{itemize}
    
    
    
    %% TODO: Sacar bien el problema del paper de LIGHT
    %% El problema está en que todavía los métodos de indexación siguen siendo muy
    %% costosos para su implementación, creciendo exponencialmente su 
    
    %  En~\cite{5345647} se presenta un algoritmo de búsqueda Peer-to-Peer que utiliza el
    %conocimiento de los intereses de los usuarios para agruparlos en grupos para
    %encontrar contenidos de características similares. El algoritmo se ajusta
    %dinámicamente a las diferentes entradas y salidas de los usuarios, dividiendo o
    %juntando grupos a medida que los interés vallan cambiando. En comparación con
    %otros algoritmos de búsqueda como el
    %firework method~\cite{Ng02peerclustering}
    %y interest-based shortcut method~\cite{conf:infocom:SripanidkulchaiMZ03}
    % tiene una mayor robustez frente a una
    %red con altas tasas de entrada y salida de usuarios.
    
    
%    \subsubsection{Sistemas de subscripción/publicación}
%    
%    Una alternativa para la obtención del contenido de los usuarios entre sí sin
%    tener que realizar una búsqueda específica por el nuevo contenido compartido,
%    es la implementación de sistemas de subscripción, los cuales permites abstraer
%    al que publica de los problemas de distribución del contenido hacia los
%    subscriptores (o \textit{seguidores}) del mismo. Existen dos tipos de sistemas
%    de subscripción; los basados en tópicos y los basados en el contenido. Los
%    primeros, los generadores de contenido organizan la información publicada en tópicos a los cuales los otros
%    usuarios se suscriben. En cambio, los basados en tópicos publican el contenido
%    abiertamente a todos los suscriptores, siendo ellos los que pasan a filtrar el
%    contenido, según ciertos tópicos definidos por cada uno de los receptores de la
%    información.
%    
%    % TODO:
%    % mencionar porque estos sistemas son importantes (sistema de notificaciones y
%    % seguimiento de actualizaciones entre los usuarios)
%    % y los problemas involucrados en una red social p2p
%    
%    %%\subsection{No existen una precondición de confianza entre los nodos}
    
