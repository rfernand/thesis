El costo de pasar de un sistema centralizado a uno P2P son principalmente la
utilización de algoritmos más complejos, problemas de seguridad y
vulnerabilidades que permiten el abuso del sistema [Lua et al. 2005]

%1- que es una red p2p
  %Un nodo es un computador conectado a la red....

\subsubsection{Arquitectura}

Hay múltiples arquitecturas posibles para una red p2p. La elección de una de
ellas afecta la forma en que pueden realizarse búsquedas sobre la red. Para ser
capaces de buscar, se requiere un índice y una forma de calzar las consultas
contra las entradas de éste índice. A pesar de las diferentes aplicaciones que
se les pueda dar a las búsquedas en cada aplicación, los desafíos que se
enfrentan son los mismos: mantener una baja latencia mientras  mantienen las
propiedades benéficas de la red p2p, como la auto-organización y el balanceo de
carga [Daswani et al. 2003].
Basado en ésto, existen varias subtareas que afectan la latencia:

\subsubsection{Indexación}
  ¿Quién construye y actualiza el índice?
  ¿Donde se almacena y cual es el costo de su mantención?

  Los nodos involucrados en la estructuración de la información tienen una
carga de procesamiento mas alta que los demás. Sólo puede exister un gran
índice global, o cada nodo puede indexar su propio contenido.
Nodos pueden especializarse en sólo proveer espacio de almacenamiento si sólo
completan el índice, o pueden hacer ámbos. Donde el índice es almacenado
también afecta el ruteo de la consulta.


\subsubsection{Ruteo de la consulta}
  ¿Por que camino la consulta se enviará desde un nodo a otro que sea capaz de
  contestar la consulta?

  Caminos largos son costosos  en términos de latencia, y links y/o
computadoras lentas empeoran ésto. Por otro lado, la topología de la capa de red restringe los
posibles caminos a tomar.


\subsubsection{Procesamiento de la consulta}
  ¿Que nodo ejecuta el actual procesamiento de la consulta? 
    (generación de resultados por una específica consulta

  El tener más nodos involucrados en el proceso de la información aumenta la
latencia y hace la fusión de los resultados más compleja. Sin embargo, menos
  nodos involucrados aumenta la probabilidad de que resultados importantes sean ignorados.
