%\documentclass[letter, 12pt]{report}
\documentclass[conference]{IEEEtran}
%\usepackage{usmtesis}
\usepackage{graphicx}
\usepackage{url}
\usepackage{hyperref}
\usepackage[utf8]{inputenc}
\usepackage{enumerate}
\usepackage{amsfonts}
\usepackage{amsmath}
\usepackage{amsthm}
\usepackage{color}
\usepackage{fancyvrb}
\usepackage{fancyhdr}
\usepackage{listings}
\usepackage{slashbox}
\usepackage{natbib}

\definecolor{red}{rgb}{1,0,0}
\definecolor{green}{rgb}{0,1,0}
\definecolor{blue}{rgb}{0,0,1}
\newcommand{\blue}{\textcolor{blue}}
\newcommand{\red}{\textcolor{red}}
\newcommand{\green}{\textcolor{green}}
\newtheorem{mydef}{Definición}

\lhead{Estado del arte} %Parte superior izquierda
\rhead{\br \it Búsquedas en redes P2P}
\lfoot{} %Parte inferior izquierda.
\cfoot{} %Parte inferior central
\rfoot{\bf \thepage} %Parte inferior derecha
\renewcommand{\footrulewidth}{0.4pt} %Linea de separacion inferior

\lstset{ %
language=Octave,                % the language of the code
basicstyle=\footnotesize,       % the size of the fonts that are used for the code
numbers=left,                   % where to put the line-numbers
numberstyle=\footnotesize,      % the size of the fonts that are used for the line-numbers
stepnumber=1,                   % the step between two line-numbers. If it's 1, each line 
numbersep=5pt,                  % how far the line-numbers are from the code
backgroundcolor=\color{white},  % choose the background color. You must add \usepackage{color}
showspaces=false,               % show spaces adding particular underscores
showstringspaces=false,         % underline spaces within strings
showtabs=false,                 % show tabs within strings adding particular underscores
frame=single,                   % adds a frame around the code
tabsize=2,                      % sets default tabsize to 2 spaces
captionpos=b,                   % sets the caption-position to bottom
breaklines=true,                % sets automatic line breaking
breakatwhitespace=false,        % sets if automatic breaks should only happen at whitespace
title=\lstname,                 % show the filename of files included with \lstinputlisting; also try caption instead of title
escapeinside={\%*}{*)},         % if you want to add a comment within your code
morekeywords={*,...}            % if you want to add more keywords to the set
}


\title{Búsqueda en  redes P2P}

\author{
\IEEEauthorblockN{
Rodrigo G. Fernández\IEEEauthorrefmark{1}
}
\IEEEauthorblockA{
\IEEEauthorrefmark{1}Computer Systems Research Group~\cite{CSRG}, Universidad Técnica Federico Santa María, Av. ~España~1680, Valparaíso, Chile
%\IEEEauthorrefmark{2}Centro de Robótica~\cite{CR}, Universidad Técnica Federico Santa María, Av. ~España~1680, Valparaíso, Chile
}
}

\begin{document}

%\draft
%\nofront 

%\profguia{Xavier Bonnaire}
%\profcorr{Horst Von Brand}
%\beforepreface

%\titlep
\pagestyle{empty}
\maketitle\thispagestyle{empty}


%\tableofcontents

%\begin{dedication}
%\center Al desinteresado de alma y al corazón que no conoce razón.
%\end{dedication}
%
%\prefacesection{Agradecimientos}
%A todos los que me conocieron:
%Agradesco a todos los que me han acompañado durante mi estadía en esta
%universidad, de verdad que me han significado mucho para mi, y gracias a
%ustedes he logrado aprender muchas cosas que sólo nunca hubiera alcanzado.
%
%\prefacesection{Resumen}
%Los servicios P2P son robustos, escalables y auto-organizados por naturaleza,
pero su arquitectura diferente trae nuevos problemas y requerimientos.
Tradicionalmente, las redes P2P identifican sólo los nodos que componen el
sistema, sin diferenciar a los usuarios detrás de cada uno de ellos.
Hoy en día, las personas usan más de un dispositivo para conectarse a la red.
Este cambio de comportamiento hace que una identificación por usuario sea
necesaria. Además, el usuario normal está acostumbrado al uso de nombres de
usuarios y contraseñas para identificarse en éstos sistemas.
Mientras que existen propuestas para la implementación de un sistema de
identificación por nombre de usuario y contraseña, éstos no contemplan la
existencia de nodos maliciosos, faltando los mecanismos adecuados para
mantener seguro el proceso de identificacion.

El trabajo a continuación investiga profundamente los requerimientos y
características necesarias para un sistema de identificación seguro, junto con
los desafíos encontrados para su implementación en redes P2P.

%El objetivo principal de este trabajo es desarrollar un sistema de
%identificación de usuarios basado en nombre de usuario y contraseña en redes
%estructuradas P2P, usando sistemas de reputación y administración de nodos
%confiables para protegerse en contra de nodos maliciosos y implementar protocolos seguros
%en el sistema.

%Se realizará un análisis teórico del sistema para asegurar que los protocolos
%son seguros y no presentan riesgos en ambientes reales.
%
%La arquitectura para desarrollar el sistema será en redes P2P basadas en  Distributed Hash
%Tables (DHT).\\

{\bf Keywords:} P2P, identificación de usuario, sistemas distribuidos.

%\newpage
%
%\prefacesection{Abstract}
%Este documento presenta
%el avance y proceso de creación de una aplicación que permita desarrollar las
%rutinas de control para diferentes tipos de plataformas robóticas, unificando los diferentes mecanismos de comunicación
%con cada dispositivo en un solo lenguaje de programación, buscando simplificar la programación de las
%plataformas robóticas y hacer más eficiente el trabajo del investigador.
%Dentro de las consideraciones se encuentran: (a) la elección de un lenguaje estandarizado que posibilite un fácil
%aprendizaje y permita realizar una amplia gama de tareas; (b) la estructura del software a desarrollar, y (c)
%la perduración del mismo en el tiempo.
%Tras haber considerado todos estos puntos, se estableció que nuestro software sería desarrollado en
%Python, con una estructura modular y extensible utilizando archivos XML, y de código abierto.
%Se presentan resultados parciales sobre el desarrollo del proyecto y del software.

\textbf{Palabras Clave:}  redes sociales, p2p, 

%\newpage

%\afterpreface


%\renewcommand{\chaptername}{Capítulo}
%\chapter{Introducción}
\section{Introducción}
\label{sec:intro}
%% TODO:
%% La introducción está mal organizada. Se trata de explicar lo siguiente:
%% - Cúal es el problema?
    En un sistema centralizado, todos los algoritmos de búsqueda son completos,
    debido a que la información es indexada bajo bases de datos en donde la
    realización de consultas complejas no son un problema. Como poseen el
    conociento completo de los índices y documentos almacenados en el sistema,
    se puede sin grandes dificultades
    ordenar los resultados obtenidos y presentarlos de una forma adecuada. Ahora, el
    almacenamiento debe ser manejado por el sistema, al igual que
    los costos de ancho banda requerido para contestar todas las consultas
    requeridas, incurriendo a costos proporcionales a la cantidad de recursos
    utilizados. 

    A diferencia de los sistemas centralizados, el tema es especialmente complejo en redes basadas en DHT, principalmente
    porque las tablas de hash uniforme destruyen el todo orden para la
    localización de la información, quedando esta distribuida en cada nodo de
    la red sin existir una relación entre ellos y la información que almacenan.
    Para poder realizar búsquedas de éste estilo, es necesaria la
    implementación de índices sobre los datos almacenados en ellas.

    Por otro lado, como la información se encuentra distribuida entre los
    diferentes nodos que conforman la red, no es fácil implementar un sistema
    que ordene los resultados de cada búsqueda y entregue, por ejemplo, los 20
    más importantes. 
    Normalmente, los sistemas de
    indexación que se proponen para redes P2P no tratan el problema del manejo de
    miles de resultados, dejando ese trabajo al nodo que realizó la consulta.

    Otro punto importante es que sobre la calidad de los resultados que deben
    obtenerse. En el caso de una consulta con miles de resultados, el sistema
    debe saber discriminar cuales son los más relevantes para el usuario, ordenados
    según criterios implícitos o explícitos para ello.

    Además de las funcionalidades básicas esperadas para este tipo de
    búsquedas, se deben considerar que su implementación sea posible según
    las capacidades del sistema. Por ello, debe
    considerarse el almacenamiento requerido para la mantención de los índices
    y los gastos de red requeridos para la realización de la búsqueda
    en sí.

%% - Porque es un problema importante o actual?

    La búsquedas de datos dentro de las redes sociales  son de vital importancia para la formación de la
    red. A través de búsquedas es como los usuarios
    pueden encontrar a nuevos contactos dentro de la red y encontrar contenido
    específico según las diferentes necesidades de cada usuario.  Para ello, es
    necesario contar con un sistema que permita realizar búsquedas complejas.
    Esto considera poder buscar elementos que se encuentren dentro de un cierto
    rando de valores y/o que calcen con cierto conjunto de palabras claves.


%% - Cuál es el objetivo de la memoria?

%% -Al final hay que presentar la estructura de la memoria con los capítulos
%% posteriores.

A continuación, se desarrollará el trabajo con la siguiente estructura:

En la Sección~\ref{sec:soa_indexing}
.......
En la Sección~\ref{sec:soa_ranking} 
......
 concluyendo y viendo el trabajo futuro
en la Sección~\ref{sec:conclusiones}.



\section{Desafíos de un sistema de búsqueda P2P}
\label{sec:challenges}
El costo de pasar de un sistema centralizado a uno P2P son principalmente la
utilización de algoritmos más complejos, problemas de seguridad y
vulnerabilidades que permiten el abuso del sistema [Lua et al. 2005]

%1- que es una red p2p
  %Un nodo es un computador conectado a la red....

\subsubsection{Arquitectura}

Hay múltiples arquitecturas posibles para una red p2p. La elección de una de
ellas afecta la forma en que pueden realizarse búsquedas sobre la red. Para ser
capaces de buscar, se requiere un índice y una forma de calzar las consultas
contra las entradas de éste índice. A pesar de las diferentes aplicaciones que
se les pueda dar a las búsquedas en cada aplicación, los desafíos que se
enfrentan son los mismos: mantener una baja latencia mientras  mantienen las
propiedades benéficas de la red p2p, como la auto-organización y el balanceo de
carga [Daswani et al. 2003].
Basado en ésto, existen varias subtareas que afectan la latencia:

\subsubsection{Indexación}
  ¿Quién construye y actualiza el índice?
  ¿Donde se almacena y cual es el costo de su mantención?

  Los nodos involucrados en la estructuración de la información tienen una
carga de procesamiento mas alta que los demás. Sólo puede exister un gran
índice global, o cada nodo puede indexar su propio contenido.
Nodos pueden especializarse en sólo proveer espacio de almacenamiento si sólo
completan el índice, o pueden hacer ámbos. Donde el índice es almacenado
también afecta el ruteo de la consulta.


\subsubsection{Ruteo de la consulta}
  ¿Por que camino la consulta se enviará desde un nodo a otro que sea capaz de
  contestar la consulta?

  Caminos largos son costosos  en términos de latencia, y links y/o
computadoras lentas empeoran ésto. Por otro lado, la topología de la capa de red restringe los
posibles caminos a tomar.


\subsubsection{Procesamiento de la consulta}
  ¿Que nodo ejecuta el actual procesamiento de la consulta? 
    (generación de resultados por una específica consulta

  El tener más nodos involucrados en el proceso de la información aumenta la
latencia y hace la fusión de los resultados más compleja. Sin embargo, menos
  nodos involucrados aumenta la probabilidad de que resultados importantes sean ignorados.


%\chapter{Estado del Arte: Sistemas de indexación}
\section{Estado del Arte: Sistemas de indexación}
\label{sec:soa_indexing}
%%%\input{src/intros/soa_indexing}

\subsection{Almacenamiento de los índices.}
\label{sec:index_storage}
    

    A la hora de indexar los documentos, dependiendo de la forma en
    que ésto se realice serán los costos de red incurridos en la busqueda.
    Si consideramos que los índices serán realizado por documento, repartiendo
    los documentos entre los usuarios del sistema, cada uno mantendría un índice
    invertido de forma local. Por ello, cada consulta debe ser distribuida
    hacia todos los nodos del sistema generando una inundación innecesaria de paquetes en el sistema.
    
    Si en cambio se particionan los índices según las palabras claves que
    aparecen en cada documento, un DHT puede ser usado para mapear una palabra
    con el nodo responsable de éste.
    El problema de ésto es que si no se ordena y limita los resultados de una búsqueda, una consulta de millones de resultados conllevaría a un enorme
    costo de bando de ancha de la red. Además, una consulta por varias palabras
    claves requeriría del envío sobre la red de los diferentes índices de parte
    de cada nodo responsable, realizando posteriormente el cruze  de la
    información resultante. En el caso de de pocos resultados, esto no es un
    gran problema, pero cruzar índices de billones o millones de resultados entre
    sí es complejo.

    Si vemos las restricciones de almacenamiento de los índices en un DHT y lo
    aplicamos al caso de Twitter, tenemos que, con
    340 millones de tweets por día, en un año tenemos alrededor de 124 Billones de
    documentos con un largo de hasta 140 carácteres cada uno. Si consideramos
    un promedio de 15 palabras por tweet, un índice invertido de estos
    contendría $124 * 10^9 * 15 = 1860 * 10^9$ identificadores, una cifra
    considerable si pensamos que es similar en orden de magnitud al número de
    índices manejados por google en el año 2003.
    Ahora, en un DHT, un identificador, que es una llave con la cual puede
    obtenerse el documento, es normalmente un hash de 20 bytes del contenido del
    documento.
    Si consideramos ese peso por cada índice a almancenar en el sistema,
    tenemos en total alrededor de $4 * 10^{13}$  bytes (40 TB). Ahora, gracias
    a la escalabilidad de este tipo de sistema, los índices pueden ser repartidos
    entre todos los usuarios de la red. Considerando la cantidad de usuarios de
    Twitter (140 millones de usuarios), tenemos que cada uno de ellos sólo
    debería dedicar para la indexación de archivos  alrededor de 30 MB, lo cual
    es un costo ínfimo considerando las capacidades de almacenamiento de hoy en
    día.


\subsection{Problemas encontrados}

\begin{enumerate}
    \item Falta de mecanismos eficientes para la realización de búsquedas complejas.
\end{enumerate}

  \subsubsection{Falta de mecanismos eficientes para la realización de búsquedas complejas}

     Existen un gran rango de sistemas de indexación propuestos, los cuales pueden clasificarse en
    dos tipos: las \textit{sobre DHTs} y las \textit{dependientes del sistema de red}.
    Las primeras permiten una fácil implementación sobre cualquier DHT existente,
    haciendo uso de las interfaces genéricas que éstos proveen, sin hacer modificación
    alguna al sistema de red por sobre el cual está montado. Por el contrario, las
    otras soluciones están estrechamente ligadas con el sistema de red en el cual
    fueron implementados, normalmente haciendo uso de modificaciones a los otros
    sistemas con los que interactúan de forma de hacer posible su funcionamiento
    y/o optimizar el rendimiento de sus funcionalidades.
    Dos puntos son críticos para el rendimiento de un sistema de
    búsquedas sobre un DHT, la \textit{eficiencia de la consulta} y el
    \textit{costo de mantención del índice}. En redes sociales, cada entrada y
    salida de un nodo usualmente resultan en operaciones de inserción o borrado
    en el sistema, a la vez que la cantidad de entradas y salidas dentro del
    sistema ser tan alto como la cantidad de consultas que se realizan. Esto
    significa que cada actualización en el sistema requiere de actualizaciones de
    los índices de los datos, haciendo que el costo de la mantención del índice un
    problema no menor. Por ello, se necesita que el sistema de indexación de datos
    sea eficiente tanto con los tiempo de búsqueda como con el costo de mantención
    del mismo. Sin embargo, muchos de los sistemas existentes se enfocan en bajos
    tiempos de búsqueda a cambio de un mayor costo de mantención, manteniendo un costo prohibitivamente alto
    que vuelve poco escalable al sistema P2P.  % dentro de estas soluciones, podemos mencionar a:

    Dentro de los métodos de indexación que permitan realizar
    búsquedas complejas (LIGHT~\cite{journals_tkde_TangZX10},
    DRing~\cite{hidalgo2011dring}), también existen diversas propuestas para mejorar otros tipos de
    búsquedas dentro de la red (~\cite{5345647}, ~\cite{Ng02peerclustering},
    ~\cite{conf:infocom:SripanidkulchaiMZ03}).
    %%Otros sistemas que proponen eficiencia por ambos lados, como 
    
    %\begin{itemize}
        %\item LIGHT
        %\item ...
        %\item 1- DHT de búsquedas (LIGHT) + CORPS
    %\end{itemize}
    
    
    
    %% TODO: Sacar bien el problema del paper de LIGHT
    %% El problema está en que todavía los métodos de indexación siguen siendo muy
    %% costosos para su implementación, creciendo exponencialmente su 
    
    %  En~\cite{5345647} se presenta un algoritmo de búsqueda Peer-to-Peer que utiliza el
    %conocimiento de los intereses de los usuarios para agruparlos en grupos para
    %encontrar contenidos de características similares. El algoritmo se ajusta
    %dinámicamente a las diferentes entradas y salidas de los usuarios, dividiendo o
    %juntando grupos a medida que los interés vallan cambiando. En comparación con
    %otros algoritmos de búsqueda como el
    %firework method~\cite{Ng02peerclustering}
    %y interest-based shortcut method~\cite{conf:infocom:SripanidkulchaiMZ03}
    % tiene una mayor robustez frente a una
    %red con altas tasas de entrada y salida de usuarios.
    
    
%    \subsubsection{Sistemas de subscripción/publicación}
%    
%    Una alternativa para la obtención del contenido de los usuarios entre sí sin
%    tener que realizar una búsqueda específica por el nuevo contenido compartido,
%    es la implementación de sistemas de subscripción, los cuales permites abstraer
%    al que publica de los problemas de distribución del contenido hacia los
%    subscriptores (o \textit{seguidores}) del mismo. Existen dos tipos de sistemas
%    de subscripción; los basados en tópicos y los basados en el contenido. Los
%    primeros, los generadores de contenido organizan la información publicada en tópicos a los cuales los otros
%    usuarios se suscriben. En cambio, los basados en tópicos publican el contenido
%    abiertamente a todos los suscriptores, siendo ellos los que pasan a filtrar el
%    contenido, según ciertos tópicos definidos por cada uno de los receptores de la
%    información.
%    
%    % TODO:
%    % mencionar porque estos sistemas son importantes (sistema de notificaciones y
%    % seguimiento de actualizaciones entre los usuarios)
%    % y los problemas involucrados en una red social p2p
%    
%    %%\subsection{No existen una precondición de confianza entre los nodos}
    


\subsection{Almacenamiento de los índices.}
\label{sec:index_storage}
    

    A la hora de indexar los documentos, dependiendo de la forma en
    que ésto se realice serán los costos de red incurridos en la busqueda.
    Si consideramos que los índices serán realizado por documento, repartiendo
    los documentos entre los usuarios del sistema, cada uno mantendría un índice
    invertido de forma local. Por ello, cada consulta debe ser distribuida
    hacia todos los nodos del sistema generando una inundación innecesaria de paquetes en el sistema.
    
    Si en cambio se particionan los índices según las palabras claves que
    aparecen en cada documento, un DHT puede ser usado para mapear una palabra
    con el nodo responsable de éste.
    El problema de ésto es que si no se ordena y limita los resultados de una búsqueda, una consulta de millones de resultados conllevaría a un enorme
    costo de bando de ancha de la red. Además, una consulta por varias palabras
    claves requeriría del envío sobre la red de los diferentes índices de parte
    de cada nodo responsable, realizando posteriormente el cruze  de la
    información resultante. En el caso de de pocos resultados, esto no es un
    gran problema, pero cruzar índices de billones o millones de resultados entre
    sí es complejo.

    Si vemos las restricciones de almacenamiento de los índices en un DHT y lo
    aplicamos al caso de Twitter, tenemos que, con
    340 millones de tweets por día, en un año tenemos alrededor de 124 Billones de
    documentos con un largo de hasta 140 carácteres cada uno. Si consideramos
    un promedio de 15 palabras por tweet, un índice invertido de estos
    contendría $124 * 10^9 * 15 = 1860 * 10^9$ identificadores, una cifra
    considerable si pensamos que es similar en orden de magnitud al número de
    índices manejados por google en el año 2003.
    Ahora, en un DHT, un identificador, que es una llave con la cual puede
    obtenerse el documento, es normalmente un hash de 20 bytes del contenido del
    documento.
    Si consideramos ese peso por cada índice a almancenar en el sistema,
    tenemos en total alrededor de $4 * 10^{13}$  bytes (40 TB). Ahora, gracias
    a la escalabilidad de este tipo de sistema, los índices pueden ser repartidos
    entre todos los usuarios de la red. Considerando la cantidad de usuarios de
    Twitter (140 millones de usuarios), tenemos que cada uno de ellos sólo
    debería dedicar para la indexación de archivos  alrededor de 30 MB, lo cual
    es un costo ínfimo considerando las capacidades de almacenamiento de hoy en
    día.


\subsection{Problemas encontrados}

\begin{enumerate}
    \item Falta de mecanismos eficientes para la realización de búsquedas complejas.
\end{enumerate}

  \subsubsection{Falta de mecanismos eficientes para la realización de búsquedas complejas}

     Existen un gran rango de sistemas de indexación propuestos, los cuales pueden clasificarse en
    dos tipos: las \textit{sobre DHTs} y las \textit{dependientes del sistema de red}.
    Las primeras permiten una fácil implementación sobre cualquier DHT existente,
    haciendo uso de las interfaces genéricas que éstos proveen, sin hacer modificación
    alguna al sistema de red por sobre el cual está montado. Por el contrario, las
    otras soluciones están estrechamente ligadas con el sistema de red en el cual
    fueron implementados, normalmente haciendo uso de modificaciones a los otros
    sistemas con los que interactúan de forma de hacer posible su funcionamiento
    y/o optimizar el rendimiento de sus funcionalidades.
    Dos puntos son críticos para el rendimiento de un sistema de
    búsquedas sobre un DHT, la \textit{eficiencia de la consulta} y el
    \textit{costo de mantención del índice}. En redes sociales, cada entrada y
    salida de un nodo usualmente resultan en operaciones de inserción o borrado
    en el sistema, a la vez que la cantidad de entradas y salidas dentro del
    sistema ser tan alto como la cantidad de consultas que se realizan. Esto
    significa que cada actualización en el sistema requiere de actualizaciones de
    los índices de los datos, haciendo que el costo de la mantención del índice un
    problema no menor. Por ello, se necesita que el sistema de indexación de datos
    sea eficiente tanto con los tiempo de búsqueda como con el costo de mantención
    del mismo. Sin embargo, muchos de los sistemas existentes se enfocan en bajos
    tiempos de búsqueda a cambio de un mayor costo de mantención, manteniendo un costo prohibitivamente alto
    que vuelve poco escalable al sistema P2P.  % dentro de estas soluciones, podemos mencionar a:

    Dentro de los métodos de indexación que permitan realizar
    búsquedas complejas (LIGHT~\cite{journals_tkde_TangZX10},
    DRing~\cite{hidalgo2011dring}), también existen diversas propuestas para mejorar otros tipos de
    búsquedas dentro de la red (~\cite{5345647}, ~\cite{Ng02peerclustering},
    ~\cite{conf:infocom:SripanidkulchaiMZ03}).
    %%Otros sistemas que proponen eficiencia por ambos lados, como 
    
    %\begin{itemize}
        %\item LIGHT
        %\item ...
        %\item 1- DHT de búsquedas (LIGHT) + CORPS
    %\end{itemize}
    
    
    
    %% TODO: Sacar bien el problema del paper de LIGHT
    %% El problema está en que todavía los métodos de indexación siguen siendo muy
    %% costosos para su implementación, creciendo exponencialmente su 
    
    %  En~\cite{5345647} se presenta un algoritmo de búsqueda Peer-to-Peer que utiliza el
    %conocimiento de los intereses de los usuarios para agruparlos en grupos para
    %encontrar contenidos de características similares. El algoritmo se ajusta
    %dinámicamente a las diferentes entradas y salidas de los usuarios, dividiendo o
    %juntando grupos a medida que los interés vallan cambiando. En comparación con
    %otros algoritmos de búsqueda como el
    %firework method~\cite{Ng02peerclustering}
    %y interest-based shortcut method~\cite{conf:infocom:SripanidkulchaiMZ03}
    % tiene una mayor robustez frente a una
    %red con altas tasas de entrada y salida de usuarios.
    
    
%    \subsubsection{Sistemas de subscripción/publicación}
%    
%    Una alternativa para la obtención del contenido de los usuarios entre sí sin
%    tener que realizar una búsqueda específica por el nuevo contenido compartido,
%    es la implementación de sistemas de subscripción, los cuales permites abstraer
%    al que publica de los problemas de distribución del contenido hacia los
%    subscriptores (o \textit{seguidores}) del mismo. Existen dos tipos de sistemas
%    de subscripción; los basados en tópicos y los basados en el contenido. Los
%    primeros, los generadores de contenido organizan la información publicada en tópicos a los cuales los otros
%    usuarios se suscriben. En cambio, los basados en tópicos publican el contenido
%    abiertamente a todos los suscriptores, siendo ellos los que pasan a filtrar el
%    contenido, según ciertos tópicos definidos por cada uno de los receptores de la
%    información.
%    
%    % TODO:
%    % mencionar porque estos sistemas son importantes (sistema de notificaciones y
%    % seguimiento de actualizaciones entre los usuarios)
%    % y los problemas involucrados en una red social p2p
%    
%    %%\subsection{No existen una precondición de confianza entre los nodos}
    


\subsection{Almacenamiento de los índices.}
\label{sec:index_storage}
    

    A la hora de indexar los documentos, dependiendo de la forma en
    que ésto se realice serán los costos de red incurridos en la busqueda.
    Si consideramos que los índices serán realizado por documento, repartiendo
    los documentos entre los usuarios del sistema, cada uno mantendría un índice
    invertido de forma local. Por ello, cada consulta debe ser distribuida
    hacia todos los nodos del sistema generando una inundación innecesaria de paquetes en el sistema.
    
    Si en cambio se particionan los índices según las palabras claves que
    aparecen en cada documento, un DHT puede ser usado para mapear una palabra
    con el nodo responsable de éste.
    El problema de ésto es que si no se ordena y limita los resultados de una búsqueda, una consulta de millones de resultados conllevaría a un enorme
    costo de bando de ancha de la red. Además, una consulta por varias palabras
    claves requeriría del envío sobre la red de los diferentes índices de parte
    de cada nodo responsable, realizando posteriormente el cruze  de la
    información resultante. En el caso de de pocos resultados, esto no es un
    gran problema, pero cruzar índices de billones o millones de resultados entre
    sí es complejo.

    Si vemos las restricciones de almacenamiento de los índices en un DHT y lo
    aplicamos al caso de Twitter, tenemos que, con
    340 millones de tweets por día, en un año tenemos alrededor de 124 Billones de
    documentos con un largo de hasta 140 carácteres cada uno. Si consideramos
    un promedio de 15 palabras por tweet, un índice invertido de estos
    contendría $124 * 10^9 * 15 = 1860 * 10^9$ identificadores, una cifra
    considerable si pensamos que es similar en orden de magnitud al número de
    índices manejados por google en el año 2003.
    Ahora, en un DHT, un identificador, que es una llave con la cual puede
    obtenerse el documento, es normalmente un hash de 20 bytes del contenido del
    documento.
    Si consideramos ese peso por cada índice a almancenar en el sistema,
    tenemos en total alrededor de $4 * 10^{13}$  bytes (40 TB). Ahora, gracias
    a la escalabilidad de este tipo de sistema, los índices pueden ser repartidos
    entre todos los usuarios de la red. Considerando la cantidad de usuarios de
    Twitter (140 millones de usuarios), tenemos que cada uno de ellos sólo
    debería dedicar para la indexación de archivos  alrededor de 30 MB, lo cual
    es un costo ínfimo considerando las capacidades de almacenamiento de hoy en
    día.


\subsection{Problemas encontrados}

\begin{enumerate}
    \item Falta de mecanismos eficientes para la realización de búsquedas complejas.
\end{enumerate}

  \subsubsection{Falta de mecanismos eficientes para la realización de búsquedas complejas}

     Existen un gran rango de sistemas de indexación propuestos, los cuales pueden clasificarse en
    dos tipos: las \textit{sobre DHTs} y las \textit{dependientes del sistema de red}.
    Las primeras permiten una fácil implementación sobre cualquier DHT existente,
    haciendo uso de las interfaces genéricas que éstos proveen, sin hacer modificación
    alguna al sistema de red por sobre el cual está montado. Por el contrario, las
    otras soluciones están estrechamente ligadas con el sistema de red en el cual
    fueron implementados, normalmente haciendo uso de modificaciones a los otros
    sistemas con los que interactúan de forma de hacer posible su funcionamiento
    y/o optimizar el rendimiento de sus funcionalidades.
    Dos puntos son críticos para el rendimiento de un sistema de
    búsquedas sobre un DHT, la \textit{eficiencia de la consulta} y el
    \textit{costo de mantención del índice}. En redes sociales, cada entrada y
    salida de un nodo usualmente resultan en operaciones de inserción o borrado
    en el sistema, a la vez que la cantidad de entradas y salidas dentro del
    sistema ser tan alto como la cantidad de consultas que se realizan. Esto
    significa que cada actualización en el sistema requiere de actualizaciones de
    los índices de los datos, haciendo que el costo de la mantención del índice un
    problema no menor. Por ello, se necesita que el sistema de indexación de datos
    sea eficiente tanto con los tiempo de búsqueda como con el costo de mantención
    del mismo. Sin embargo, muchos de los sistemas existentes se enfocan en bajos
    tiempos de búsqueda a cambio de un mayor costo de mantención, manteniendo un costo prohibitivamente alto
    que vuelve poco escalable al sistema P2P.  % dentro de estas soluciones, podemos mencionar a:

    Dentro de los métodos de indexación que permitan realizar
    búsquedas complejas (LIGHT~\cite{journals_tkde_TangZX10},
    DRing~\cite{hidalgo2011dring}), también existen diversas propuestas para mejorar otros tipos de
    búsquedas dentro de la red (~\cite{5345647}, ~\cite{Ng02peerclustering},
    ~\cite{conf:infocom:SripanidkulchaiMZ03}).
    %%Otros sistemas que proponen eficiencia por ambos lados, como 
    
    %\begin{itemize}
        %\item LIGHT
        %\item ...
        %\item 1- DHT de búsquedas (LIGHT) + CORPS
    %\end{itemize}
    
    
    
    %% TODO: Sacar bien el problema del paper de LIGHT
    %% El problema está en que todavía los métodos de indexación siguen siendo muy
    %% costosos para su implementación, creciendo exponencialmente su 
    
    %  En~\cite{5345647} se presenta un algoritmo de búsqueda Peer-to-Peer que utiliza el
    %conocimiento de los intereses de los usuarios para agruparlos en grupos para
    %encontrar contenidos de características similares. El algoritmo se ajusta
    %dinámicamente a las diferentes entradas y salidas de los usuarios, dividiendo o
    %juntando grupos a medida que los interés vallan cambiando. En comparación con
    %otros algoritmos de búsqueda como el
    %firework method~\cite{Ng02peerclustering}
    %y interest-based shortcut method~\cite{conf:infocom:SripanidkulchaiMZ03}
    % tiene una mayor robustez frente a una
    %red con altas tasas de entrada y salida de usuarios.
    
    
%    \subsubsection{Sistemas de subscripción/publicación}
%    
%    Una alternativa para la obtención del contenido de los usuarios entre sí sin
%    tener que realizar una búsqueda específica por el nuevo contenido compartido,
%    es la implementación de sistemas de subscripción, los cuales permites abstraer
%    al que publica de los problemas de distribución del contenido hacia los
%    subscriptores (o \textit{seguidores}) del mismo. Existen dos tipos de sistemas
%    de subscripción; los basados en tópicos y los basados en el contenido. Los
%    primeros, los generadores de contenido organizan la información publicada en tópicos a los cuales los otros
%    usuarios se suscriben. En cambio, los basados en tópicos publican el contenido
%    abiertamente a todos los suscriptores, siendo ellos los que pasan a filtrar el
%    contenido, según ciertos tópicos definidos por cada uno de los receptores de la
%    información.
%    
%    % TODO:
%    % mencionar porque estos sistemas son importantes (sistema de notificaciones y
%    % seguimiento de actualizaciones entre los usuarios)
%    % y los problemas involucrados en una red social p2p
%    
%    %%\subsection{No existen una precondición de confianza entre los nodos}
    


%\section{Estado del Arte: Rankeo de resultados}
%\label{sec:soa_ranking}
%\input{src/soa_ranking}

\section{Estado del Arte: Selección de $k$ resultados}
\label{sec:soa_top_results}


%4.2.3. Limiting the Number of Results to Process with Top k Approaches. Processing only a
%subset of items during the search process can yield performance benefits: less data
%processing and lower latency. Various algorithms, discussed shortly, can be used to
%retrieve the top items for a particular query without having to calculate the scores for
%all the items. Retrieving top items makes sense as it has been shown that users of
%Web search engines prefer quality over quantity with respect to search results: more
%precision and less recall [Oulasvirta et al. 2009]. Top k approaches have been applied
%to various architectures and at various stages in peer-to-peer information retrieval:

Al realizar una búsqueda, los usuarios prefieren calidad por sobre cantidad de
resultados. Por ello, en vez de entregar millones de resultados para una
consulta, lo que se busca es obtener los objetos más relevantes para el usuario~\citealt{oulasvirta2009more}. %[Oulasvirta et al. 2009].
Para limitar la cantidad de resultados obtenidos y mejorar la calidad de éstos,
existen propuestas para obtener sólo los $k$ resultados mejor evaluados.
Esto trae múltiples beneficios para el sistema. El procesar sólo un subconjunto
de los objetos durante una búsqueda mejora el rendimiento al reducir la
cantidad de información que se debe manejar, otorgando una menor latencia para las búsquedas.

Dentro de los algoritmos propuestos para lograr esto, se encuentran:



%
%— Top k results requesting [Cuenca-Acuna et al. 2003]
%A simple way to optimise the system is to only request the top results. Approaches
%that use local indices always apply a variable form of limited result requesting im-
%plicitly by bounding the number of hops made when flooding or by performing a
%random walk that terminates. However, that number can also be explicitly set to
%a constant by the requester as is done for the globally replicated index used by
%[Cuenca-Acuna et al. 2003]. They first obtain a list of k search results and keep con-
%tacting nodes as long as the chance of them contributing to this top k remains high.
%The top results stabilise after a few rounds.

\paragraph{Top k results requesting~\cite{cuenca2003planetp}} % [Cuenca-Acuna et al. 2003]
Una forma simple de optimizar un sistema es haciendo que sólo se pidan los
resultados ``top''. Sistemas que usan índices locales siempre aplican una forma
variable para limitar los resultados. Por ejemplo, al limitar la cantidad de saltos a realizar
cuando se busca por \textit{flooding} o a través de un \textit{random walk}.
Sin embargo, ese número también puede ser explícitamente puesto en una
constante por quien realiza la consulta, como lo hecho por~\citealt{cuenca2003planetp} % [Cuenca-Acuna et al. 2003]
para índices globalmente replicados. Ellos primero obtienen una lista de $k$
resultados de búsqueda y continúan contactando nodos mientras que se mantenga una
probabilidad alta de que éstos aporten a la lista de $k$ resultados ``top''. La lista de
los ``top'' $k$ resultados se estabilizan después de unas cuantas rondas.

%
%— Top k query processing [Suel et al. 2003; Balke et al. 2005; Michel et al. 2005a;
%Zhang and Suel 2005]
%This approach has its roots in the database community, particularly in the work of
%[Fagin et al. 2001]. Several variations exist, all with the same basic idea: we can
%determine the top k documents given several input lists without having to examine
%these lists completely and while not adversely affecting performance. This is often
%used in cases where a distributed global index is used and posting lists have to be
%intersected. The threshold algorithm is the most popular [Michel et al. 2005a; Suel
%et al. 2003]. This algorithm maintains two data structures: a queue with peers to
%contact for obtaining search results and a list with the current top k results. Peers in
%the queue are processed one by one, each returning a limited set of k search results
%of the form (document, score) sorted by score in descending order. For a distributed
%global index these are the top items in the posting list for a particular term. The
%algorithm tracks two scores for each unique document: worst and best. The worst
%score is the sum of the scores for a document d found in all result lists in which
%d appeared. The best score is the worst score plus the lowest score (of some other
%document) encountered in the result lists in which d did not appear. Since all the
%result lists are truncated, this last score forms an upper bound of the best possible
%score that would be achievable for document d. The current top k is formed by the
%highest scoring documents seen so far based on their worst score. If the best score
%of a document is lower than the threshold, which is the worst score of the docu-
%ment at position k in the current top k results, it need not be considered for the
%top k. The algorithm thus bases the final intersection on only the top k results from
%each peer, which provably yields performance equivalent to ‘sequentially’ intersect-
%ing the entire lists. This thus saves both bandwidth and computational costs without
%negatively affecting result quality. A drawback is that looking up document scores
%requires random access to the result lists [Suel et al. 2003]. [Zhang and Suel 2005]
%later investigated the combination of top k query processing with several optimi-
%sation techniques. They draw the important conclusion that different optimisations
%may be appropriate for queries of different lengths. [Balke et al. 2005] show that top
%k query processing can also be effective in peer-to-peer networks with aggregated
%local indices.

\paragraph{Top k query processing~\cite{suel2003odissea, balke2005progressive, michel2005klee, zhang2005efficient}} % [Suel et al. 2003; Balke et al. 2005; Michel et al. 2005a; Zhang and Suel 2005]
Este método tiene sus inicios en la comunidad en torno a las bases de datos,
particularmente en el trabajo de~\citealt{fagin2001optimal} %[Fagin et al. 2001].
Múltiples variaciones existen, todas con la misma idea básica: es posible
determinar los ``top'' $k$ documentos dadas muchas listas de resultados sin tener
que examinarlas por completo y sin tener que afectar de forma negativa al
rendimiento de la búsqueda. Esto es comúnmente usado en presencia de índices
globales distribuidos en casos donde listas con publicaciones deben ser intersectadas.
Dentro de los algoritmos propuestos, el \textit{threshold algorithm} es el más
popular~\citealt{michel2005klee, suel2003odissea}. %[Michel et al. 2005a; Suelet al. 2003]
Este algoritmo mantiene dos estructuras para la información: una cola con
\textit{peers} para contactar y obtener resultados de búsquedas y una lista con
los actuales ``top'' $k$ resultados. Mientras los \textit{Peers} en la cola van
siendo procesados, cada uno va retornando un set limitado de $k$ resultados de búsqueda con
la forma (documento, puntaje), ordenados en orden descendiente según su puntaje.
Para un índice global distribuido estos son los objetos ``top'' de una lista de
publicaciones para un término en particular. El algoritmo mantiene dos puntajes
para cada documento único: el ``mejor'' y el ``peor''. El ``peor'' puntaje es la suma de
los puntajes de un documento $d$ encontrado en todas las listas de resultados en
donde $d$ aparece. El ``mejor'' puntaje es el peor puntaje más el menor puntaje
(de algún otro documento) encontrado en la lista de resultados en donde $d$
\textit{no} aparece. Como cada lista de resultados es cortada, este último
puntaje forma una cota superior del mejor puntaje posible que puede ser obtenido
por el documento $d$. El $k$ `top'' actual  es formado por los documentos con más altos
puntajes vistos según el ``peor'' puntaje de cada uno. Si el ``mejor'' puntaje
de un documento es menor que el umbral definido, siendo este umbral definido por el ``peor'' puntaje del
documento en posición $k$ en la lista de $k$ ``top'' resultados, el documento no
necesita ser considerado para la lista de ``top'' $k$ elementos.

El algoritmo se basa en la intersección final de sólo los ``top'' $k$ resultados
de cada \textit{peer}, lo cual probablemente entrega un rendimiento equivalente
a intersectar secuencialmente las listas completas. De esta forma se ahorra
ancho de banda y costos de cómputo sin afectar negativamente la calidad de los
resultados.

Un inconveniente de esto es que al buscar por puntajes de documentos requiere
de accesos aleatorios a las listas de resultados~\citealt{suel2003odissea}. %[Suel et al. 2003].
%[Zhang and Suel 2005]
\citealt{zhang2005efficient}~
investigó más tarde la combinación de de búsquedas con procesamiento de los
``top'' $k$ resultados junto con diferentes optimizaciones, indicando
conclusiones importantes sobre que optimizaciones pueden ser apropiadas para
consultas de diferentes largos.
%[Balke et al. 2005] 
\citealt{balke2005progressive}~
muestra que el procesamiento de sólo los ``top'' $k$ resultados puede también
ser efectiva en redes P2P con índices locales agregados.

%— Top k result storing [Tang et al. 2002; Tang and Dwarkadas 2004; Skobeltsyn and
%One step further is only storing the top k results for a query, or term, in the index.
%[Skobeltsyn and Aberer 2006] take this approach as a means to further reduce traffic
%consumption. Related to this is the approach of [Tang and Dwarkadas 2004] that
%store postings only for the top terms in a document. They state that while indexing
%only these top terms might degrade the quality of search results, it likely does not
%matter since such documents would not rank high for queries for the other non-top
%terms they contain anyway.

\paragraph{Top k result storing~\cite{tang2004hybrid, skobeltsyn2006distributed, skobeltsyn2009query, skobeltsyn2007query}} % [Tang et al. 2002; Tang and Dwarkadas 2004; Skobeltsyn and %Aberer 2006; Skobeltsyn et al. 2007; Skobeltsyn et al. 2009]
Una forma de adelantarse un paso adelante es sólo almacenar los ``top'' $k$
resultados para una consulta (o término) dentro del índice.
\citealt{skobeltsyn2006distributed}~%[Skobeltsyn and Aberer 2006]
utiliza este método como medio para reducir aún más el tráfico consumido.
Otro método similar es abordado por~\citealt{tang2004hybrid} %[Tang and Dwarkadas 2004] 
el cual almacena publicaciones sólo de los términos ``top'' de un documento.
Ellos suponen que mientras la indexación de sólo los términos ``top'' puede
degradar la calidad de los resultados de búsqueda, es más probable que esto no
importe, ya que tales documentos de todas formas no quedarían dentro del ranking ``top''
dentro de las consultas realizadas por los otros términos no ``top'' que contienen.



%\chapter{Conclusiones y trabajo futuro}
%\label{sec:conclusiones}
%%# Intro
%#SOA SN
%# SOA P2P

Las soluciones utilizadas por los sitios de redes sociales
centralizados no es compatible con el esquema P2P, debido a las grandes
diferencias de como se distribuye la información y se mantiene la arquitectura
del sistema.

Los sistemas P2P más conocidos son capaces de proveer una capa de red que no
dependa de servidores centralizados, pero no suplen por si solos las
funcionalidades necesarias para cubrir los requerimientos
especializados de una red social P2P. Subsistemas con servicios especializados
son requeridos para el manejo del almacenamiento, búsqueda de la información,
manejo de permisos de usuario, seguridad contra ataques, etc., agregando una
complejidad adicional al diseño de la arquitectura del sistema, junto con
nuevos costos para la red que todavía no han sido investigados a cabalidad.

%que surgen de la falta de la precondición de confianza entre todas las entidades que mantienen el sistema.

%# SOA P2PSN
Analizando las diferentes propuestas de redes sociales P2P actuales, nos
encontramos que no se encuentran desarrolladas con la madurez  necesaria para hacer
frente a todos los problemas identificados. 

%#peerson
En PeerSoN, el sistema de encriptación utilizado es muy básico, requiriendo para poder compartir
un dato a grupo una cantidad de procesamiento proporcional a la
cantidad de usuarios. Además, en caso de modificación del grupo, se requeriría
la reencriptación de todos los archivos compartidos entre sí, con todos los
costos que esto significaría. En el caso del sistema de búsqueda, al basarse en
un DHT sin mecanismos adicionales, no permitiría búsquedas complejas o de más
de una dimensión, dificultando el proceso de establecimiento conexiones entre
la red.

%# Safebook
Safebook requiere de diferentes niveles de confianza entre un usuario y sus
conexiones dentro de la red social, y almacena réplicas de los datos personales
entre los amigos más cercanos, pudiendo haber problemas de disponibilidad de
los datos en los momentos en que éstos no se encuentren en linea. Para la identificación requiere de un servicio 
externo que pueda ser confiado para la asignación de ID únicas. La transmisión
de mensajes a través de la red se realiza anonimizando al que envía y recibe el
mensaje, aumentando el costo y tiempo de envío de cada dato por cada salto que
de entre nodo y nodo.

%#Diaspora
%Diaspora, no siendo un servicio completamente descentralizado, posee el
%problema de la mantención de los servidores públicos en donde se mantiene la
%información, siendo pocos los que hasta el día de hoy se mantienen en linea, y
%surgen dudas sobre el destino de los datos depositados en ellos. En la realidad
%son muy pocos los que tienen el conocimiento y los recursos para levantar su
%propio \textit{pod} para mantenerse en la red social. Estas son algunas de las
%razones por la cual no vemos un gran crecimiento en la cantidad de usuarios en
%la red social, y a menos que la situación cambie, se ve difícil una 
%implementación exitosa basada en su arquitectura.


%para la generación de X llaves unalmacenamiento
%requerido por dato escala proporcionamente a la cantidad de usuarios con el
%cual el dato se quiere compartir. permite compartir información a grupos
%de usuarios

%%%%%%%%%% PROBLEMATICAS MAS IMPORTANTES


%%Dentro de los problemas que se encuentran sin una solución definitiva y
%%abiertos a una mayor investigación, podemos nombrar:
Analizando las problemáticas encontradas, podemos concluir que quedan varios
desafíos complejos por resolver, entre los cuales se encuentran:
%# Problems?

\begin{enumerate}
    %%\item Límite del espacio de almacenamiento que los usuarios están dispuestos a compartir.
    \item Lograr incentivar a los usuarios a que compartan recursos con el sistema.

    Para lograr que el sistema se mantenga auto-sustentable y escale de forma
    adecuada frente a un crecimiento de la información almacenada, pero no de nodos
    dentro de la misma, sumado a que cada usuario tiene la capacidad de elegir
    que recursos comparte con el sistema, podría llegar a producirse una la falta de recursos de
    almacenamiento en la red si se enfrentan una cantidad considerable de
    usuarios egoístas en el sistema. Esto es un problema complejo de resolver
    debido a que el sistema no cuenta con un control central que pueda penalizar el
    comportamiento de un usuario, dejando a los mismos usuarios la decisión de
    penalizar o no este tipo de comportamientos. 

    %%\item Falta de mecanismos eficientes para la realización de búsquedas complejas.
    \item Elaboración de un subsistema de búsqueda de datos almacenados en la
    red social que sea eficiente tanto como en sus búsquedas como en la mantención
    de los índices.

    Como ya fue mencionado anteriormente, lograr ambas propiedades en redes
    basadas en un DHT es complejo debido a la pérdida de localidad de la
    información, y dependiendo de no sólo la cantidad de dimensiones que se
    requieran consultar, si no también del orden que éstas se consulten, el costo
    de búsqueda de la información crece en peores casos de forma exponencial.
    Considerando la inmensa cantidad de datos a indexar y la frecuencia con que
    son modificados, la mantención del índice de datos no es una tarea trivial,
    y requiere de algoritmos inteligentes que minimicen la cantidad de recursos
    utilizados para ello. El problema que esto conlleva, es que es difícil
    tener búsquedas optimizadas manteniendo un índice simple de los archivos,
    por lo que muchas veces el minimizar el costo de mantención aumenta el costo y
    eficiencia de las búsquedas en el sistema.
    %%\item No existen una precondición de confianza entre los nodos.
    %%\item Falta de un sistema de control que asegure la integridad de las operaciones en la red
    %\item Elaboración de un sistema de control que permita asegurar la integridad de las operaciones en la red
    %%\item No se cuenta con sistemas de control para asegurar la privacidad de los datos compartidos en el sistema.
    \item Elaboración de un subsistema de control para asegurar la privacidad de los datos compartidos en el sistema.
    

    Como vimos anteriormente, las soluciones normalmente utilizadas para asegurar
    los datos almacenados dentro de una aplicación no son aplicables en un ambiente
    distribuido P2P. Es por ello que normalmente se utilizan sistemas basados en la
    encriptación para poder mantener segura la información en estos sistemas.
    Ahora, viendo el caso específico de redes sociales, los esquemas de
    encriptación tradicionales no cuentan con todas las funcionalidades deseables
    para la mantención del sistema. 

    Las soluciones tradicionales de utilizar encriptación y llaves compartidas
    entre las partes involucradas no cumplen actualmente con todas las
    funcionalidades requeridas para tratar de forma eficaz y permitir al mismo
    tiempo funcionalidades requeridas en grupos conformados por varios usuarios de
    la red social. El tema es complejo debido a que no existe un método de
    encriptación que su costo constante frente a diversa cantidad de usuarios y
    que permita el manejo fino de permisos para cada archivo compartido.
    
    % esto se saco, pero podría agregarse una seción como estado del arte
   % %encriptacion
   % Analizando los sistemas de encriptación disponibles en la actualidad, la combinación de criptografía simétrica y asimétrica no tiene
   % la suficiente eficiencia y funcionalidad para una red social P2P. CP-ABE es
   % inferior a BE ya que ninguno de los CP-ABE logran soportar múltiples tópicos o
   % atributos, acceso oculto a las estructuras de acceso y bajo costo de
   % almacenamiento y costo computacional al mismo tiempo. Es por eso, que la
   % propuesta de usar BE para la red social P2P por sobre las demás es prometedora,
   % ya que no posee las características negativas de las demás, a pesar de no
   % soporta la funcionalidad de encriptar para un grupo del cual uno no pertenece y
   % para ``amigos de amigos''. 
   % %\begin{itemize}
   %     %\item .
    %\end{itemize}
    %%Además, si consideramos el problema de manejo de 
    %%%\item Manejo de los diferentes privilegios de acceso y lectura para usuarios y grupos de usuarios.
    %%%\item Mantención de datos actualizados en las diferentes réplicas frente a frecuentes salidas y entradas de usuarios
    %%\item Falta de un sistema que permita la interacción entre agentes externos y la red social
    \item Elaboración de un subsistema que permita interactuar con agentes externos a la red social.

    El tema es complejo debido a que requiere de la existencia de una interfaz
    estable que pueda hacer accesible la red P2P sin ser un nodo dentro de
    ella desde el internet. Debido al alto dinamismo de estas redes y la
    descentralización de la misma, no es posible utilizar un sólo nodo como
    puerta de acceso a la red social. Y aún así, aunque se pudiera utilizar un
    nodo especializado para ello o un conjunto de nodos para que agentes
    externos accedan a la red social, se crearía un cuello de botella para el
    acceso a ella con un costo que los usuarios no querrán pagar.
    Además, se necesita poder monitorear y controlar las acciones que los agentes
    externos realizan en la red social, de tal forma de poder bloquear ataques
    y prevenir abusos de éstos hacia los usuarios de la red P2P.

    % esto se saco, pero podría agregarse una seción como estado del arte
    %%integración con otras apps
    %%\paragraph{Integración con otras aplicaciones}
    %La integración con otras aplicaciones también requiere un mayor desarrollo para
    %el establecimiento de confianza entre los usuarios y los entes que las
    %mantienen. De la misma forma que OAuth requiere de una entidad confiable para
    %la autorización de aplicaciones dentro de un esquema centralizado, en el
    %esquema distribuido cada usuario debe hacerse cargo de la autorización, siendo
    %posible bajo DIBBE, pero abriendo la posibilidad de nuevas vulnerabilidades a
    %ataques que no han sido estudiadas todavía.

\end{enumerate}


%%%%%%%%%%%%%%%%% TRABAJO FUTURO?

%# Solutions?

%identificacion
%Los sistemas centralizados (ejemplos?)
%Los sistemas distribuidos (ejemplos?)...
%# Solutions?

%resistencia ataques
%Las defensas utilizadas para defenderse diferentes ataques en un servidor
%combencional también es muy diferente al de una implementación P2P. Mientras
%que las primeras son elaboradas en un ambien

Otros temas que no fueron vistos anteriormente debido a que no se encuentran
asociados directamente con las funcionalidades de las redes sociales, sino que
nacen de vulnerabilidades y comportamientos no esperados de parte de agentes
maliciosos, requeriría de un análisis especializado una vez que se hallan
definido el funcionamiento de los diferentes subsistemas que mantendrían las
funcionalidades del sistema. De todas formas, vale la pena mencionar que para
el manejo y control de ataques y nodos maliciosos dentro de la red social,
existen propuestas para utilizar las relaciones de confianza y colaboración
entre los usuarios, pero no existen estudios concretos para la solución de éste
tipo de problemáticas.


%# between SOA P2P and SOA P2PSN
Finalmente, podemos concluir que utilizando como base a los sistemas P2P
estructurados, estos sistemas podrían extenderse para soportar funcionalidades
más avanzadas, sin tener que sacrificar las características deseables que
poseen (escalabilidad, robustez e independencia de otros sistemas), pero
primero se necesitará resolver las problemáticas existentes. Mientras algunas de
ellas pueden ser resueltas poniendo en práctica ciertas investigaciones, otras
requerirán de más trabajo para poder llegar a ser resueltas.
Aún así, existe la esperanza de que investigaciones en proceso encuentren soluciones que puedan
enfrentar parte de estos problemas y continuar con el desarrollo de una red
social P2P completamente distribuida.
%, y probablemente podramos encontrar formas
%de implementar una solución deseada si es que se sigue manteniendo con una
%tendencia positiva.

%%%## almacenamiento
%%%\paragraph{Almacenamiento}
%%\begin{enumerate}
%% \item Control de usuarios maliciosos o \textit{free riders} que se niegan compartir espacio de almacenamiento con otros usuarios.
%% \item Mejorar la disponibilidad de datos dentro de la red, especialmente los
%%    relacionados con el perfil y datos privados de los usuarios para el
%%    establecimiento de relaciones entre ellos.\\
%%    %### disponibilidad
%%    Considerando las diferentes estrategias para aumentar la disponibilidad de los
%%    datos en la red, éstas requerirán ajustarse para alcanzar un equilibrio entre
%%    los costos y beneficios que ofrezca. Es importante no aumentar innecesariamente
%%    la cantidad de almacenamiento necesario para mantener la disponibilidad alta de
%%    los datos, teniendo en cuenta que el sistema no debe presentar riesgos para la
%%    privacidad de los datos de las personas.
%% \item Mejorar la velocidad de carga y distribución de los datos, junto con el
%%    costo de ancho de banda asociado a ello.\\
%%    %### banda ancha
%%    Son pocas las soluciones que tratan el problema de los costos de ancho de
%%    banda para la distribución de los archivos por la red, siendo un problema que
%%    variará de intensidad dependiendo mayoritariamente de la arquitectura implementada por el
%%    sistema, especialmente por las políticas de transferencia de datos a través
%%    de los nodos que se encuentren bajo NAT y que requieran de nodos intermediarios
%%    para ser alcanzados.
%%%## busquedas
%%%\paragraph{Búsqueda}
%%%## seguridad
%%%\paragraph{Seguridad}
%% \item Sistema de identificación de usuarios que permita manejar ataques del tipo
%%    robo de identidad y clonación de perfiles.
%% \item Disminución de los costos de encriptación y desencriptación del sistema
%%    de encriptación utilizado, idealmente llegando a un costo constante, sin
%%    depender de la cantidad de usuarios para los cuales se encripte el dato.
%% %\item Que la adición y remoción de nuevos usuarios de un grupo no dependa de la cantidad de objetos compartidos en éste.
%% %\item Tamaño del encabezado de encriptación no debe depender de la cantidad destinatarios, de tal forma que el peso de los archivos sea escalable.
%% \item Implementación de operaciones de unión e intersección entre usuarios pertenecientes a diferentes grupos.
%% % no estoy tan seguro:
%% %\item Que sólo los usuarios autorizados puedan acceder a la lista de acceso (de la cabecera de encriptación). 
%%     %\item Habilidad de encriptar para grupos de los cuales uno no es miembro.
%%     %\item Habilidad de encriptar para ``amigos de amigos''.
%%     %\item Habilidad que permita no revelar las estructuras de acceso en la cabecera de los objetos encriptados.
%% \item Sistema de recuperación de llaves de acceso y otro tipo de información
%%    relevante para la identificación del usuario con la red social.
%%\end{enumerate}




\appendix

%\renewcommand{\bibname}{Bibliografía}
\bibliographystyle{plainnat}
\bibliography{../bib/article,../bib/paper,../bib/url}
\end{document}
